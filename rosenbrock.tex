\subsubsection{Функция Розенброка}

Одним из классических тестов для различных алгоритмов оптимизации
является тест Розенброка, заключающийся в минимизации следующей
функции:
\begin{equation}
  \label{eq:rosenbrock}
  f(x, y) = 100(y - x²)² + (1 - x)²
\end{equation}
Точку $(-1.2, 1)$ предлагается взять в качестве начального
приближения.

\paragraph{Выпуклость}

Матрица Гессе этой функции имеет вид:
\begin{equation}
  \label{eq:rosenbrock-hess}
  \begin{pmatrix}
    1200x^2-400y+2 & -400x \\
    -400x & \phm200\phantom{x}
  \end{pmatrix}
\end{equation}

Воспользуемся теоремой \ref{th:convex-f-hess} для проверки функции на
выпуклость. Применим критерий Сильвестра к матрице
\eqref{eq:rosenbrock-hess}. Запишем условия положительности её угловых
миноров:
\begin{align*}
  &\Delta_1 = 1200\,x^2-400\,y+2 > 0 \iff y < 3x^2+\frac{1}{200}\\
  &\Delta_2 = 80\,000\,x^2-80\,000\,y+400 > 0 \iff y < x^2+\frac{1}{200}
\end{align*}
При этом второе ограничение является более сильным. Таким образом,
граница области выпуклости функции Розенброка проходит вдоль кривой
\begin{equation}
  \label{eq:rosenbrock-convex-bound}
  y = x^2 +\frac{1}{200}
\end{equation}
Ниже этой кривой функция Розенброка строго выпукла. Согласно теореме
\ref{th:convex-f-smin}, она имеет в своей области выпуклости
единственную точку минимума.

При значениях $(x, y)$, лежащих выше кривой
\eqref{eq:rosenbrock-convex-bound}, функция Розенброка вогнута, а
потому, в силу теоремы \ref{th:convex-f-nomax}, не может иметь внутри
этой области точек минимума.

\paragraph{Овражность}

Собственные числа матрицы Гессе функции Розенброка в точке $(x, y)$
имеют следующие значения, отличающиеся лишь знаком перед корнем:
\begin{gather*}
  \begin{split}
    \lambda_1(x, y) = -\sqrt{400\,000\,y^2 +
      \left(39\,600-240\,000\,x^2\right)y + 360\,000\,x^4 +
      41\,200\,x^2+9801}- \\
    - 200y+600x^2+101
  \end{split}\\
  \begin{split}
    \lambda_2(x, y) = \sqrt{400\,000\,y^2 +
      \left(39\,600-240\,000\,x^2\right)y + 360\,000\,x^4 +
      41\,200\,x^2+9801}- \\
    - 200y+600x^2+101
  \end{split}
\end{gather*}

В точке начального приближения, лежащей в области выпуклости,
собственные значения матрицы Гессе функции Розенброка приблизительно
равны $23.63$ и $1506.37$, откуда согласно \eqref{eq:gully} получается
степень овражности
\begin{equation}
  \label{eq:rosenbrock-gully-start}
  \eta = \frac{1506.37}{23.63} \approx 63.7
\end{equation}

Исследуемая функция имеет кривую перегиба $y=x^2+\frac{1}{200}$,
разделяющую её области выпуклости и вогнутости. Вдоль этой кривой
степень овражности функции $\eta = \infty$.

\paragraph{Аналитическое решение задачи}

Найдём минимум функции \eqref{eq:rosenbrock} аналитически.

Из необходимых условий стационарности точки имеем:
\begin{align*}
  \pardiff{f}{x} &= 400x^3-400yx+2x-2 = 0 \\
  \pardiff{f}{y} &= 200y-200x^2 = 0
\end{align*}
откуда $x = y = 1$. Найденная точка $A(1, 1)$ не принадлежит кривой
перегиба $y = x^2+\frac{1}{200}$, а потому необходимые условия
экстремума для неё оказываются достаточными. Таким образом, с учётом
теорем \ref{th:convex-f-smin} и \ref{th:convex-f-nomax} точка $A(1, 1)$
является глобальным минимумом функции Розенброка.

Вычисление значения $\eta$ в $A$ даёт
\begin{equation}
  \label{eq:rosenbrock-gully-extr}
  \eta = \frac{1001.6006}{0.3994} \approx 2507.7631
\end{equation}

\paragraph{Минимизация с помощью \relch{} при ручном выборе параметра}

На иллюстрации \ref{fig:rosenbrock-contours} представлены линии уровня
функции Розенброка, которые сильно вытягиваются вдоль всей кривой
$y=x^2$ за счёт близости перегиба и быстрого роста функции в выпуклой
области. С учётом наблюдаемой картины, а также перепада в значениях
степени овражности \eqref{eq:rosenbrock-gully-start} и
\eqref{eq:rosenbrock-gully-extr} можно утверждать, что в ходе работы
алгоритму придётся бороться с сильной овражностью целевой функции.

\begin{figure}[!thb]
  \centering
  \begin{tikzpicture}
    \begin{axis}
      [x=3cm, y=2.7cm,
      xmin=-1.5, xmax=1.5,
      ymin=-1,ymax=2]
      \input{rosenbrock_hq-contours.tkz.tex}
      \addplot[mark=none,thick,black,densely dashed] plot[domain=-1.41:1.41] function{x**2};
      \node[circle,fill=black,scale=0.5,label={right:\contour{white}{$A$}}] at
      (axis cs:1,1) {};
    \end{axis}
  \end{tikzpicture}
  \caption[Функция Розенброка]{Линии уровней $1, 2, 2.5, 4, 5, 50,
    100$ функции Розенброка \eqref{eq:rosenbrock}, её кривая перегиба
    $y=x^2+\frac{1}{200}$ и глобальный минимум в точке $A(1, 1)$}
  \label{fig:rosenbrock-contours}
\end{figure}

В качестве пробных оценок $\eta$ для применения в формуле
\eqref{eq:cheb-param} выберем значения
\eqref{eq:rosenbrock-gully-start} и \eqref{eq:rosenbrock-gully-extr}, тогда:
\begin{gather*}
  s_1 = 1.3 \sqrt{64} \approx 10 \\
  s_2 = 1.3 \sqrt{2507} \approx 66
\end{gather*}
Из эмпирических соображений в качестве $s$ выбиралось ближайшее к
$1.3\sqrt{\eta}$ \emph{чётное} число (в ходе тестов алгоритм
демонстрировал наилучшую сходимость именно при чётных $s$).

\begin{figure}[!thb]
  \centering
  \begin{tikzpicture}
    \begin{axis}[x=2.2cm, y=1.8cm]
      \input{rosenbrock-contours.tkz.tex}
      \input{rosenbrock_relch_-1.2,1_15_200-trace.tkz.tex}
        \node[circle,fill=black,scale=0.25,pin={below right:\contour{white}{$A$}}] at
        (axis cs:1,1) {};
    \end{axis}
  \end{tikzpicture}
  \caption[\relch{} на функции Розенброка, $s=200$]{Минимизация
    функции Розенброка алгоритмом \relch{} при $s=200$}
\end{figure}

\begin{figure}[!thb]
  \centering
  \begin{tikzpicture}
    \begin{axis}[x=2.2cm, y=1.8cm]
      \input{rosenbrock-contours.tkz.tex}
      \input{rosenbrock_gdrelch_-1.2,1__0.0019-trace.tkz.tex}
        \node[circle,fill=black,scale=0.25,pin={below right:\contour{white}{$A$}}] at
        (axis cs:1,1) {};
    \end{axis}
  \end{tikzpicture}
  \caption[\gdrelch{} на функции Розенброка, $s=0.0019$]{Минимизация
    функции Розенброка алгоритмом \gdrelch{} при $h=0.0019$}
\end{figure}