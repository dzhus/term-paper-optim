\subsubsection{Функция Розенброка}

Одним из классических тестов для различных алгоритмов оптимизации
является тест Розенброка, заключающийся в минимизации следующей
функции:
\begin{equation}
  \label{eq:rosenbrock}
  \tag{$\rho$-\theequation}
  f(x, y) = 100(y - x²)² + (1 - x)²
\end{equation}
Точку $(-1.2, 1)$ предлагается взять в качестве начального
приближения.

\paragraph{Выпуклость}

Матрица Гессе этой функции имеет вид:
\begin{equation}
  \label{eq:rosenbrock-hess}
  \begin{pmatrix}
    1200x^2-400y+2 & -400x \\
    -400x & \phm200\phantom{x}
  \end{pmatrix}
\end{equation}

Применим критерий Сильвестра к матрице \eqref{eq:rosenbrock-hess} для
исследования её на определённость. Запишем условия положительности
угловых миноров:
\begin{align*}
  &\Delta_1 = 1200\,x^2-400\,y+2 > 0 \iff y < 3x^2+\frac{1}{200}\\
  &\Delta_2 = 80\,000\,x^2-80\,000\,y+400 > 0 \iff y < x^2+\frac{1}{200}
\end{align*}
При этом второе ограничение является более сильным. Значит, граница
области выпуклости функции Розенброка проходит вдоль кривой
\begin{equation}
  \label{eq:rosenbrock-convex-bound}
  y = x^2 +\frac{1}{200}
\end{equation}
Ниже этой кривой функция Розенброка строго выпукла. Согласно теореме
\ref{th:convex-f-smin}, она имеет в своей области выпуклости
единственную точку минимума.

При значениях $(x, y)$, лежащих выше кривой
\eqref{eq:rosenbrock-convex-bound}, функция Розенброка вогнута, а
потому в силу теоремы \ref{th:convex-f-nomax} не может иметь внутри
этой области точек минимума.

\paragraph{Аналитическое решение}

Найдём минимум функции \eqref{eq:rosenbrock} аналитически.

Из необходимых условий стационарности точки имеем:
\begin{align*}
  \pardiff{f}{x} &= 400x^3-400yx+2x-2 = 0 \\
  \pardiff{f}{y} &= 200y-200x^2 = 0
\end{align*}
откуда $x = y = 1$. Найденная точка $A(1, 1)$ лежит в области
выпуклости, а потому является глобальным минимумом функции Розенброка.

\paragraph{Овражность}

В точке начального приближения, лежащей в области выпуклости,
собственные значения матрицы Гессе функции Розенброка приблизительно
равны $23.63$ и $1506.37$, откуда согласно \eqref{eq:gully} получается
степень овражности
\begin{equation}
  \label{eq:rosenbrock-gully-start}
  \eta = \frac{1506.37}{23.63} \approx 63.7
\end{equation}

Вычисление значения $\eta$ в $A$ даёт
\begin{equation}
  \label{eq:rosenbrock-gully-extr}
  \eta = \frac{1001.6006}{0.3994} \approx 2507.7631
\end{equation}

Исследуемая функция имеет кривую перегиба $y=x^2+\frac{1}{200}$,
разделяющую её области выпуклости и вогнутости. Вдоль этой кривой
степень овражности функции $\eta = \infty$.

На иллюстрации \ref{fig:rosenbrock-contours} представлены линии уровня
функции Розенброка, которые сильно вытягиваются вдоль всей кривой
$y=x^2$ за счёт близости перегиба и быстрого роста функции в выпуклой
области.

\begin{figure}[thb]
  \centering
  \begin{tikzpicture}
    \begin{axis}
      [x=3cm, y=2.7cm,
      xmin=-1.5, xmax=1.5,
      ymin=-1,ymax=2]
      \input{rosenbrock_hq-contours.tkz.tex}
      \addplot[mark=none,thick,black,densely dashed] plot[domain=-1.41:1.41] function{x**2};
      \node[circle,fill=black,scale=0.25,label={right:\contour{white}{$A$}}] at
      (axis cs:1,1) {};
    \end{axis}
  \end{tikzpicture}
  \caption[Функция Розенброка]{Линии уровней $1, \dotsc 100$ функции
    Розенброка \eqref{eq:rosenbrock}, её кривая перегиба
    $y=x^2+\frac{1}{200}$ и глобальный минимум в точке $A(1, 1)$}
  \label{fig:rosenbrock-contours}
\end{figure}

С учётом наблюдаемой картины, а также перепада в значениях
степени овражности \eqref{eq:rosenbrock-gully-start} и
\eqref{eq:rosenbrock-gully-extr} можно утверждать, что в ходе работы
алгоритму придётся бороться с сильной овражностью целевой функции.

\paragraph{Минимизация с помощью \relch{} при ручном выборе параметра}

В качестве пробных оценок $\eta$ для применения в формуле
\eqref{eq:cheb-param} выберем значения
\eqref{eq:rosenbrock-gully-start} и \eqref{eq:rosenbrock-gully-extr}, тогда:
\begin{gather*}
  s_1 = 1.3 \sqrt{64} \approx 10 \\
  s_2 = 1.3 \sqrt{2507} \approx 66
\end{gather*}
Из эмпирических соображений в качестве $s$ выбиралось ближайшее к
$1.3\sqrt{\eta}$ \emph{чётное} число (в ходе тестов алгоритм
демонстрировал наилучшую сходимость именно при чётных $s$).

На иллюстрации \ref{fig:rosenbrock-relch} представлены результаты
работы \relch{} на функции Розенброка. Видно, что при малом значении
$s=10$ \relch{} быстро попадает в окрестность минимума, но испытывает
трудности при продвижении по дну оврага. С ростом $s$ до 66 алгоритм
крупными шагами стремительно движется вдоль оврага к точке минимума.
Протестирована также работа алгоритма с $s=200$. Высокая скорость
сходимости с таким значением соответствует ожиданиям.

\paragraph{Применение \gdrelch{}}

Описанный в разделе \ref{sec:cheb-param} двухэтапный метод \gdrelch{}
применим к функции Розенброка. На иллюстрации
\ref{fig:rosenbrock-gdrelch} показано, как сначала метод \gd{}
зацикливается в овраге, что соответствует стабилизации
\eqref{eq:gdrelch-stabilization}, после чего включается обычный
\relch{} со значением параметра, определённым на первом этапе. В
данном случае стабилизацию \eqref{eq:gdrelch-stabilization} удалось
обеспечить при шаге $h=0.003$. В таблице \ref{tab:rosenbrock-gdrelch},
содержащей информацию о ходе вычисления, заметен участок в окрестности
точки $(-0.48, 0.23)$, у которой происходит стабилизация нормы
градиента целевой функции на уровне $\approx 2.13$. После этого по формуле
\eqref{eq:gdrelch-param} определяется $s$ и включается алгоритм
\relch{}, который быстро сводит значение функции до уровня порядка
$10^{-13}$.

\begin{figure}[thb]
  \centering
  \begin{tikzpicture}
    \begin{axis}[x=2.7cm, y=2cm]
      \input{rosenbrock-contours.tkz.tex}
      \input{rosenbrock_gdrelch_-1.2,1__0.003-trace.tkz.tex}
      \node[circle,fill=black,scale=0.25,pin={below
        right:\contour{white}{$A$}}] at (axis cs:1,1) {};
      \node[circle,scale=3,draw=black,densely dashed] at (axis
      cs:-0.48,0.23) {};
    \end{axis}
  \end{tikzpicture}
  \caption[\gdrelch{} на функции Розенброка, $h=0.003$]{Минимизация
    функции Розенброка алгоритмом \gdrelch{} при значении шага
    $h=0.003$.}
  \label{fig:rosenbrock-gdrelch}
\end{figure}

\begin{table}[hpb]
  \centering
  \input{rosenbrock_gdrelch_-1.2,1_50_0.003-trace.tbl.tex}
  \caption{Минимизация
    функции Розенброка алгоритмом \gdrelch{} при $h=0.003$.
    Зацикливание \gd{} происходит в окрестности точки $(-0.48, 0.23)$.}
  \label{tab:rosenbrock-gdrelch}
\end{table}

\begin{figure}[p]
  \center
  \pgfplotsset{every axis/.append style={x=2cm,y=1.5cm}}
  \subfigure[$s=10$]{
    \begin{tikzpicture}
      \begin{axis}[]
        \input{rosenbrock-contours.tkz.tex}
        \input{rosenbrock_relch_-1.2,1_40_10-trace.tkz.tex}
      \end{axis}      
    \end{tikzpicture}}
  \hskip 0.5cm
  \subfigure[$s=66$]{
    \begin{tikzpicture}
      \begin{axis}[]
        \input{rosenbrock-contours.tkz.tex}
        \input{rosenbrock_relch_-1.2,1_40_66-trace.tkz.tex}
      \end{axis}      
    \end{tikzpicture}}
  \\\vskip 0.5cm
  \subfigure[$s=200$]{
    \begin{tikzpicture}
      \begin{axis}[]
        \input{rosenbrock-contours.tkz.tex}
        \input{rosenbrock_relch_-1.2,1_20_200-trace.tkz.tex}
      \end{axis}      
    \end{tikzpicture}}
  \caption[\relch{} на функции Розенброка]{Минимизация функции
    Розенброка алгоритмом \relch{} при различных значениях параметра
    $s$}
  \label{fig:rosenbrock-relch}
\end{figure}
