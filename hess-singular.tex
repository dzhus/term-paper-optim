\subsubsection{Многопараметрическая оптимизация}

В качестве дополнительного примера приведём результат оптимизации
функции четырёх переменных, заданной следующим образом:
\begin{equation}
  \label{eq:multivar}
  f(x) = (x₁ + 10x₂)² + 5(x₃ - x₄)² + (x₂-2x₃)⁴+10(x₁-x₄)⁴
\end{equation}

Особенность функции \eqref{eq:multivar} состоит в том, что она имеет
вырожденную матрицу Гессе в точке минимума.

\begin{table}[h]
  \centering
  \pgfkeys{/pgfplots/table/font={\scriptsize}}
  \input{hess-singular_relch_-3,-1,-3,-1_30_500-trace.tbl.tex}
  \caption[\relch{} на функции четырёх переменных]{Минимизация функции \eqref{eq:multivar} методом \relch{} при $s=100$}
  \label{tab:multivar}
\end{table}

Получаемый результат близок к эталонному — точке $(0,0,0,0)$,
приведённой в \cite{himmelblau75}.
