\section{Предварительные сведения}
В данном разделе изложены теоретические сведения общего характера,
которые будут использоваться в дальнейших частях работы.

\subsection{Выпуклость и вогнутость}
\label{sec:convexity}
Интуитивно ясное понятие выпуклости опирается на наглядные
геометрические представления, но вместе с тем имеет и чисто
аналитическую формулировку.

Более подробное рассмотрение выпуклых множеств и функций представлено
в \cite{polovinkin04}, \cite{kolmogorov72} и \cite{fikhtengolz03}.

\begin{dfn}
  \label{dfn:convex-set}
  Множество $D \subset \set{R}^n$ называется \neword{выпуклым}, если
  для $\forall x, y \in D, \forall \alpha \in [0;1]$ выполняется
  \begin{equation*}
    \alpha x + (1-\alpha) y \in D
  \end{equation*}
\end{dfn}
\begin{dfn}
  Если в определении \ref{dfn:convex-set} для $\forall \alpha \in
  (0;1)$ условие выполняется в смысле принадлежности ко
  \emph{внутренности} $D$, то есть
  \begin{equation*}
    \alpha x + (1-\alpha) y \in D \setminus \partial D,
  \end{equation*}
  то множество $D$ называется \neword{строго выпуклым}.
\end{dfn}
\begin{dfn}
  Множество называется \neword{вогнутым}, если оно не является
  выпукло.
\end{dfn}
Таким образом, отрезок, соединяющий две точки выпуклого множества,
целиком принадлежит этому множеству, а две точки строго выпуклого
множества соединяются отрезком, внутренние точки которого не
принадлежат границе множества (рис. \ref{fig:convex-sets}).

\begin{figure}[!thb]
  \centering
  \begin{tikzpicture}
    \coordinate (A) at (-2, 0);
    \coordinate (B) at (2, -0.5);
    \coordinate (C) at (0, -2.5);

    % Convex
    \draw[thick] (A) -- ++(1,1.2) -- ++(1.3,-1) -- ++(-0.7, -2) -- ++(-1.5, 0.2) -- cycle;
    \draw ($ (A) + (1.25,0.75) $) -- ++(0.5, -1.5);
    \node[] at ($ (A) + (0.5, -1) $) {$A$};
    
    % Strictly convex
    \draw[thick] (B) to[out=80, in=170] ($ (B) + (1.5, 1.5) $)
                     to[out=350, in=70] ($ (B) + (2.5, -0.5) $)
                     to[out=250, in=260] (B);
    \draw ($ (B) + (0.7, 1) $) -- ++(1, -1);
    \node[] at ($ (B) + (1, -0.5) $) {$B$};                     

    % Concave
    \draw[densely dashed] ($ (C) + (1,-0.1) $) -- ++(+0.85, -2);
    % Concave set is drawn twice because of clipping
    \draw[thick] (C) -- ++(2.5,0.25) --++(-2,-1.25) -- ++(3.5, -1.8) to[out=175,in=-70]
    ($ (C) - (0.1, 2) $) to[out=110, in=-100] (C);
    \clip (C) -- ++(2.5,0.25) --++(-2,-1.25) -- ++(3.5, -1.8) --
    ($ (C) - (0.1, 2) $) -- cycle;
    \draw ($ (C) + (1,-0.1) $) -- ++(+0.85, -2);
    \node[] at ($ (C) + (0.5, -1.5) $) {$C$};
  \end{tikzpicture}
  \caption[Выпуклость множеств]{Выпуклое множество $A$, строго
    выпуклое множество $B$ и вогнутое множество $C$}
  \label{fig:convex-sets}
\end{figure}


\begin{dfn}
  \label{dfn:convex-f}
  Функция $f(x), x \in \set{R}^n$ называется \neword{выпуклой}, если
  для $\forall x, y \in \set{R}^n, \forall \alpha \in [0;1]$
  выполняется неравенство
  \begin{equation*}
    f(\alpha x + (1-\alpha)y) \leq \alpha f(x) + (1-\alpha) f(y)
  \end{equation*}
\end{dfn}
\begin{dfn}
  \label{dfn:strictly-convex-f}
  Если в определении \ref{dfn:convex-f} для $\forall \alpha \in (0;1)$
  неравенство выполняется строго, функция $f(x)$ называется
  \neword{строго выпуклой}.
\end{dfn}
\begin{rem}
  \label{rem:lin-f-convex}
  Линейные функции выпуклы, но не строго выпуклы.
\end{rem}
\begin{dfn}
  Функция $f(x)$ называется \neword{вогнутой}, если $-f(x)$ является
  выпуклой функцией.
\end{dfn}

Таким образом, геометрический смысл определений \ref{dfn:convex-f} и
\ref{dfn:strictly-convex-f} ясен: функция выпукла, если выпукла
область над её графиком, и строго выпукла, если область над её
графиком строго выпукла (рис. \ref{fig:convex-function}).

\begin{figure}[!h]
  \centering
  \begin{tikzpicture}
    \begin{axis}[
      xlabel=$x$, ylabel=$f(x)$,
      xtick={-2, -1, 1, 2},
      xmin=-2.5, xmax=2.5,
      ymin=-0.9, ymax=4.9,
      axis x line=middle,
      axis y line=middle,
      axis on top,
      pattern color=gray!50]
      
      
      \draw[draw=none, pattern=north east lines] (axis cs:-1.9, 3.61) 
                       parabola bend (axis cs:0, 0)
                       (axis cs:1.9, 3.61) -- cycle;
      \addplot[thick,mark=none] plot[domain=-2:2] function{x**2};
      \addplot[mark=none] plot[domain=0.25:2] function{2*x-1};
      \addplot[mark=none] plot[domain=-1.5:-0.1] function{-x-0.25};
    \end{axis}
  \end{tikzpicture}
  \caption[Выпуклая функция]{Выпуклая функция $f(x)=x^2$ и касательные
    прямые к ней в точках $\left(-\frac{1}{2}, \frac{1}{4}\right)$ и
    $(1, 1)$}
  \label{fig:convex-function}
\end{figure}


Следущие теоремы оказываются полезными при доказательстве выпуклости
функций и множеств.

\begin{thm}
  \label{th:convex-f-sum}
  Сумма выпуклых функций также выпукла.
\end{thm}

\begin{thm}
  \label{th:convex-f-diff2}
  Для выпуклости непрерывно-дифференцируемой функции $f(x)$ необходимо
  и достаточно, чтобы её вторая производная $f''(x)$ была
  неотрицательна:
  \begin{equation*}
    f''(x) \geq 0
  \end{equation*}
\end{thm}

\begin{thm}
  \label{th:convex-f-tangent}
  Для выпуклости непрерывно-дифференцируемой функции $f(x)$,
  необходимо и достаточно, чтобы её график всем точками лежал над
  любой своей касательной или на ней. Для строгой выпуклости функции
  её график должен лежать строго над любой своей касательной.
\end{thm}

Теорема \ref{th:convex-f-diff2} обобщается на случай функции
нескольких переменных следующим образом.
\begin{thm}
  \label{th:convex-f-hess}
  Для выпуклости непрерывно-дифференцируемой функции $f(x), x \in
  \set{R}^n$ необходимо и достаточно, чтобы её матрица Гессе была
  положительно определена.
\end{thm}

Обобщение теоремы \ref{th:convex-f-tangent} получается с помощью
рассмотрения касательной гиперплоскости вместо касательной прямой.

\begin{thm}
  \label{th:convex-set}
  Если $g_j(x), j=\overline{1,m}$ — выпуклые функции, то множество
  $D$, заданное условиями $g_j(x) \leq b_j$, является выпуклым.
\end{thm}
