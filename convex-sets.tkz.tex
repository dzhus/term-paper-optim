\begin{figure}[!thb]
  \centering
  \begin{tikzpicture}
    \coordinate (A) at (-2, 0);
    \coordinate (B) at (2, -0.5);
    \coordinate (C) at (0, -2.5);

    % Convex
    \draw[thick] (A) -- ++(1,1.2) -- ++(1.3,-1) -- ++(-0.7, -2) -- ++(-1.5, 0.2) -- cycle;
    \draw ($ (A) + (1.25,0.75) $) -- ++(0.5, -1.5);
    \node[] at ($ (A) + (0.5, -1) $) {$A$};
    
    % Strictly convex
    \draw[thick] (B) to[out=80, in=170] ($ (B) + (1.5, 1.5) $)
                     to[out=350, in=70] ($ (B) + (2.5, -0.5) $)
                     to[out=250, in=260] (B);
    \draw ($ (B) + (0.7, 1) $) -- ++(1, -1);
    \node[] at ($ (B) + (1, -0.5) $) {$B$};                     

    % Concave
    \draw[densely dashed] ($ (C) + (1,-0.1) $) -- ++(+0.85, -2);
    % Concave set is drawn twice because of clipping
    \draw[thick] (C) -- ++(2.5,0.25) --++(-2,-1.25) -- ++(3.5, -1.8) --
    ($ (C) - (0.1, 2) $) -- cycle;
    \clip (C) -- ++(2.5,0.25) --++(-2,-1.25) -- ++(3.5, -1.8) --
    ($ (C) - (0.1, 2) $) -- cycle;
    \draw ($ (C) + (1,-0.1) $) -- ++(+0.85, -2);
    \node[] at ($ (C) + (0.5, -1.5) $) {$C$};
  \end{tikzpicture}
  \caption[Выпуклость множеств]{Выпуклое множество $A$, строго
    выпуклое множество $B$ и вогнутое множество $C$}
  \label{fig:convex-sets}
\end{figure}
