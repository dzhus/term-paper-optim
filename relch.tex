\section{Безусловная оптимизация}
\label{sec:relch}

В этом разделе представлена общая схема градиентных методов,
рассмотрено понятие функции релаксации и описан метод
многопараметрической оптимизации с чебышёвскими функциями релаксации
\relch{}, предложенный И. Г. Черноруцким в
\cite{chernorutsky04}.

\subsection{Теоретические сведения}

\subsubsection{Постановка задачи}

Настоящий раздел посвящён решению следующей задачи \emph{безусловной}
минимизации:
\begin{equation}
  \label{eq:optim-problem-form}
  f(x) \to \min,\quad x \in \set{R}^n,\, f(x) \in C^2(\set{R}^n)
\end{equation}

Рассматриваемые итерационных методов построены по общей схеме,
выражаемой следующей рабочей формулой, которая определяет способ
перехода к новому приближению $x^{k+1}$ точки минимума на очередной
итерации:
\begin{equation}
  \label{eq:iter-method}
  x^{k+1} = \phi_k(x^k)
\end{equation}

При этом используются следующие критерии останова процесса
\eqref{eq:iter-method} на $k$-м шаге:
\begin{enumerate}
\item Близость к нулю\footnote{Здесь и в дальнейшем, $\epsilon>0$ —
    некоторое наперёд заданное малое число.} нормы градиента функции:
  \begin{equation*}
  \norm{f'(x^k)} < \epsilon
\end{equation*}

\item Близость соседних приближений:
  \begin{equation*}
    \norm{x^{k+1}-x^k} < \epsilon
  \end{equation*}
  
\item Близость значений целевой функции $f(x)$ в точках соседних
  приближений:
  \begin{equation*}
    \norm{f(x^{k+1}) - f(x^k)} < \epsilon
  \end{equation*}
\item Остановка процесса после выполнения предельного количества
  итераций:
  \begin{equation*}
    k = k_{\max}
  \end{equation*}
  
\end{enumerate}

\subsubsection{Градиентные методы}

\begin{dfn}
  \neword{Градиентными} называются итерационные методы оптимизации со
  следующей рабочей формулой:
  \begin{equation}
    \label{eq:grad-methods}
    x^{k+1} = x^k - H_k\left(G_k, h_k\right) g_k
  \end{equation}
  здесь $H_k$ — некоторая функция от матрицы Гессе $G_k = G(x^k) =
  f''(x^k)$ и параметра $h_k$, а $g_k = g(x^k) = f'(x^k)$ — градиент
  функции в точке $x_k$.
\end{dfn}

При постоянных значениях параметров на различных шагах итерации
соответствующие индексы $k$ в формуле \eqref{eq:grad-methods}
опускаются.

Предполагается, что в некоторой $\epsilon_k$-окрестности $\left\{
  \norm{x-x^k} < \epsilon_k\right\}$ точки $x^k$ функция $f(x)$
аппроксимируется гиперболоидом:
\begin{equation}
  \label{eq:sqr-approx}
  f(x) \approx \frac{1}{2}\scalmult{G_k x, x} - \scalmult{a_k,x} + b_k \approx \frac{1}{2}\scalmult{G_k x, x}
\end{equation}

Ставится задача построения таких матричных функций $H_k$, при которых
выполняется \neword{условие релаксации} процесса
\begin{equation}
  \label{eq:relax-cond}
  f(x^{k+1}) < f(x^k)
\end{equation}
При этом требуется, чтобы величина нормы $\norm{x^{k+1}-x^{k}}$ была
ограничена сверху лишь параметром $\epsilon_k$, который характеризует
область справедливости локальной квадратичной модели
\eqref{eq:sqr-approx}.

\subsubsection{Функция релаксации}
\label{sec:relax}
\begin{dfn}
  \neword{Функцией релаксации} называется скалярная функция
  \begin{equation}
    \label{eq:relax-fun}
    R_h(\lambda) = 1 - H(\lambda, h)\lambda,\quad \lambda,h \in \set{R}
  \end{equation}
  где $H(\lambda, h)$ — скалярный аналог матричной функции $H(G, h)$
  из формулы \eqref{eq:grad-methods}.
\end{dfn}
В дальнейшем индекс $h$ у функции релаксации $R_h(\lambda)$ иногда
будем опускать.

\begin{dfn}
  \neword{Множителями релаксации} для точки $x^k$ называются значения
  функции релаксации на спектре матрицы Гессе:
  \begin{equation}
    \label{eq:relax-fac}
    R_h(\lambda_i),\, \lambda_i \in \Sp{G_k}
  \end{equation}
\end{dfn}

Благодаря следующей теореме, функция релаксации используется для
анализа различных градиентных методов.

\begin{thm}
  \label{thm:relax-thm}
  Для выполнения условия релаксации \eqref{eq:relax-cond} при любых
  $x^k$ необходимо и достаточно, чтобы
  \begin{equation}
    \label{eq:relax-thm}
    \begin{aligned}
      & \abs{R(\lambda_i)} \geq 1 & \lambda_i& < 0 \\
      & \abs{R(\lambda_i)} \leq 1 & \lambda_i& > 0\\
      &&i& = \overline{1, n}
    \end{aligned}
  \end{equation}
\end{thm}

\begin{figure}[tb]
  \centering
  \label{fig:relax-thm}
  \begin{tikzpicture}
    \begin{axis}[
      xlabel=$\lambda$,xtick=\empty,
      ylabel=$R(\lambda)$,ytick={-1,1},
      y tick label style={anchor=north east},
      y tick style={draw=none},
      xmin=-3.2,xmax=3.2,ymin=-2.5,ymax=2.5,
      axis x line=middle,
      axis y line=middle,
      axis on top,
      pattern color=gray!50,
      ]

      % Forbidden areas
      \addplot[draw=none,mark=none,pattern=north west lines] coordinates{
        (-3,1) (0,1) (0,-1) (-3,-1)} \closedcycle;
      \addplot[draw=none,mark=none,pattern=north west lines] coordinates{
        (0,1) (3,1) (3,2) (0,2)} \closedcycle;
      \addplot[draw=none,mark=none,pattern=north west lines] coordinates{
        (0,-1) (3,-1) (3,-2) (0,-2)} \closedcycle;

      % Function
      \addplot[mark=none,smooth,thick] coordinates{
        (-2.5, 2)
        (-2.2, 1.1)
        (-1.8, 1.6)
        (-1.1, 1.3)
        (-0.5, 1.4)
        (0, 1)
        (0.5, -0.8)
        (1.7, 0.1)
        (2.5, 0.2)};
      
      % Dashed lines at forbidden area boundaries
      \addplot[dashed,mark=none] coordinates{
        (-3,1)
        (3,1)};
      \addplot[dashed,mark=none] coordinates{
        (-3,-1)
        (3,-1)};
    \end{axis}
  \end{tikzpicture}
  \caption[Подходящая функция релаксации]{Функция релаксации,
    удовлетворяющая условиям теоремы \ref{thm:relax-thm}}
  \label{fig:relax-thm}
\end{figure}


Скорость релаксации может быть оценена с использованием следующего
соотношения:
\begin{equation}
  \label{eq:relax-speed}
  2\abs{f(x^{k+1})-f(x^k)}=\sum_{\lambda_i^+>0} \left\{\xi_{i,k}^2
    \lambda_i^+ [1-R^2(\lambda_i^+)]\right\} + \sum_{\lambda_i^-<0} \left\{\xi_{i,k}^2
    \abs{\lambda_i^-} [R^2(\lambda_i^-)-1]\right\}
\end{equation}
здесь $\lambda_i^+$ и $\lambda_i^-$ — положительные и отрицательные
собственные значения матрицы $G_k$. Коэффициенты $\xi_{i,k}$
происходят из разложения $x^k=\suml_i{\xi_{i,k}u^i}$ по собственным
векторам $u^i$ матрицы Гессе.

Таким образом, эффективными оказываются методы, функция релаксации
которых в положительной области значений $\lambda$ как можно
\emph{меньше} уклоняется от нуля, а при отрицательных $\lambda$
становится как можно большей по модулю
(см. рис. \ref{fig:relax-thm}).

\subsubsection{Метод \gd{}}
\label{sec:gd}

В качестве примера рассмотрим\footnote{Этот метод уже упоминался в
  разделе \ref{sec:problems-ill}.} метод \neword{простого градиентного
  спуска} (\gd{}), рабочая формула которого имеет вид:
\begin{equation}
  \label{eq:gd-workhorse}
  x^{k+1}=x^k-hg_k
\end{equation}
где $h\in\set{R}$ — некоторое фиксированное число (обычно малое),
называемое \neword{шагом}.

Рассмотрим функцию релаксации метода \gd{}:
\begin{equation}
  \label{eq:gd-relax}
  R(\lambda) = 1 - \lambda h
\end{equation}

\begin{figure}[!h]
  \centering
  \label{fig:relax-thm}
  \begin{tikzpicture}
    \begin{axis}[
      xlabel=$\lambda$, ylabel=$R(\lambda)$,
      xtick={2}, xticklabels={$\frac{1}{h}$},
      ytick=\empty,
      x tick style={black},
      x tick label style={anchor=south west},
      extra x ticks={4},
      extra x tick labels={$M^*$},
      extra x tick style={grid=major},
      xmin=-3.2, xmax=6.5, ymin=-2.5, ymax=3.5,
      axis x line=middle,
      axis y line=middle,
      axis on top,
      mark=none,
      pattern color=gray!50,
      ]

      % Forbidden areas
      \addplot[draw=none,pattern=north west lines] coordinates{
        (-3,1) (0,1) (0,-1) (-3,-1)} \closedcycle;
      \addplot[draw=none,pattern=north west lines] coordinates{
        (0,1) (6,1) (6,2) (0,2)} \closedcycle;
      \addplot[draw=none,pattern=north west lines] coordinates{
        (0,-1) (6,-1) (6,-2) (0,-2)} \closedcycle;

      % Function
      \addplot[mark=none] coordinates{
        (6, -2)
        (-2, 2)};
      
      % Dashed lines at forbidden area boundaries
      \addplot[dashed] coordinates{
        (-3,1)
        (6,1)};
      \addplot[dashed] coordinates{
        (-3,-1)
        (6,-1)};
    \end{axis}
  \end{tikzpicture}
  \caption{Функция релаксации $R(\lambda)=1-\lambda h$ метода простого
    градиентного спуска}
  \label{fig:relax-thm}
\end{figure}


Из её графика на рисунке \ref{fig:gd-relax} видно, что метод применим
лишь в том случае, когда собственные значения матрицы Гессе
оптимизируемой функции не превосходят некоторого критического значения
$M^*=\frac{2}{h}$. Кроме того, в области значений $\lambda$, близких к
нулю или $M^*$, согласно \eqref{eq:relax-speed} скорость релаксации
сильно снижается. 

Так объясняется не самая широкая область применимости \gd{} и проблемы
градиентного спуска, обозначенные в разделе \ref{sec:problems-ill}.

Имеются различные модификации метода \gd{}. Одной из наиболее простых
является метод \rgd{} — простой градиентный спуск \neword{с дроблением
  шага}. В \rgd{} изначально $h$ полагается достаточно большим, но в
случае нарушения условия релаксации \eqref{eq:relax-cond} на
какой-либо итерации $h$ делится пополам до тех пор, пока
\eqref{eq:relax-cond} вновь не станет выполняться.

\subsection{Описание метода \relch{}}

\subsubsection{Использование полиномов Чебышёва\\
  в функции релаксации}

Рассмотрим смещённые полиномы Чебышёва второго рода, которые
определяются согласно рекуррентному соотношению:
\begin{align}
  \label{eq:chebyshev}
  P_0(\lambda) &= 0 \\
  P_1(\lambda) &= 1 \\
  P_k(\lambda) &= 2(1-2\lambda)P_{k-1}(\lambda) - P_{k-2}(\lambda),\, k
  \geq 2
\end{align}

Функциональная последовательность таких полиномов обладает важным
свойством, а именно
\begin{equation}
  \label{eq:cheb-limit}
  \lim_{s\to\infty}{\frac{P_s(\lambda)}{s}} \to 0 \text{ на } (0;1)
\end{equation}
причём сходимость равномерная.

Предположим, что собственные числа матрицы Гессе целевой функции
$f(x)$ задачи \eqref{eq:optim-problem-form} в положительной области
спектра не превосходят 1 (для этого достаточно в рассмотрении считать
градиент и матрицу Гессе нормированными). Тогда использование
следующей функции в качестве релаксационной:
\begin{equation}
  \label{eq:cheb-relax}
  R_s(\lambda) = \frac{P_s(\lambda)}{s}
\end{equation}
позволяет обеспечить, согласно \eqref{eq:relax-speed}, сколь угодно
быструю релаксацию. На иллюстрации \ref{fig:cheb-relax} приведены
графики функции $R_s(\lambda)$ вблизи $[0;1]$ для нескольких значений
$s$.

\begin{figure}[!h]
  \centering
  \begin{tikzpicture}
    \begin{axis}[x=7cm, y=2cm,
      xlabel=$\lambda$, ylabel=$R_s(\lambda)$,
      xtick=\empty, ytick={-1, 1},
      y tick label style={anchor=north east},
      y tick style={draw=none},
      extra x ticks={1},
      extra x tick style={grid=major},
      xmin=-0.2, xmax=1.4, ymin=-2, ymax=2,
      axis x line=middle,
      axis y line=middle,
      axis on top,
      pattern color=gray!50
      ]

      \addplot[mark=o] plot[domain=-0.2:1.1] function{(16*x**2-16*x+3)/3};
      \addplot[mark=square] plot[domain=-0.2:1.1] function{(-64*x**3+96*x**2-40*x+4)/4};
      \addplot[mark=triangle] plot[domain=-0.2:1.1] function{(256*x**4-512*x**3+336*x**2-80*x+5)/5};
      \legend{$L=3$, $L=4$, $L=5$}


      % Forbidden areas
      \addplot[draw=none,pattern=north west lines] coordinates{
        (-0.2,1) (0,1) (0,-1) (-0.2,-1)} \closedcycle;
      \addplot[draw=none,pattern=north west lines] coordinates{
        (0,1) (1.2,1) (1.2,1.4) (0,1.4)} \closedcycle;
      \addplot[draw=none,pattern=north west lines] coordinates{
        (0,-1) (1.2,-1) (1.2,-1.4) (0,-1.4)} \closedcycle;
      
      % Dashed lines at forbidden area boundaries
      \addplot[dashed] coordinates{
        (-0.2,1)
        (1.2,1)};
      \addplot[dashed] coordinates{
        (-0.2,-1)
        (1.2,-1)};
    \end{axis}
  \end{tikzpicture}
  \caption{Функции релаксации на основе полиномов Чебышёва}
  \label{fig:cheb-relax}
\end{figure}


Уже при $s=8$ значение $R_s(\lambda)$ не превосходит по модулю $0.23$
на отрезке $[0.025; 0.975]$, обеспечивая хорошее подавление слагаемых
\eqref{eq:relax-speed}, соответствующих большому диапазону
положительных собственных чисел.

При этом стремление $R_s(\lambda)$ к $+\infty$ в
отрицательной части спектра также соответствует требованиям к хорошей
функции релаксации.

\subsubsection{Реализация метода}
Согласно \eqref{eq:relax-fun}, выбранной функции релаксации
соответствует зависимость
\begin{equation*}
  H(\lambda) = \frac{1-\frac{P_s(\lambda)}{s}}{\lambda}
\end{equation*}
откуда с учётом \eqref{eq:chebyshev} получим следующие рекуррентные
соотношения уже для функции $H(\lambda)$:
\begin{equation}
  \label{eq:cheb-scalarfun}
  \begin{aligned}
    H_1 &= 0 \\
    H_2 &= 2 \\
    H_{s+1} \mul (s+1) &= 2s\mul(1-2\lambda)\mul H_s-(s-1)\mul
    H_{s-1}+4s
  \end{aligned}
\end{equation}
Параметр метода $s$ равен степени используемых полиномов Чебышёва.
Соображения по выбору $s$ приведены ниже.

Подстановка полученного соотношения \eqref{eq:cheb-scalarfun}
в \eqref{eq:grad-methods} даёт следующую рабочую форумулу
\begin{multline}
  x^{k+1} = x^k - H_{s+1}g_k =\\=
  x^k-\frac{2s}{s+1}(E-2G_k)H_sg_k+\frac{s-1}{s+1}H_{s-1}g_k-\frac{4s}{s+1}g_k
\end{multline}

Таким образом, вектор смещения $d_{s+1} = x^{k+1} - x^k$ при выбранном
значении параметра $s$ на каждом шаге $k$ вычисляется по следующей
рекуррентной формуле:
\begin{equation}
  \label{eq:cheb-workhorse}
  \begin{aligned}
    d_1 &= 0\\
    d_2 &= -2g_k \\
    d_{s+1} &=
    \frac{2s}{s+1}(E-2G_k)d_{s}-\frac{s-1}{s+1}d_{s-1}-\frac{4s}{s+1}g_k
  \end{aligned}
\end{equation}

После вычисления $d_{s+1}$ применяется регулировка шага путём деления
его пополам:
\begin{equation}
  \label{eq:cheb-regulation}
  d_{s+1} = \frac{d_{s+1}}{2}
\end{equation}
причём регулировка \eqref{eq:cheb-regulation} последовательно
продолжается до тех пор, пока не будет обеспечено условие релаксации
\eqref{eq:relax-cond} процесса оптимизации. Подобное ограничение шага
$d_{s+1}$ используется в целях предотвращения выхода из области
справедливости локальной квадратичной модели \eqref{eq:sqr-approx}. На
идее такой регулировки также построен метод \rgd{}, описанный ранее в
разделе \ref{sec:gd}.

\subsubsection{Выбор параметра метода}
\label{sec:cheb-param}

Для применения на практике рассматриваемый метод минимизации требует
задания параметра $s$. Обратимся к способам выбора $s$.

Автором метода предлагается выбирать значение $s$ из следующего
соотношения:
\begin{equation}
  \label{eq:cheb-param}
  s = 1.3 \sqrt{\eta}
\end{equation}
где $\eta$ — оценка овражности минимизируемой функции. Таким образом,
возникает необходимость определения $\eta$ перед применением метода.
Рассмотрим предлагаемые для этого способы.

\paragraph{Использование гессиана}

Можно определять степень овражности непосредственно из
определения \ref{dfn:gully}, исследуя аналитические выражения для
собственных чисел гессиана функции или же вычисляя их численно. При
алгоритмическом построении спектра матрцы Гессе и делении на
$\lambda_{\min}$ могут возникнуть проблемы численного переполнения,
вызванные нехваткой машинной точности.

\paragraph{Использование метода \gd{}}

Данный подход предусматривает применение метода простого градиентного
спуска (см. раздел \ref{sec:gd}) для оценки $\eta$. 

Для этого сначала запускается решение исходной задачи минимизации
\eqref{eq:optim-problem-form} с помощью \gd{}. Когда метод простого
градиентного спуска зацикливается (см. \ref{sec:problems-ill}),
отношение норм градиентов целевой функции в точках соседних
приближений стабилизируется около некоторого $\mu$:
\begin{equation}
  \label{eq:gdrelch-stabilization}
  \frac{\norm{f'(x^{k+1})}}{\norm{f'(x^k)}} \approx \mu
\end{equation}
Тогда оценка овражности может быть найдена из следующего
соотношения:
\begin{equation}
  \label{eq:gdrelch-param}
  \eta = \frac{2}{\abs{1-\mu}}
\end{equation}
После этого из \eqref{eq:cheb-param} определяется $s$, и метод
\relch{} начинает работу из последней найденной методом \gd{} точки
приближения.

Для точности оценки $s$ необходимо в качестве шага $h$ в рабочей
формуле \eqref{eq:gd-workhorse} метода \gd{} выбирать как можно
большее значение, при котором обеспечивается стабилизация отношения
\eqref{eq:gdrelch-stabilization}. При этом для определения $h$,
удовлетворяющего такому свойству, могут потребоваться предварительные
попытки решения задачи с различными пробными значениями $h$.

Отметим, что в случае функции малой овражности достаточно хорошее
решение может получиться уже на этапе применения \gd{}.

При алгоритмическом построении $s$ нужно учитывать возможность
получения столь большого значения параметра, что на вычисления будет
затрачиваться непозволительно долгое время. Поэтому при начальном
рассмотрении задачи может потребоваться искусственное ограничение
параметра $s$ сверху.

Двухэтапную модификацию метода \relch{} в дальнейшем будем обозначать
как \gdrelch{}.

\begin{rem}
  \label{rem:cheb-rel-speed}
  При увеличении параметра $s$ на некоторое ограниченное значение рост
  скорости релаксации, вообще говоря, не гарантируется.
\end{rem}

В разделе \ref{sec:test-problems} представлены результаты применения
\gdrelch{} на тестовых функциях.

\subsubsection{Трудности, возникающие при реализации \relch{}}

В ходе тестирования \relch{} были выявлены случаи, когда метод
попадает в такую точку, что норма градиента и значение гессиана
функции приводят к численному переполнению в результате многократных
вычислений по формуле \eqref{eq:cheb-workhorse}, так что условие
релаксации нарушается, несмотря на регулировку
\eqref{eq:cheb-regulation}.

Для борьбы с таким поведением использовалась следующая
\emph{эвристическая} техника: после вычисления
\eqref{eq:cheb-workhorse} при обнаружении бесконечного значения
$\norm{d_{s+1}}$ следующее приближение выбиралось в случайном
направлении на расстоянии $\sqrt{n}$ от текущего. В общем случае
невозможно доказать результатовность таких действий, однако подход
продемонстрировал успех на ряде тестовых задач.

При выборе начального приближения вдали от точки минимума многократные
вычисления по формуле \eqref{eq:cheb-workhorse} также могут привести к
численному переполнению и из-за большого по модулю значения целевой
функции. В определённых случаях проблемы можно избежать, используя на
первом этапе метод \gd{} или \rgd{} для начального снижения значения
целевого функционала.

\clearpage
\subsection{Тестовые задачи}
\label{sec:test-problems}

В данном разделе рассмотрено несколько тестовых функций с их анализом
на выпуклость и овражность, а также приведены результаты применения
метода \relch{} для решения задач их минимизации. 

Для анализа выпуклости функций использовалась теорема
\ref{th:convex-f-hess}, применимая в силу того, что все
рассматриваемые функции дважды непрерывно-дифференцируемы.

Овражность оценивалась с помощью определения \ref{dfn:gully},
эвристически и на основе информации о применимости метода \gd{} к
целевой функции (см. разделы \ref{sec:gd}, \ref{sec:problems-ill}).

Все численные значения приведены с точностью до двух значащих цифр.

\pgfplotsset{every axis/.append style={xlabel=$x$, ylabel=$y$}}

\subsubsection{Функция Розенброка}

Одним из классических тестов для различных алгоритмов оптимизации
является тест Розенброка, заключающийся в минимизации следующей
функции:
\begin{equation}
  \label{eq:rosenbrock}
  f(x, y) = 100(y - x²)² + (1 - x)²
\end{equation}
Точку $(-1.2, 1)$ предлагается взять в качестве начального
приближения.

\paragraph{Выпуклость}

Матрица Гессе этой функции имеет вид:
\begin{equation}
  \label{eq:rosenbrock-hess}
  \begin{pmatrix}
    1200x^2-400y+2 & -400x \\
    -400x & \phm200\phantom{x}
  \end{pmatrix}
\end{equation}

Воспользуемся теоремой \ref{th:convex-f-hess} для проверки функции на
выпуклость. Применим критерий Сильвестра к матрице
\eqref{eq:rosenbrock-hess}. Запишем условия положительности её угловых
миноров:
\begin{align*}
  &\Delta_1 = 1200\,x^2-400\,y+2 > 0 \iff y < 3x^2+\frac{1}{200}\\
  &\Delta_2 = 80\,000\,x^2-80\,000\,y+400 > 0 \iff y < x^2+\frac{1}{200}
\end{align*}
При этом второе ограничение является более сильным. Таким образом,
граница области выпуклости функции Розенброка проходит вдоль кривой
\begin{equation}
  \label{eq:rosenbrock-convex-bound}
  y = x^2 +\frac{1}{200}
\end{equation}
Ниже этой кривой функция Розенброка строго выпукла. Согласно теореме
\ref{th:convex-f-smin}, она имеет в своей области выпуклости
единственную точку минимума.

При значениях $(x, y)$, лежащих выше кривой
\eqref{eq:rosenbrock-convex-bound}, функция Розенброка вогнута, а
потому, в силу теоремы \ref{th:convex-f-nomax}, не может иметь внутри
этой области точек минимума.

\paragraph{Овражность}

Собственные числа матрицы Гессе функции Розенброка в точке $(x, y)$
имеют следующие значения, отличающиеся лишь знаком перед корнем:
\begin{gather*}
  \begin{split}
    \lambda_1(x, y) = -\sqrt{400\,000\,y^2 +
      \left(39\,600-240\,000\,x^2\right)y + 360\,000\,x^4 +
      41\,200\,x^2+9801}- \\
    - 200y+600x^2+101
  \end{split}\\
  \begin{split}
    \lambda_2(x, y) = \sqrt{400\,000\,y^2 +
      \left(39\,600-240\,000\,x^2\right)y + 360\,000\,x^4 +
      41\,200\,x^2+9801}- \\
    - 200y+600x^2+101
  \end{split}
\end{gather*}

В точке начального приближения, лежащей в области выпуклости,
собственные значения матрицы Гессе функции Розенброка приблизительно
равны $23.63$ и $1506.37$, откуда согласно \eqref{eq:gully} получается
степень овражности
\begin{equation}
  \label{eq:rosenbrock-gully-start}
  \eta = \frac{1506.37}{23.63} \approx 63.7
\end{equation}

Исследуемая функция имеет кривую перегиба $y=x^2+\frac{1}{200}$,
разделяющую её области выпуклости и вогнутости. Вдоль этой кривой
степень овражности функции $\eta = \infty$.

\paragraph{Аналитическое решение задачи}

Найдём минимум функции \eqref{eq:rosenbrock} аналитически.

Из необходимых условий стационарности точки имеем:
\begin{align*}
  \pardiff{f}{x} &= 400x^3-400yx+2x-2 = 0 \\
  \pardiff{f}{y} &= 200y-200x^2 = 0
\end{align*}
откуда $x = y = 1$. Найденная точка $A(1, 1)$ не принадлежит кривой
перегиба $y = x^2+\frac{1}{200}$, а потому необходимые условия
экстремума для неё оказываются достаточными. Таким образом, с учётом
теорем \ref{th:convex-f-smin} и \ref{th:convex-f-nomax} точка $A(1,1)$
является глобальным минимумом функции Розенброка.

Вычисление значения $\eta$ в $A$ даёт
\begin{equation}
  \label{eq:rosenbrock-gully-extr}
  \eta = \frac{1001.6006}{0.3994} \approx 2507.7631
\end{equation}

\paragraph{Минимизация с помощью \relch{} при ручном выборе параметра}

На иллюстрации \ref{fig:rosenbrock-contours} представлены линии уровня
функции Розенброка, которые сильно вытягиваются вдоль всей кривой
$y=x^2$ за счёт близости перегиба и быстрого роста функции в выпуклой
области. С учётом наблюдаемой картины, а также перепада в значениях
степени овражности \eqref{eq:rosenbrock-gully-start} и
\eqref{eq:rosenbrock-gully-extr} можно утверждать, что в ходе работы
алгоритму придётся бороться с сильной овражностью целевой функции.

\begin{figure}[!thb]
  \centering
  \begin{tikzpicture}
    \begin{axis}
      [x=3cm, y=2.7cm,
      xmin=-1.5, xmax=1.5,
      ymin=-1,ymax=2]
      \input{rosenbrock_hq-contours.tkz.tex}
      \addplot[mark=none,thick,black,densely dashed] plot[domain=-1.41:1.41] function{x**2};
      \node[circle,fill=black,scale=0.5,label={right:\contour{white}{$A$}}] at
      (axis cs:1,1) {};
    \end{axis}
  \end{tikzpicture}
  \caption[Функция Розенброка]{Линии уровней $1, 2, 2.5, 4, 5, 50,
    100$ функции Розенброка \eqref{eq:rosenbrock}, её кривая перегиба
    $y=x^2+\frac{1}{200}$ и глобальный минимум в точке $A(1, 1)$}
  \label{fig:rosenbrock-contours}
\end{figure}

В качестве пробных оценок $\eta$ для применения в формуле
\eqref{eq:cheb-param} выберем значения
\eqref{eq:rosenbrock-gully-start} и \eqref{eq:rosenbrock-gully-extr}, тогда:
\begin{gather*}
  s_1 = 1.3 \sqrt{64} \approx 10 \\
  s_2 = 1.3 \sqrt{2507} \approx 66
\end{gather*}
Из эмпирических соображений в качестве $s$ выбиралось ближайшее к
$1.3\sqrt{\eta}$ \emph{чётное} число (в ходе тестов алгоритм
демонстрировал наилучшую сходимость именно при чётных $s$).

На иллюстрации \ref{fig:rosenbrock-relch} представлены результаты
работы \relch{} на функции Розенброка. Видно, что при малом значении
$s=10$ \relch{} быстро попадает в окрестность минимума, но испытывает
трудности при продвижении по дну оврага. С ростом $s$ до 66
(рис. \ref{fig:rosenbrock-relch66}) алгоритм крупными шагами
стремительно движется вдоль оврага к точке минимума. Протестирована
также работа алгоритма с $s=200$. Высокая скорость сходимости с таким
значением соответствует ожиданиям.

\begin{figure}[p]
  \center
  \pgfplotsset{every axis/.append style={x=2cm,y=1.5cm}}
  \subfigure[$s=10$]{
    \begin{tikzpicture}
      \begin{axis}[]
        \input{rosenbrock-contours.tkz.tex}
        \input{rosenbrock_relch_-1.2,1_40_10-trace.tkz.tex}
      \end{axis}      
    \end{tikzpicture}}
  \hskip 0.5cm
  \subfigure[$s=66$]{
    \begin{tikzpicture}
      \begin{axis}[]
        \input{rosenbrock-contours.tkz.tex}
        \input{rosenbrock_relch_-1.2,1_40_66-trace.tkz.tex}
      \end{axis}      
    \end{tikzpicture}}
  \\\vskip 0.5cm
  \subfigure[$s=200$]{
    \begin{tikzpicture}
      \begin{axis}[]
        \input{rosenbrock-contours.tkz.tex}
        \input{rosenbrock_relch_-1.2,1_20_200-trace.tkz.tex}
      \end{axis}      
    \end{tikzpicture}}
  \caption[\relch{} на функции Розенброка]{Минимизация функции
    Розенброка алгоритмом \relch{} при различных значениях параметра
    $s$}
  \label{fig:rosenbrock-relch}
\end{figure}

\paragraph{Тестирование \gdrelch{}}

\begin{figure}[!thb]
  \centering
  \begin{tikzpicture}
    \begin{axis}[x=2.2cm, y=1.8cm]
      \input{rosenbrock-contours.tkz.tex}
      \input{rosenbrock_gdrelch_-1.2,1__0.0019-trace.tkz.tex}
        \node[circle,fill=black,scale=0.25,pin={below right:\contour{white}{$A$}}] at
        (axis cs:1,1) {};
    \end{axis}
  \end{tikzpicture}
  \caption[\gdrelch{} на функции Розенброка, $s=0.0019$]{Минимизация
    функции Розенброка алгоритмом \gdrelch{} при $h=0.0019$}
\end{figure}



\clearpage
\subsubsection{Функция Химмельблау}
\label{sec:himmelblau}

Рассмотрим уже виденную
ранее\footnote{См. раздел \ref{sec:problems-ill}} функцию Химмельблау,
которая задаётся следующим образом:
\begin{equation}
  \label{eq:himmelblau}
  \tag{$\chi$-\theequation}
  f(x, y) = (x² + y - 11)² + (x + y² - 7)²
\end{equation}

\paragraph{Выпуклость}

Матрица Гессе функции Химмельблау имеет вид:
\begin{equation}
  \label{eq:himmelblau-hess}
  \begin{pmatrix}
    4y+12x^2-42 & 4y+4x\\
    4y+4x & 12y^2+4x-26
  \end{pmatrix}
\end{equation}
Исследуем её на определённость с помощью критерия Сильвестра, записав
условие положительности угловых миноров:
\begin{align}
  \label{eq:himmelblau-conv}
  &\Delta_1 = 4y+12x^2-42 > 0 \iff y > 10\frac{1}{2}-3x^2 \\
  &\Delta_2 =
  4\left(12y^3+(36x^2-130)y^2-(4x+26)y+12x^3-82x^2-42x+273\right) > 0
\end{align}
Первое условие имеет простой вид, в то время как границей решения
второго является кривая четвёртого порядка. Геометрический образ
условий \eqref{eq:himmelblau-conv} вместе с линиями уровня функции
Химмельблау представлен на рисунке \ref{fig:himmelblau-conv}. 

\begin{figure}[p]
  \centering
  \begin{tikzpicture}
    \begin{axis}
      [x=.7cm,y=.7cm]
      \addplot[mark=none,black,thick, densely dashed]
      plot[domain=-2.5:2.5] function{10.5-3*x**2};
      \label{plot:himmelblau-hyperbole}
      \input{himmelblau_max-contours.tkz.tex}
      \input{himmelblau_conv-contours.tkz.tex}

      \node[] at (axis cs:0,0)
      {\contour{white}{$\underset{\cancel{\text{\scriptsize{Выпукла}}}}{\text{\small{Вогнута}}}$}};

      % Concavity labels
      \node[] at (axis cs:4,0)
      {\contour{white}{\small{Вогнута}}};
      \node[] at (axis cs:-4,0)
      {\contour{white}{\small{Вогнута}}};
      \node[] at (axis cs:0,4)
      {\contour{white}{\small{Вогнута}}}; 
      \node[] at (axis cs:0,-4)
      {\small{Вогнута}};
      
      % Convexity labels
      \node[] at (axis cs:5,-5)
      {\contour{white}{\small{Выпукла}}};
      \node[] at (axis cs:-5,-5)
      {\contour{white}{\small{Выпукла}}};
      \node[] at (axis cs:-5,5)
      {\contour{white}{\small{Выпукла}}};
      \node[] at (axis cs:5,5)
      {\contour{white}{\small{Выпукла}}};
      
    \end{axis}      
  \end{tikzpicture}
  \caption[Функция Химмельблау]{Линии уровня функции Химмельблау
    \eqref{eq:himmelblau} и границы областей выпуклости}
  \label{fig:himmelblau-conv}
\end{figure}


Во всей области под гиперболой $10\frac{1}{2}-3x^2$
(\ref{plot:himmelblau-hyperbole}) функция вогнута, несмотря на то, что
на графическом изображении второго условия вокруг начала координат
виден скруглённый прямоугольник, внутри которого выполняется условие
$\Delta_2>0$. В этом отношении функция Химмельблау отличается от
функции Розенброка, где условие $\Delta_1>0$ было безусловно слабее
условия $\Delta_2>0$.

\paragraph{Аналитическое решение}
Известно, что локальные минимумы\footnote{Численные значения координат
  $A, B, C, D$ приведены в литературе, см. \cite{himmelblau75}. Их
  аналитические выражения имеют весьма сложный вид.} функции
\eqref{eq:himmelblau} расположены в точках $A=(-3.78, -3.28)$,
$B=(-2.80, 3.13)$, $C=(3.58, -1.85)$, \mbox{$D=(3.00, 2.00)$}, и в
каждой из них достигается значение $0$. 
Все они находятся в областях выпуклости функции, чего и следует
ожидать.

\paragraph{Овражность}
Как было продемонстрировано ещё в разделe \ref{sec:problems-ill}, даже
метод \gd{} с фиксированным шагом может локализовать точку минимума
функции Химмельблау. Исходя из этого, можно предположить, что высокой
степенью овражности эта функция не обладает.

Действительно, даже в точках минимумов $\eta$ не превосходит
\emph{четырёх}:
\begin{equation}
  \begin{aligned}
    \eta(A) & \approx 1.9&\qquad
    \eta(B) & \approx 3.18\\
    \eta(C) & \approx 3.65&\qquad
    \eta(D) & \approx 3.2
\end{aligned}
\end{equation}

С учётом данных соображений можно заключить, что предпосылок для
выбора очень большого значения $s$ при использовании \relch{} для
оптимизации функции нет. Необходимость применения метода \gdrelch{} в
данном случае также ничем не обоснована.

\paragraph{Использование \relch{}}

На рисунке \ref{fig:himmelblau} представлены результаты работы
алгоритма \relch{} для решения задачи минимизации функции
\eqref{eq:himmelblau}. Во всех случаях использовалось значение $s=50$.
На иллюстрации также видно, что из-за многоэкстремальности целевой
функции результат работы алгоритма зависит от выбора начального
приближения.

В отсутствие эталонных данных о минимумах проверить найденные решения
на единственность можно было бы, воспользовавшись информацией об
областях выпуклости функции и теоремами из раздела
\ref{sec:convexity}.

\begin{figure}[thb]
  \centering
  \begin{tikzpicture}
    \begin{axis}[x=.8cm,y=.8cm]
      \input{himmelblau-contours.tkz.tex}
      \input{himmelblau_relch_0.1,0.1_50-trace.tkz.tex}
      \input{himmelblau_relch_0.5,-0.5_50-trace.tkz.tex}
      \input{himmelblau_relch_-0.5,0.5_50-trace.tkz.tex}
      \input{himmelblau_relch_-0.5,-0.5_50-trace.tkz.tex}

      \input{himmelblau_relch_0,8_50-trace.tkz.tex}
      \input{himmelblau_relch_4,7.2_50-trace.tkz.tex}
      
      \node[circle,fill=black,scale=0.25,
      label={above left:\contour{white}{$A$}}] at (axis cs:-3.78,-3.28) {};
        
      \node[circle,fill=black,scale=0.25,
      label={below left:\contour{white}{$B$}}] at (axis cs:-2.8,3.13) {};
      
      \node[circle,fill=black,scale=0.25,
      label={below right:\contour{white}{$C$}}] at (axis cs:3.58,-1.85) {};

      \node[circle,fill=black,scale=0.25,
      label={above right:\contour{white}{$D$}}] at (axis cs:3,2) {};
    \end{axis}      
  \end{tikzpicture}
  \caption[\relch{} на функции Химмельблау]{Результаты работы
    алгоритма с функцией Химмельблау \eqref{eq:himmelblau} в
    зависимости от выбора начальной точки}
  \label{fig:himmelblau}
\end{figure}


\clearpage
\subsubsection{Экспоненциальная функция}
\label{sec:exptest}

Рассмотрим функцию
\begin{equation}
  \label{eq:exptest}
  \tag{$e$-\theequation}
  f(x, y) = \suml_{a\in A}\left [
    e^{-xa}-e^{-ya}-(e^{-a}-e^{-10a})\right ]^2
\end{equation}
Здесь суммирование происходит по значениям $a \in A =\{0.1, 0.2, \dotsc, 1\}$.

\paragraph{Выпуклость}

Сложность функции \eqref{eq:exptest} не позволяет провести анализ её
выпуклости так легко, как это было сделано для функции Розенброка.
Матрица Гессе в точке $(x, y)$ имеет вид:
\begin{equation}
  \label{eq:exptest-hess}
  \suml_{a\in A}\left[2a^2×
  \begin{pmatrix}
    e^{-ax}(2e^{-ax}-e^{-ay}-c_a) & -e^{-a(x+y)} \\
    -2a^2e^{-a(x+y)} & e^{-ay}(e^{-ax}-c_a)
  \end{pmatrix}\right]
\end{equation}
где $c_a = e^{-a}-e^{-10a}$. В силу положительности $a$, $c_a>0$.

Попробуем использовать для исследования её на определённость критерий
Сильвестра. Угловые миноры должны быть положительны:
\begin{align*}
  &\Delta_1 = \suml_{a\in A}
  \left[ 2a^2e^{-ax}(2e^{-ax}-e^{-ay}-c_a) \right] > 0\\
  \begin{split}
    \Delta_2 = \suml_{a\in A}
    \left[ e^{-ax}(2e^{-ax}-e^{-ay}-c_a) \right]×
    \suml_{a\in A}
    \left[ 2a^2e^{-ay}(2e^{-ay}-e^{-ax}+c_a) \right]-\\
    -\left(\suml_{a\in A}
      \left[ 2a^2e^{-ax-ay} \right] \right)^2
    > 0
  \end{split}
\end{align*}
Решить это неравенства аналитически нелегко. Обратимся к численным
методам. На рисунке \ref{fig:exptest-convexity} изображены найденные
численно границы областей выпуклости, в которых миноры $\Delta_1$ и
$\Delta_2$ обращаются в ноль. Оказывается, что функция выпукла лишь
внутри узкой области, заключённой между границами $\Delta_2=0$.

\paragraph{Аналитическое решение}

Найдём минимальное значение \eqref{eq:exptest} аналитически. Из
условий стационарности имеем
\begin{equation*}
  \pardiff{f}{x} = -\pardiff{f}{y} =\suml_{a\in A}\left[2ae^{-ay}(e^{-ax}-e^{-ay}-(e^{-a}-e^{-10a}))\right] = 0
\end{equation*}
откуда определяется единственное решение — точка $A (1,10)$. Точка $A$
попадает в область выпуклости функции
(см. рис. \ref{fig:exptest-convexity}), а потому является глобальным
минимумом.

\paragraph{Овражность}

Несмотря на видимую вытянутость линий уровня и быстрое стремление к
бесконечности при $x<0,y<0$, рассматриваемая функция не обладает
высокой степенью овражности даже в окрестности точки минимума:

\begin{equation}
  \begin{aligned}
    \eta(0,1) &\approx 140.57&\qquad
    \eta(1,5) &\approx 40.5\\
    \eta(1,9) &\approx 257.93&\qquad
    \eta(1.5,11.5) &\approx 142.5
\end{aligned}
\end{equation}

Как оказывается, и в точке $A$ степень овражности невысока:
\begin{equation*}
  \eta(A) = \frac{1.75}{0.004} \approx 437.5
\end{equation*}

Поэтому использование высоких $s$ в \relch{} не обязательно.

\paragraph{Минимизация с помощью \relch{}}

На рисунке \ref{fig:exptest-relch} представлены результаты работы
\relch{} с рассматриваемой функцией при $s=20$. Алгоритм стабильно
определяет минимум в точке $(1, 10)$.

\paragraph{Применение \gdrelch{}}

Как видно из таблицы \ref{tab:exptest-gdrelch}, при начальной точке в
$(2,6)$ и шаге $h=0.0065$ метод \gdrelch{} стабилизируется уже в
окрестности \emph{начальной точки} с $\norm{f'(x)} = 0.86$, после чего
быстро минимизирует функцию до порядка $10^{-10}$. 

Аналогичная ситуация имеет место при другом выборе начального
приближения, см. рис. \ref{fig:exptest-gdrelch}.

\begin{figure}[p]
  \centering
  \begin{tikzpicture}
    \begin{axis}
      [x=1cm, y=1cm]
      \input{exptest_conv-contours.tkz.tex}
      \label{plot:exptest-first-minor}

      \input{exptest-contours.tkz.tex}
      \input{exptest_conv2-contours.tkz.tex}
      
      \node[] at (axis cs:1.2,6.5)
      {\contour{white}{\small{Выпукла}}};
      \node[] at (axis cs:-0.5,6)
      {\contour{white}{\small{Вогнута}}};
      \node[] at (axis cs:3.5,5)
      {\contour{white}{\large{Вогнута}}};

      \node[circle,fill=black,scale=0.25,label={above:\contour{white}{$A(1,10)$}}]
      at (axis cs:1,10) {};
    \end{axis}
  \end{tikzpicture}
  \caption[Экспоненциальная функция]{Линии уровней $0.01,\dotsc,2$
    функции \eqref{eq:exptest}, границы области выпуклости
    $\Delta_1=0$, $\Delta_2=0$ и глобальный минимум в точке $A(1,
    10)$.}
  \label{fig:exptest-convexity}
\end{figure}

\begin{figure}[p]
  \centering
  \subfigure[$x^0=(5,6)$]{
    \begin{tikzpicture}
      \begin{axis}[x=.65cm,y=1cm,ymin=-1]
        \input{exptest-contours.tkz.tex}
        \input{exptest_relch_5,6_20_20-trace.tkz.tex}
        \node[circle,fill=black,scale=0.25,label={above:\contour{white}{$A\,(1,10)$}}]
        at (axis cs:1,10) {};
      \end{axis}
    \end{tikzpicture}}\hskip 0.5cm
    \subfigure[$x^0=(0,1)$]{
    \begin{tikzpicture}
      \begin{axis}[x=.65cm,y=1cm,ymin=-1,yticklabel pos=right, ylabel={}]
        \input{exptest-contours.tkz.tex}
        \input{exptest_relch_0,1_20_20-trace.tkz.tex}
      \end{axis}
    \end{tikzpicture}}
  \caption[\relch{} на экспоненциальной функции]{Минимизация функции
    \eqref{eq:exptest} алгоритмом \relch{} при $s=20$.}
  \label{fig:exptest-relch}
\end{figure}

\begin{figure}[p]
  \centering
  \subfigure[$x^0=(2,6),\,h=0.0065$
  (табл. \ref{tab:exptest-gdrelch})]{
    \begin{tikzpicture}
      \begin{axis}[x=.65cm,y=1cm]
        \input{exptest-contours.tkz.tex}
        \input{exptest_gdrelch_2,6_30_0.0065-trace.tkz.tex}
        \node[circle,scale=3,draw=black,densely dashed] at (axis cs:2,6) {};
        \node[circle,fill=black,scale=0.25,label={below left:\contour{white}{$A(1,10)$}}]
        at (axis cs:1,10) {};
      \end{axis}
    \end{tikzpicture}}\hskip 1cm
  \subfigure[$x^0=(1,1),\,h=0.001$]{
    \begin{tikzpicture}
      \begin{axis}[x=.65cm,y=1cm,yticklabel pos=right,ylabel={}]
        \input{exptest-contours.tkz.tex}
        \input{exptest_gdrelch_1,1_50_0.001-trace.tkz.tex}
        \node[circle,scale=3,draw=black,densely dashed] at (axis cs:1,1) {};
        \node[circle,fill=black,scale=0.25] at (axis cs:1,10) {};
      \end{axis}
    \end{tikzpicture}}
  \caption[\gdrelch{} на экспоненциальной функции]{Минимизация функции
    \eqref{eq:exptest} алгоритмом \gdrelch{}.}
  \label{fig:exptest-gdrelch}
\end{figure}

\begin{table}[hpb]
  \centering
  \input{exptest_gdrelch_2,6_30_0.0065-trace.tbl.tex}
  \caption{Минимизация
    функции \eqref{eq:exptest} алгоритмом \gdrelch{} при шаге $h=0.0065$.}
  \label{tab:exptest-gdrelch}
\end{table}


\clearpage
\subsubsection{Многопараметрическая оптимизация}

В качестве примера приведём трассировку процесса оптимизации функции
четырёх переменных, заданной следующим образом:
\begin{equation}
  \label{eq:multivar}
  f(x) = (x₁ + 10x₂)² + 5(x₃ - x₄)² + (x₂-2x₃)⁴+10(x₁-x₄)⁴
\end{equation}

Особенность функции \eqref{eq:multivar} состоит в том, что она имеет
вырожденную матрицу Гессе в точке минимума.

\begin{table}
  \centering
  \input{hess-singular_relch_-3,-1,-3,-1_30_500-trace.tbl.tex}
  \caption[\relch{} на функции четырёх переменных]{Минимизация функции \eqref{eq:multivar} методом \relch{} при $s=100$}
  \label{tab:multivar}
\end{table}

Получаемый результат — точка $(0,0,0,0)$ — совпадает с эталонным,
приведённым в \cite{himmelblau75}.


\subsection{Выводы}

Описанный и протестированный алгоритм \relch{} обладает следующими
особенностями:
\begin{itemize}
\item Теоретически доказана бесконечная скорость релаксации \relch{}.
\item Метод \relch{} обладает устойчивостью к овражности целевого
  функционала.
\item Возможно алгоритмическое определение значения параметра $s$ с
  помощью предварительного использования метода \gd{}.
\item Метод относительно просто реализуется, в алгоритме нет
  концептуально сложных вычислений.
\item Автором метода в \cite{chernorutsky04} также указывается на
  возможность применения алгоритма для оптимизации
  многопараметрических систем, в которых гессиан целевого функционала
  представляет собой разреженную матрицу большой размерности.
\end{itemize}

В разделе \ref{sec:sources} представлены исходные тексты реализации
\relch{} и \gdrelch{} на языке Scheme.
