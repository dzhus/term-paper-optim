\subsubsection{Экспоненциальная функция}
\label{sec:exptest}

Рассмотрим функцию
\begin{equation}
  \label{eq:exptest}
  \tag{$\Xi$-\theequation}
  f(x, y) = \suml_{a\in A}\left [
    e^{-xa}-e^{-ya}-(e^{-a}-e^{-10a})\right ]^2
\end{equation}
Здесь суммирование происходит по значениям $a \in A =\{0.1, 0.2, \dotsc, 1\}$.

\paragraph{Выпуклость}

Сложность функции \eqref{eq:exptest} не позволяет провести анализ её
выпуклости так легко, как это было сделано для функции Розенброка.
Матрица Гессе в точке $(x, y)$ имеет вид:
\begin{equation}
  \label{eq:exptest-hess}
  \suml_{a\in A}\left[2a^2×
  \begin{pmatrix}
    e^{-ax}(2e^{-ax}-e^{-ay}-c_a) & -e^{-a(x+y)} \\
    -2a^2e^{-a(x+y)} & e^{-ay}(e^{-ax}-c_a)
  \end{pmatrix}\right]
\end{equation}
где $c_a = e^{-a}-e^{-10a}$. В силу положительности $a$, $c_a>0$.

Попробуем использовать для исследования её на определённость критерий
Сильвестра. Угловые миноры должны быть положительны:
\begin{align*}
  \Delta_1 &= \suml_{a\in A}
  \left[ 2a^2e^{-ax}(2e^{-ax}-e^{-ay}-c_a) \right] > 0\\
  \Delta_2 &= \suml_{a\in A}
  \left[ e^{-ax}(2e^{-ax}-e^{-ay}-c_a) \right]×
  \suml_{a\in A}
  \left[ 2a^2e^{-ay}(2e^{-ay}-e^{-ax}+c_a) \right]-\\
  &-\left(\suml_{a\in A}
  \left[ 2a^2e^{-ax-ay} \right] \right)^2
  > 0
\end{align*}
Решить это неравенства аналитически нелегко. Обратимся к численным
методам. На рисунке \ref{fig:exptest-convexity} изображены найденные
численно границы областей выпуклости, в которых миноры $\Delta_1$ и
$\Delta_2$ обращаются в ноль. В области ниже кривой $\Delta_1=0$
(\ref{plot:exptest-first-minor}) функция вогнутая, при этом в область
вогнутости попадает одно из множеств, удовлетворяющих условию
$\Delta_2 > 0$, как было в случае с функцией Химмельблау.

\paragraph{Аналитическое решение}

Найдём минимальное значение \eqref{eq:exptest} аналитически. Из
условий стационарности имеем
\begin{equation*}
  \begin{aligned}
    &\suml_{a\in A}\left[2ae^{-ay}(e^{-ax}-e^{-ay}-(e^{-a}-e^{-10a}))\right] = 0\\
    -&\suml_{a\in A}\left[2ae^{-ay}(e^{-ax}-e^{-ay}-(e^{-a}-e^{-10a}))\right] = 0
  \end{aligned}
\end{equation*}
откуда определяется единственное решение — точка $A (1,10)$. Точка $A$
попадает в область выпуклости функции
(см. рис. \ref{fig:exptest-convexity}), а потому является глобальным
минимумом.

\paragraph{Овражность}

Несмотря на видимую вытянутость линий уровня и быстрое стремление к
бесконечности при $x<0,y<0$, рассматриваемая функция не обладает
высокой степенью овражности даже в окрестности точки минимума:

\begin{equation}
  \begin{aligned}
    \eta(0,1) &= \frac{9.84}{0.07} \approx 140.57&\qquad
    \eta(1,5) &= \frac{1.62}{0.04} \approx 40.5\\
    \eta(1,9) &= \frac{1.74}{0.0067} \approx 257.93&\qquad
    \eta(1.5,11.5) &= \frac{0.57}{0.004} \approx 142.5
\end{aligned}
\end{equation}

Как оказывается, и в точке $A$ степень овражности невысока:
\begin{equation*}
  \eta(A) = \frac{1.75}{0.004} \approx 437.5
\end{equation*}

Поэтому использование высоких $s$ в \relch{} не обязательно.

\paragraph{Минимизация с помощью \relch{}}

На рисунке \ref{fig:exptest-relch} представлены результаты работы
\relch{} с рассматриваемой функцией при $s=20$. Алгоритм стабильно
определяет минимум в точке $(1, 10)$.

\paragraph{Применение \gdrelch{}}

Как видно из таблицы \ref{tab:exptest-gdrelch}, при начальной точке в
$(2,6)$ и шаге $h=0.006$ метод \gdrelch{} стабилизируется уже в
окрестности \emph{начальной точки} с $\norm{f'(x)} = 0.86$, после чего
быстро минимизирует функцию до порядка $10^{-10}$.

\begin{figure}[thb]
  \centering
  \begin{tikzpicture}
    \begin{axis}
      [x=1cm, y=1cm]
      \input{exptest_conv-contours.tkz.tex}
      \label{plot:exptest-first-minor}

      \input{exptest-contours.tkz.tex}
      \input{exptest_conv2-contours.tkz.tex}
      
      \node[] at (axis cs:1.2,6.5)
      {\contour{white}{\small{Выпукла}}};
      \node[] at (axis cs:-0.5,6)
      {\contour{white}{\small{Вогнута}}};
      \node[] at (axis cs:3.5,5)
      {\contour{white}{\large{Вогнута}}};

      \node[circle,fill=black,scale=0.25,label={above:\contour{white}{$A(1,10)$}}]
      at (axis cs:1,10) {};
    \end{axis}
  \end{tikzpicture}
  \caption[Экспоненциальная функция]{Линии уровней $0.01,\dotsc,2$
    функции \eqref{eq:exptest}, границы области выпуклости
    $\Delta_1=0$, $\Delta_2=0$ и глобальный минимум в точке $A(1,
    10)$.}
  \label{fig:exptest-convexity}
\end{figure}

\begin{figure}[thb]
  \centering
  \begin{tikzpicture}
    \begin{axis}[x=1cm,y=1cm]
      \input{exptest-contours.tkz.tex}
      \input{exptest_relch_-0.5,1.5_20_20-trace.tkz.tex}
      \input{exptest_relch_5,5_20_20-trace.tkz.tex}
      \input{exptest_relch_3,-1_25_20-trace.tkz.tex}
      \node[circle,fill=black,scale=0.25,label={left:\contour{white}{$A(1,10)$}}]
      at (axis cs:1,10) {};
      \node[circle,fill=black,scale=0.25,label={left:\contour{white}{$A(1,10)$}}]
      at (axis cs:1,10) {};
      \node[circle,fill=black,scale=0.25,label={left:\contour{white}{$A(1,10)$}}]
      at (axis cs:1,10) {};
    \end{axis}
  \end{tikzpicture}
  \caption[\relch{} на экспоненциальной функции]{Минимизации функции
    \eqref{eq:exptest} алгоритмом \relch{} при $s=20$ и различных
    значениях начального приближения: $(-0.5,1.5)$, $(5,5)$ и
    $(3,-1)$.}
  \label{fig:exptest-relch}
\end{figure}

\begin{figure}[!thb]
  \centering
  \begin{tikzpicture}
    \begin{axis}[x=1cm,y=1cm,ymin=4]
      \input{exptest-contours.tkz.tex}
      \input{exptest_gdrelch_2,6_30_0.006-trace.tkz.tex}
      \node[circle,scale=3,draw=black,densely dashed] at (axis
      cs:2,6) {};
      \node[circle,fill=black,scale=0.25,label={left:\contour{white}{$A(1,10)$}}]
      at (axis cs:1,10) {};
    \end{axis}
  \end{tikzpicture}
  \caption[\gdrelch{} на экспоненциальной функции]{Минимизация функции
    \eqref{eq:exptest} алгоритмом \gdrelch{} при шаге $h=0.006$.}
\end{figure}

\begin{table}[hpb]
  \centering
  \input{exptest_gdrelch_2,6_30_0.006-trace.tbl.tex}
  \caption{Минимизация
    функции \eqref{eq:exptest} алгоритмом \gdrelch{} при шаге $h=0.006$.}
  \label{tab:exptest-gdrelch}
\end{table}
