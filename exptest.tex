\subsubsection{Экспоненциальная функция}
\label{sec:exptest}

Теперь рассмотрим функцию
\begin{equation}
  \label{eq:exptest}
  f(x, y) = \sum\limits_{a=\overline{0.1, 1.0}}\left [
    e^{-xa}-e^{-ya}-(e^{-a}-e^{-10a})\right ]^2
\end{equation}
Здесь суммирование происходит по значениям $a = 0.1, 0.2, \dotsc, 1$.

\begin{figure}[!thb]
  \centering
  \begin{tikzpicture}
    \begin{axis}[x=1cm,y=1cm]
      \input{exptest-contours.tkz.tex}
      \input{exptest_relch_-0.5,1.5_20-trace.tkz.tex}
      \input{exptest_relch_5,5_20-trace.tkz.tex}
      \input{exptest_relch_3,-1_25_12-trace.tkz.tex}
      \node[circle,fill=black,scale=0.5,label=right:$A$] at (axis cs:1,10) {};
    \end{axis}
  \end{tikzpicture}
  \caption[Экспоненциальная функция]{Линии уровня и трассировка процесса минимизации функции
    \eqref{eq:exptest}}
\end{figure}

Алгоритм стабильно определяет минимум в точке $(1, 10)$.
