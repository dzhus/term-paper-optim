
\subsection{Метод возможных направлений Зойтендейка}
\label{sec:zoutendijk}

\subsubsection{Общая схема методов возможных направлений}

Итерационные методы возможных направлений, к которым относится и метод
Зойтендейка, реализуют решение задачи
\eqref{eq:cond-optim-problem-form} по общей схеме, согласно которой
поиск начинается в допустимой точке пространства решений и
продолжается по траектории, обеспечивающей улучшение значения целевой
функции, но не выходящей за границы допустимой области.

На каждом шаге очередное приближение к точке минимума находится по
формуле
\begin{equation}
  \label{eq:pd-iter}
  x^{k+1} = x^k + s_k \mul d^k
\end{equation}

Различие между методами возможных направлений заключается в стратегии
выбора направления $d^k$ и шага $s_k$.

\subsubsection{Алгоритм метода Зойтендейка}

В методе Зойтендейка на каждой итерации направление и шаг выбираются
так, чтобы обеспечить наименьшую минимизирующую поправку к целевой
функции без нарушения какого-либо ограничения.

Пусть в \eqref{eq:cond-optim-problem-form} функции ограничений
$g_j(x)$ линейны или линеаризованы (с помощью разложения в ряд
Тейлора) так, что получена следующая задача (рассматривается случай
минимизации):
\begin{equation}
  \label{eq:zoutendijk-problem-form}
  \begin{cases}
    f(x) \to \min \\
    \scalmult{a^j,x} \leq b_j,\, j=\overline{1,m} \\
    x_i \geq 0,\, i=\overline{1,n} \\
    x \in \set{R}^n
  \end{cases}
\end{equation}

Теперь сформулируем алгоритм метода Зойтендейка, описанный в
\cite{himmelblau75}.

Начальное приближение $x^1$ выбирается из числа точек, удовлетворяющих
ограничениям задачи \eqref{eq:zoutendijk-problem-form}.

\begin{enumerate}
  \renewcommand{\labelenumi}{\textbf{Шаг \arabic{enumi}.}}
\item Определяются множества номеров активных ограничений $I^k$ — для
  ограничений вида $x_i \geq 0$ и $J^k$ — для ограничений вида
  $\scalmult{a^j,x} \leq b_j$.
\item Если $I^k=J^k = \varnothing$, то $d^k=-f'(x^k)$. В противном
  случае решают задачу линейного программирования по минимизации
  скалярного произведения
  \begin{equation*}
    \scalmult{f'(x^k), d^k} \to \min
  \end{equation*}
  при наличии ограничений
  \begin{equation*}
    \scalmult{a^j, d^k} \leq 0
  \end{equation*}
  что соответствует поиску такого допустимого направления, вдоль
  которого целевая функция убывает быстрее всего.

  Для этого в векторе $d^k$ каждую $i$-ю координату
  представляют\footnote{Верхние индексы у $d_i, d_{i+}, d_{i-}$
    опущены для компактности} как $d_i$ для $i \in I^k$ и
  $d_{i+}-d_{i-}$ для $i \notin I^k$, так что решаемая задача
  приобретёт следующий вид:
  \begin{equation}
    \label{eq:zoutendijk-linprog}
    \begin{cases}
      \scalmult{f'(x^k), d^k} = 
      \suml_{i\in I^k}{\pardiff{f(x^k)}{x_i}d_i} + 
      \suml_{i \notin I^k}{\pardiff{f(x^k)}{x_i}(d_{i+}-d_{i-})}
      \to \min\\
      
      \scalmult{a^j, d^k} = 
      \suml_{i\in I^k}{a^j_id_i} +
      \suml_{i \notin I^k}{a^j_i(d_{i+}-d_{i-})} \leq 0,
      \, i\in J^k \\
      
      \suml_{i=1}^n\abs{d_i^k} = \suml_{i\in I^k}{d_i}+
      \suml_{i \notin I^k}{\left( d_{i+} + d_{i-}\right)} \leq 1\\
      
      d_{i+} \geq 0,\, d_{i-} \geq 0,\,d_{i+}\mul d_{i-}=0,\, i\notin I^k
    \end{cases}
  \end{equation}
\item Если выполняется условие
  \begin{equation}
    \label{eq:zoutendijk-halt}
    \scalmult{f'(x^k), d^k} = 0
  \end{equation}
  то дальнейшее улучшение решения невозможно и процесс прерывается.

  В противном случае определяют величину шага
  \begin{equation}
    \label{eq:zoutendijk-step}
    s_k =
    \min\{s_*,\hat{s}_1,\dotsc,\hat{s}_n,\bar{s}_1,\dotsc,\bar{s}_m\}
  \end{equation}
  где компоненты под $\min$ находятся по следующим правилам:
  \begin{itemize}
  \item $s_*$ — решение задачи безусловной одномерной оптимизации
    \begin{equation} 
      \label{eq:zoutendijk-1optim}
      f(x^k+s_* \mul d^k) \to \min
    \end{equation}

  \item $\hat{s}_i$ — наибольший шаг, при котором ограничение
    \begin{equation*}
      x^k+\hat{s}_i d^k_i \geq 0
    \end{equation*}
    остаётся удовлетворённым. Если компонента найденного вектора
    направления $d^k_i \geq 0$, то $\hat{s}_i = +\infty$ (так что
    соответствующая компонента $\hat{s}_j$ согласно
    \eqref{eq:zoutendijk-step} никак не ограничивает выбор $s_k$),
    иначе
    \begin{equation*}
      \hat{s}_i = -\frac{x^k_i}{d^k_i}
    \end{equation*}

  \item $\bar{s}_j$ — наибольший шаг, при котором при движении из
    $x^k$ по направлению $d_k$ ограничение
    \begin{equation*}
      \scalmult{a^j, x^k+\bar{s}_j d^k} \leq b_j
    \end{equation*}
    остаётся выполненным. Если $\scalmult{a^j, d^k} \leq 0$, то
    ограничение будет удовлетворено при любом шаге, поэтому $\bar{s}_j
    = +\infty$. В противном случае, $\bar{s}_j$ определяется как
    \begin{equation*}
      \bar{s}_j = \frac{b_j-\scalmult{a^j, x^k}}{\scalmult{a^j, d^k}}
    \end{equation*}
  \end{itemize}
  Отметим, что $s_k>0$.
\item С учётом шага $s_k$ и направления $d^k$ согласно
  \eqref{eq:pd-iter} полагают новое приближение равным
  \begin{equation*}
    x^{k+1} = x^k + s_k \mul d^k
  \end{equation*}
  и переходят к следующей итерации.
\end{enumerate}

Таким образом, критерием останова процесса оптимизации $k$-й итерации
является выполнение равенства \eqref{eq:zoutendijk-halt}, что
равносильно невозможности дальнейшего уменьшения значения целевой
функции при заданных ограничениях.

\subsubsection{Применение метода Зойтендейка}

Рассмотрим работу метода возможных направлений на примере задачи
\eqref{eq:cond-optim-problem-raw}. Будем искать точку минимума.
Ограничения уже линейны, поэтому дополнительная линеаризация не
требуется. Задача уже приведена к виду
\eqref{eq:zoutendijk-problem-form}:
\begin{equation}
  \label{eq:zoutendijk-problem}
  \begin{cases}
    f(x) = (x_1+4)^2 + (x_2-4)^2 \to \min \\
    2x_1-x_2 \leq 2 \\
    x_1 \geq 0 \\
    x_2 \geq 0
  \end{cases}
\end{equation}

\begin{enumerate}
  \renewcommand{\labelenumi}{\textbf{Итерация \arabic{enumi}.}}
  \renewcommand{\labelenumii}{\textbf{Шаг \arabic{enumii}.}}
\item
  В качестве первого приближения $x^1$ возьмём точку $(1, 1)$, очевидно
  удовлетворяющую всем ограничениям.
  \begin{enumerate}
  \item Ни одно из ограничений не активно в данной точке, поэтому
    множества $I^k = J^k = \varnothing$.
  \item Направление движения определим как противоположное вектору
    градиента:
    \begin{equation*}
      d^1 = -f(x^1) = \begin{pmatrix} -10 \\ \phm 6 \end{pmatrix}
    \end{equation*}
  \item Определим коэффициенты $s_*, \hat{s}_1, \hat{s}_2, \bar{s}_1$.
    \begin{itemize}
    \item Минимум функции $f(x^1+s_*\mul d^1) = (5-10s_*)^2+(6s_*-3)^2$
      достигается при $s_*=\frac{1}{2}$.
    \item $d^1_1 = -10 \ngeq 0 \implies \hat{s}_1 =
      -\frac{x^1_1}{d^1_1} = -\frac{1}{-10} = \frac{1}{10}$
    \item $d^1_2 = 6 \geq 0 \implies \hat{s}_2 = +\infty$
    \item $\scalmult{a^1, d^1} = -26 \leq
      0 \implies \bar{s}_1 = +\infty$
    \end{itemize}
    Находим величину шага: $s_1 = \min\{\frac{1}{2}, \frac{1}{10}\} =
    \frac{1}{10}$.
  \item Вычисляем новое приближение $x^2$:
    \begin{equation*}
      x^2 = \begin{pmatrix} 1 \\ 1 \end{pmatrix} +
      \frac{1}{10} \begin{pmatrix} -10 \\ \phm 6 \end{pmatrix}
      = \begin{pmatrix} 0 \\ \frac{8}{5} \end{pmatrix}
    \end{equation*}
  \end{enumerate}
\item Продолжим поиск минимума из точки
  \begin{equation*}
    x^2 = \begin{pmatrix} 0 \\ \frac{8}{5} \end{pmatrix}
  \end{equation*}
  \begin{enumerate}
  \item В $x^2$ активно лишь ограничение $x_1\geq 0$, поэтому $I^2 =
    \varnothing, J^2 = \{1\}$.
  \item Составим задачу линейного программирования:
    \begin{equation*}
      \begin{cases}
        \scalmult{f'(x^2), d^2} = \tilde{f}_2 =
        8d_1 - \frac{3}{5} (d_{2+}-d_{2-}) \to \min \\
        d_1+(d_{2+}+d_{2-}) \leq 1 \\
        d_1, d_{2+}, d_{2-} \geq 0 \\
        d_{2+} \mul d_{2-} = 0
      \end{cases}
    \end{equation*}
    Приведём её к каноническому виду, получив в первом ограничении
    равенство путём добавления переменной $x_1$:
    \begin{equation*}
      \begin{cases}
        \tilde{f}_2 = 8d_1-\frac{3}{5}(d_{2+}-d_{2-}) \to \min \\
        d_1+(d_{2+}+d_{2-}) + x_1 = 1 \\
        d_1, d_{2+}, d_{2-}, x_1 \geq 0 \\
        d_{2+} \mul d_{2-} = 0
      \end{cases}
    \end{equation*}
    Решим задачу симплекс-методом. Составим матрицу коэффициентов
    ограничений и целевой функции, затем исключим $x_1$ из состава
    базисных переменных:
    \begin{equation}
      \begin{gmatrix}[b]
        \dagger & d_1 & d_{2+} & d_{2-} & x_1 & \diamond\\
        x_1 & 1 & 1 & 1 & 1 & \mathbf{1} \\
        \tilde{f}_2 & \mm{8} & \mathbf{\frac{3}{5}} & \mm{\frac{3}{5}} & 0 & 0
        \rowops \add[-\frac{3}{5}]{1}{2}
      \end{gmatrix}
    \end{equation}
    Получим матрицу с одними отрицательными коэффициентами в нижнем
    ряду:
    \begin{equation*}
      \begin{gmatrix}[b]
        \dagger & d_1 & d_{2+} & d_{2-} & x_1 & \diamond\\
        x_1 & 1 & 1 & 1 & 1 & 1 \\
        \tilde{f}_2 & \mm{\frac{43}{5}} & 0 & \mm{\frac{6}{5}} & \mm{\frac{3}{5}}
        & \mm{\frac{3}{5}}
      \end{gmatrix}
    \end{equation*}
    При этом условие $d_{2+} \mul d_{2-} = 0$ выполнено. Итак, найдено
    оптимальное решение задачи линейного программирования:
    \begin{equation*}
      \begin{cases}
        d_{1\phantom{+}} = 0 \\
        d_{2+} = 1 \\
        d_{2-} = 0
      \end{cases}
    \end{equation*}
    с учётом которого построим вектор $d^2$:
    \begin{equation*}
      d^2 = \begin{pmatrix} 0 \\ 1 \end{pmatrix}
    \end{equation*}
  \item Определим коэффициенты $s_*, \hat{s}_1, \hat{s}_2, \bar{s}_1$.
    \begin{itemize}
    \item Минимум $f(x^2+s_*\mul d^2) = 16+(s_*-\frac{32}{5})^2$
      достигается при $s_* = \frac{32}{5}$.
    \item $d^2_1 = 0 \geq 0 \implies \hat{s}_1 = +\infty$
    \item $d^2_1 = 1 \geq 0 \implies \hat{s}_1 = +\infty$
    \item $\scalmult{a^1, d^2} = -1 \leq 0 \implies \bar{s}_1 =
      +\infty$ Итак, ограничения никак не влияют на величину шага,
      поэтому $s_k = \frac{32}{5}$.
    \end{itemize}
  \item Найдём приближение $x^3$:
    \begin{equation*}
      x^3 = \begin{pmatrix} 0 \\ \frac{8}{5} \end{pmatrix} +
      \frac{32}{5} \begin{pmatrix} 0 \\ 1 \end{pmatrix}
      = \begin{pmatrix} 0 \\ 4 \end{pmatrix}
    \end{equation*}
  \end{enumerate}
\item Как известно из точного аналитического решения
  (см. раздел \ref{sec:kuhn-tucker}), точка $(0, 4)$ является точным
  решением задачи \eqref{eq:zoutendijk-problem}. Значит,
  удовлетворение \eqref{eq:zoutendijk-halt} на этой итерации должно
  сигнализировать об этом. Выполним первые два шага алгоритма, что
  убедиться в этом.
  \begin{enumerate}
  \item Как и на прошлой итерации, $I^2 = \varnothing, J^2 = \{1\}$.
  \item Для нахождения компонент вектора $d^k$ решим задачу линейного
    программирования:
    \begin{equation*}
      \begin{cases}
        \scalmult{f'(x^3), d^3} = \tilde{f}_3 =
        8d_1 \to \min \\
        d_1+(d_{2+}+d_{2-}) + x_1 = 1 \\
        d_1, d_{2+}, d_{2-}, x_1 \geq 0 \\
        d_{2+} \mul d_{2-} = 0
      \end{cases}
    \end{equation*}
    оптимальным решением которой является тривиальный набор
    \begin{equation*}
      \begin{cases}
        d_{1\phantom{+}} = 0 \\
        d_{2+} = 0 \\
        d_{2-} = 0
      \end{cases}
    \end{equation*}

    Таким образом,
    \begin{equation*}
      d^3 = \begin{pmatrix} 0 \\ 0 \end{pmatrix}
    \end{equation*}

    Очевидно, что выполняется критерий останова
    \eqref{eq:zoutendijk-halt}. Значит, далее уменьшать значение
    $f(x)$ в \eqref{eq:zoutendijk-problem} не представляется
    возможным.
  \end{enumerate}
\end{enumerate}

Итак, с помощью метода возможных направлений Зойтендейка найдено
решением задачи условной минимизации \eqref{eq:zoutendijk-problem},
которое согласуется с решением, найденным в \ref{sec:kuhn-tucker}.

\begin{figure}[!h]
  \centering
  \begin{tikzpicture}
    \begin{axis}[
      x=2.4 cm, y=1.3 cm,
      xmin=0, ymin=0, xmax=2.5,
      xlabel=$x_1$, ylabel=$x_2$,
      enlargelimits=0.05]
      \input{condextr-contours.tkz.tex}
      \addplot[ultra thick, dashed, red!50!black] coordinates{(0,5) (0,0) (1,0) (2.5,3)};
      
      \addplot[mark=*, only marks,
               mark options={fill=blue!70!black}] coordinates{(1,1)
                 (0,1) (0, 4)};
      
      \addplot[blue, thick, mark=none, arrows=-triangle 45] coordinates{(1,1) (0,1)};
      \addplot[blue, thick, mark=none, arrows=-triangle 45] coordinates{(0,1) (0,4)};      
    \end{axis}
  \end{tikzpicture}
  \caption[Метод Зойтендейка]{Ход решения задачи
    \eqref{eq:zoutendijk-problem} методом Зойтендейка}
  \label{fig:cond-optim}
\end{figure}

\subsubsection{Использование метода}

Метод Зойтендейка подходит для реализации на компьютере, поскольку не
содержит принципиально сложных шагов. Единственным проблемным этапом
может быть одномерная минимизация \eqref{eq:zoutendijk-1optim} для
нахождения значения $s_*$.

Линеаризация ограничений может вносить определённую погрешность в
решение.

Основным же ограничением методом Зойтендейка является невозможность
решения задач с ограничениями типа равенств.
