\pgfplotsset{every axis/.append style={xmin=-5,xmax=5,ymin=-5,ymax=5,x=0.35cm,y=0.45cm}}
\begin{figure}[p]
  \center
  \subfigure[$p=0$]{
    \begin{tikzpicture}
      \begin{axis}[]
        \input{condextr_full-contours.tkz.tex}
        \addplot[thick,densely dashed] coordinates{(0,5)
          (0,0) (1,0) (3.5,5)};
        \label{plot:pen-boundaries}
      \end{axis}      
    \end{tikzpicture}}
  \hskip 0.5cm
  \subfigure[$p=5$]{
    \begin{tikzpicture}
      \begin{axis}[]
        \input{penalty_mini-contours.tkz.tex}
        \addplot[thick,densely dashed] coordinates{(0,5)
          (0,0) (1,0) (3.5,5)};
      \end{axis}      
    \end{tikzpicture}}\\\vskip 0.5cm
  \subfigure[$p=50$]{
    \begin{tikzpicture}
      \begin{axis}[]
        \input{penalty-contours.tkz.tex}
        \addplot[thick,densely dashed] coordinates{(0,5)
          (0,0) (1,0) (3.5,5)};
      \end{axis}      
    \end{tikzpicture}}\hskip 0.5cm
  \subfigure[$p=150$]{
    \begin{tikzpicture}
      \begin{axis}[]
        \input{penalty_maxi-contours.tkz.tex}
        \addplot[thick,densely dashed] coordinates{(0,5)
          (0,0) (1,0) (3.5,5)};
      \end{axis}      
    \end{tikzpicture}}
  \caption[Влияние коэффициента штрафа на вид целевой функции]{Линии
    уровня функции $P(x)$ при различных значениях коэффициента штрафа $p$}
  \label{fig:pen-contours}
\end{figure}
