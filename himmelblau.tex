\subsubsection{Функция Химмельблау}
\label{sec:himmelblau}

В этому рассмотрим уже виденную
ранее\footnote{См. раздел \ref{sec:problems-ill}} функцию Химмельблау,
которая задаётся следующим образом:
\begin{equation}
  \label{eq:himmelblau}
  \tag{$\chi$-\theequation}
  f(x, y) = (x² + y - 11)² + (x + y² - 7)²
\end{equation}

\paragraph{Выпуклость}

Матрица Гессе функции Химмельблау имеет вид:
\begin{equation}
  \label{eq:himmelblau-hess}
  \begin{pmatrix}
    4y+12x^2-42 & 4y+4x\\
    4y+4x & 12y^2+4x-26
  \end{pmatrix}
\end{equation}
Исследуем её на определённость с помощью критерия Сильвестра, записав
условие положительности угловых миноров:
\begin{align}
  \label{eq:himmelblau-conv}
  &\Delta_1 = 4y+12x^2-42 > 0 \iff y > 10\frac{1}{2}-3x^2 \\
  &\Delta_2 =
  4\left(12y^3+(36x^2-130)y^2-(4x+26)y+12x^3-82x^2-42x+273\right) > 0
\end{align}
Первое условие имеет простой вид, в то время как границей решения
второго является кривая четвёртого порядка. Геометрический образ
условий \eqref{eq:himmelblau-conv} вместе с линиями уровня функции
Химмельблау представлен на рисунке \ref{fig:himmelblau-conv}. 

Во всей области под гиперболой $10\frac{1}{2}-3x^2$
(\ref{plot:himmelblau-hyperbole}) функция вогнута, несмотря на то, что
на графическом изображении второго условия вокруг начала координат
виден скруглённый прямоугольник, внутри которого выполняется условие
$\Delta_2>0$. В этом функция Химмельблау отличается от функции
Розенброка, где условие $\Delta_1>0$ было безусловно слабее условия
$\Delta_2>0$.

\begin{figure}[thb]
  \centering
  \begin{tikzpicture}
    \begin{axis}
      [x=.7cm,y=.7cm]
      \addplot[mark=none,black,thick, densely dashed]
      plot[domain=-2.5:2.5] function{10.5-3*x**2};
      \label{plot:himmelblau-hyperbole}
      \input{himmelblau_max-contours.tkz.tex}
      \input{himmelblau_conv-contours.tkz.tex}

      \node[] at (axis cs:0,0)
      {\contour{white}{$\underset{\cancel{\text{\scriptsize{Выпукла}}}}{\text{\footnotesize{Вогнута}}}$}};

      % Concavity labels
      \node[] at (axis cs:4,0)
      {\contour{white}{\small{Вогнута}}};
      \node[] at (axis cs:-4,0)
      {\contour{white}{\small{Вогнута}}};
      \node[] at (axis cs:0,4)
      {\contour{white}{\small{Вогнута}}}; 
      \node[] at (axis cs:0,-4)
      {\small{Вогнута}};
      
      % Convexity labels
      \node[] at (axis cs:5,-5)
      {\contour{white}{\small{Выпукла}}};
      \node[] at (axis cs:-5,-5)
      {\contour{white}{\small{Выпукла}}};
      \node[] at (axis cs:-5,5)
      {\contour{white}{\small{Выпукла}}};
      \node[] at (axis cs:5,5)
      {\contour{white}{\small{Выпукла}}};
      
    \end{axis}      
  \end{tikzpicture}
  \caption[Функция Химмельблау]{Линии уровня функции Химмельблау
    \eqref{eq:himmelblau} и границы областей выпуклости}
  \label{fig:himmelblau-conv}
\end{figure}

\paragraph{Аналитическое решение}
Известно, что локальные минимумы функции \eqref{eq:himmelblau}
расположены в точках $A=(-3.78, -3.28)$, $B=(-2.80, 3.13)$, $C=(3.58,
-1.85)$, \mbox{$D=(3.00, 2.00)$}, и в каждой из них достигается
значение $0$. Все они находятся в областях выпуклости функции, чего и
следует ожидать.

\paragraph{Овражность}
Как было продемонстрировано ещё в разделe \ref{sec:problems-ill}, даже
метод \gd{} с фиксированным шагом может локализовать точку минимума
функции Химмельблау. Исходя из этого, можно предположить, что высокой
степенью овражности эта функция не обладает.

Действительно, даже в точках минимумов $\eta$ не превосходит
\emph{четырёх}:
\begin{equation}
  \begin{aligned}
    \eta(A) &= \frac{133.761}{70.561} \approx 1.9&\qquad
    \eta(B) &= \frac{91.25}{28.67} \approx 3.18\\ 
    \eta(C) &= \frac{105.03}{28.76} \approx 3.65&\qquad
    \eta(D) &= \frac{82.28}{25.72} \approx 3.2
\end{aligned}
\end{equation}

С учётом данных соображений можно заключить, что предпосылок для
выбора очень большого значения $s$ при использовании \relch{} для
оптимизации функции нет. Необходимость применения метода \gdrelch{} в
данном случае также ничем не обоснована.

\paragraph{Использование \relch{}}

На рисунке \ref{fig:himmelblau} представлены результаты работы
алгоритма \relch{} для решения задачи минимизации функции
\eqref{eq:himmelblau}. Во всех случаях использовалось значение $s=50$.
На иллюстрации также видно, что из-за многоэкстремальности целевой
функции результат работы алгоритма зависит от выбора начального
приближения.

В отсутствие эталонных данных\footnote{Численные значения $A, B, C, D$
  приведены в литературе, см. \cite{himmelblau75}. Аналитические
  выражения для координат точек минимума $A,B,C,D$ имеют весьма
  сложный вид.} о минимумах проверить найденные решения на
единственность можно было бы, воспользовавшись информацией об областях
выпуклости функции и теоремами из раздела \ref{sec:convexity}.

\begin{figure}[thb]
  \centering
  \begin{tikzpicture}
    \begin{axis}[x=.8cm,y=.8cm]
      \input{himmelblau-contours.tkz.tex}
      \input{himmelblau_relch_0.1,0.1_50-trace.tkz.tex}
      \input{himmelblau_relch_0.5,-0.5_50-trace.tkz.tex}
      \input{himmelblau_relch_-0.5,0.5_50-trace.tkz.tex}
      \input{himmelblau_relch_-0.5,-0.5_50-trace.tkz.tex}

      \input{himmelblau_relch_0,8_50-trace.tkz.tex}
      \input{himmelblau_relch_4,7.2_50-trace.tkz.tex}
      
      \node[circle,fill=black,scale=0.25,label=above left:$A$] at (axis cs:-3.78,-3.28) {};
      \node[circle,fill=black,scale=0.25,label=below left:$B$] at (axis cs:-2.8,3.13) {};
      \node[circle,fill=black,scale=0.25,label=below right:$C$] at (axis cs:3.58,-1.85) {};
      \node[circle,fill=black,scale=0.25,label=above right:$D$] at (axis cs:3,2) {};
    \end{axis}      
  \end{tikzpicture}
  \caption[\relch{} на функции Химмельблау]{Результаты работы
    алгоритма с функцией Химмельблау \eqref{eq:himmelblau} в
    зависимости от выбора начальной точки}
  \label{fig:himmelblau}
\end{figure}
