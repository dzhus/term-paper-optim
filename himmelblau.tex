\subsubsection{Функция Химмельблау}
\label{sec:himmelblau}

В следующем примере рассмотрим уже виденную
ранее\footnote{См. раздел \ref{sec:problems-ill}} функцию Химмельблау,
которая задаётся следующим образом:
\begin{equation}
  \label{eq:himmelblau}
  f(x, y) = (x² + y - 11)² + (x + y² - 7)²
\end{equation}
Известно, что локальные минимумы функции \eqref{eq:himmelblau}
расположены в точках $A=(-3.78, -3.28)$, $B=(-2.80, 3.13)$, $C=(3.58,
-1.85)$, $D=(3.00, 2.00)$, и в каждой из них достигается значение $0$.
Этим обусловлена следующая особенность работы алгоритма — результат
значительно зависит от выбора начального приближения, что
продемонстрировано на рисунке \ref{fig:himmelblau}.

\begin{figure}[!thb]
  \centering
  \begin{tikzpicture}
    \begin{axis}[x=.8cm,y=.8cm]
      \input{himmelblau-contours.tkz.tex}
      \input{himmelblau_relch_0.1,0.1_10-trace.tkz.tex}
      \input{himmelblau_relch_0.5,-0.5_10-trace.tkz.tex}
      \input{himmelblau_relch_-0.5,0.5_10-trace.tkz.tex}
      \input{himmelblau_relch_-0.5,-0.5_10-trace.tkz.tex}

      \node[circle,fill=black,scale=0.5,label=above left:$A$] at (axis cs:-3.78,-3.28) {};
      \node[circle,fill=black,scale=0.5,label=below left:$B$] at (axis cs:-2.8,3.13) {};
      \node[circle,fill=black,scale=0.5,label=below right:$C$] at (axis cs:3.58,-1.85) {};
      \node[circle,fill=black,scale=0.5,label=above right:$D$] at (axis cs:3,2) {};
    \end{axis}      
  \end{tikzpicture}
  \caption[Функция Химмельблау]{Результаты работы алгоритма с функцией
    Химмельблау \eqref{eq:himmelblau} в зависимости от выбора
    начальной точки}
  \label{fig:himmelblau}
\end{figure}
