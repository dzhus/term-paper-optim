\begin{figure}[p]
  \centering
  \begin{tikzpicture}
    \begin{axis}
      [x=.7cm,y=.7cm]
      \addplot[mark=none,black,thick, densely dashed]
      plot[domain=-2.5:2.5] function{10.5-3*x**2};
      \label{plot:himmelblau-hyperbole}
      \input{himmelblau_max-contours.tkz.tex}
      \input{himmelblau_conv-contours.tkz.tex}

      \node[] at (axis cs:0,0)
      {\contour{white}{$\underset{\cancel{\text{\scriptsize{Выпукла}}}}{\text{\small{Вогнута}}}$}};

      % Concavity labels
      \node[] at (axis cs:4,0)
      {\contour{white}{\small{Вогнута}}};
      \node[] at (axis cs:-4,0)
      {\contour{white}{\small{Вогнута}}};
      \node[] at (axis cs:0,4)
      {\contour{white}{\small{Вогнута}}}; 
      \node[] at (axis cs:0,-4)
      {\small{Вогнута}};
      
      % Convexity labels
      \node[] at (axis cs:5,-5)
      {\contour{white}{\small{Выпукла}}};
      \node[] at (axis cs:-5,-5)
      {\contour{white}{\small{Выпукла}}};
      \node[] at (axis cs:-5,5)
      {\contour{white}{\small{Выпукла}}};
      \node[] at (axis cs:5,5)
      {\contour{white}{\small{Выпукла}}};
      
    \end{axis}      
  \end{tikzpicture}
  \caption[Функция Химмельблау]{Линии уровня функции Химмельблау
    \eqref{eq:himmelblau} и границы областей выпуклости}
  \label{fig:himmelblau-conv}
\end{figure}
