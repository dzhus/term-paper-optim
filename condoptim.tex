\section{Условная оптимизация}

\subsection{Использование теоремы Куна—Таккера}
\label{sec:kuhn-tucker}

\subsubsection{Теоретические сведения}

Сформулируем ряд определений и теорем, которые используются при
решении задач условной оптимизации нелинейных функционалов. Более
подробно тема освещена в \cite{alekseev05}.

Рассмотрим следующую задачу на поиск экстремума целевой функции $f(x)$
в области $D$, заданной ограничениями типа неравенств:
\begin{equation}
  \label{eq:cond-optim-problem-form}
  \begin{cases}
    f(x) \to \extr \\
    g_j(x) \leq 0,\, j=\overline{1,m} \\
    x \in \set{R}^n
  \end{cases}
\end{equation}

\begin{dfn} \neword{Функцией Лагранжа} задачи
  \eqref{eq:cond-optim-problem-form} называют функцию
  \begin{equation}
    \label{eq:lagrange-form}
    \La(x, \lambda_0, \lambda) = \lambda_0 f(x) + \sum_{j=1}^m {\lambda_j g_j(x)}
  \end{equation}
  где $\lambda$ — вектор $\lambda_1, \dotsc, \lambda_m$. При
  $\lambda_0=1$ функция Лагранжа называется \neword{классической}.
\end{dfn}

\begin{dfn}
  \label{dfn:regular}
  Точка $\hat{x}$, являющаяся решением задачи
  \eqref{eq:cond-optim-problem-form} с функцией Лагранжа
  \eqref{eq:lagrange-form} при $\lambda_0 \neq 0$, называется
  \neword{регулярным экстремумом}, а при $\lambda_0 = 0$ —
  \neword{нерегулярным}.
\end{dfn}

\begin{thm}[Куна—Таккера]
  \label{th:kuhn-tucker}
  Пусть точка $\hat{x}$ является решением задачи
  \eqref{eq:cond-optim-problem-form} с функцией Лагранжа
  \eqref{eq:lagrange-form} при соответствующих $\lambda_0$ и
  $\lambda$. Тогда выполнены следующие условия:
  \begin{enumerate}
    \renewcommand{\labelenumi}{\emph{\asbuk{enumi})}}
  \item $\pardiff{\La}{x_i}=0,\, i=\overline{1,n}$ (условие
    стационарности функции Лагранжа)
  \item $\lambda_j \geq 0,\, j=\overline{1,m}$, если $\hat{x}$ — точка
    минимума, и $\lambda_j \leq 0$, если это точка максимума.
  \item $\lambda_j \mul g_j(x) = 0,\, j=\overline{1,m}$ (условие
    дополняющей нежёсткости)
  \item $g_j(x) \leq 0,\, j=\overline{1,m}$
  \item $\lambda_0^2 + \norm{\lambda}^2 > 0$ (условие нетривиальности решения)
  \end{enumerate}
\end{thm}

\begin{dfn}
  Точка $x^*$, удовлетворяющая условиям теоремы \ref{th:kuhn-tucker},
  называется \neword{условно-стационарной}.
\end{dfn}

Аналогично определению \ref{dfn:regular} введём схожее и для
условно-стационарных точек.
\begin{dfn}
  Условно-стационарную точку $x^*$, для которой условия
  \ref{th:kuhn-tucker} выполнены при $\lambda ≠ 0$, будем называть
  \neword{регулярной} условно-стационарной точкой задачи
  \eqref{eq:cond-optim-problem-form}.
\end{dfn}

При использовании условий \ref{th:kuhn-tucker} для поиска
условно-стационарных точек рассматривают два варианта значений
$\lambda_0$ в функции Лагранажа: $\lambda_0=0$ и $\lambda_0 \neq 0$. В
последнем случае без ограничения общности обычно полагают $\lambda_0 =
1$.

\begin{thm}[Условие регулярности]
  Для того, чтобы в условно-стационарной точке $x^*$ выполнялось
  неравенство $\lambda_0 \neq 0$, достаточно линейной независимости
  градиентов активных в этой точке ограничений $g_k'(x^*),
  g_{k+1}'(x^*), \dotsc, g_{k+l}'(x^*)$.
\end{thm}

Если удаётся показать, что во \emph{всех} допустимых точках задачи
\eqref{eq:cond-optim-problem-form} выполнено условие регулярности,
случай $\lambda_0=0$ можно исключить из рассмотрения. Для этого также
удобно использовать следующее условие.

\begin{thm}[Условие Слейтера]
  \label{th:slater}
  Для $\lambda_0 \neq 0$ в условиях \ref{th:kuhn-tucker} достаточно
  существования такой точки $x_s$, в которой все неравенства
  ограничений выполняются строго: $g_j(x_s)<0, \, j=\overline{1,m}$.
\end{thm}

\begin{dfn}
  \label{dfn:convex-f}
  Функцию $f(x), x \in \set{R}^n$ называется \neword{выпуклой}, если
  для $\forall x, y \in \set{R}^n, \forall \alpha \in [0;1]$
  выполняется неравенство
  \begin{equation*}
    f(\alpha x + (1-\alpha)y) \leq \alpha f(x) + (1-\alpha) f(y)
  \end{equation*}
  Когда это неравенство $\forall \alpha \in (0;1)$ выполняется строго,
  функцию $f(x)$ называют \neword{строго выпуклой}.
\end{dfn}
\begin{rem}
  \label{rem:lin-f-convex}
  Линейные функции выпуклы, но не строго выпуклы.
\end{rem}

\begin{dfn}
  Функция $f(x)$ называется \neword{вогнутой}, если $-f(x)$ является
  выпуклой функцией.
\end{dfn}

\begin{dfn}
  Множество $D \subset \set{R}^n$ называется \neword{выпуклым}, если
  для $\forall x, y \in D, \forall \alpha \in [0;1]$ выполняется
  \begin{equation*}
    \alpha x + (1-\alpha) y \in D
  \end{equation*}
\end{dfn}
Таким образом, отрезок, соединяющий две точки выпуклого множества,
целиком принадлежит этому множеству.

Следущие две теоремы оказываются полезными при доказательстве
выпуклости функций и множеств.

\begin{thm}
  \label{th:convex-f}
  Если матрица Гессе квадратичной функции положительно определена, то
  эта функция выпукла.
\end{thm}

\begin{thm}
  \label{th:convex-set}
  Если $g_j(x), j=\overline{1,m}$ — выпуклые функции, то множество
  $D$, заданное условиями $g_j(x) \leq b_j$, является выпуклым.
\end{thm}

Приведём замечание, которое говорит об условиях \emph{достаточности}
условий теоремы \ref{th:kuhn-tucker}.

\begin{rem}
  \label{rem:kt-cond}
  Для точек минимума необходимые условия Куна—Таккера становятся
  достаточными в случае, когда целевая функция $f(x)$ и ограниченное
  неравенствами $g_j(x)$ множество \emph{выпуклы}.

  В случае точек максимума достаточность достигается при
  \emph{вогнутости} функции $f(x)$ и \emph{выпуклости} множества
  допустимых решений задачи.
\end{rem}

Может оказаться, что наличие в определённой точке экстремума не
удаётся доказать, опираясь лишь на теорему \ref{th:kuhn-tucker} и
замечание \ref{rem:kt-cond}. В таком случае задачу исследуют с
применением условий высших порядков, которые приведены далее.

В следующих теоремах используются полные дифференциалы функции
Лагранжа $d^2\La$ и ограничений $dg_j$, определяемые следующим
образом:
\begin{align*}
  d^2\La &= \sum_{p=1}^n\sum_{r=1}^n{\dpardiff{\La}{x_p}{x_r}\,dx_pdx_r} \\
  dg_j &= \sum_{p=1}^n{\pardiff{g_j}{x_p}\,dx_p}
\end{align*}

\begin{thm}[Необходимое условие экстремума второго порядка]
  \label{th:if-extr-2}
  Пусть точка $\hat{x}$ — регулярный экстремум в задаче
  \eqref{eq:cond-optim-problem-form}, удовлетворяющий условиям теоремы
  \ref{th:kuhn-tucker} при соответствующем $\lambda$. Тогда
  \begin{align*}
    d^2\La(\hat{x}) &\geq 0 \qquad \text{для минимума} \\
    d^2\La(\hat{x}) &\leq 0 \qquad \text{для максимума}
  \end{align*}
  при $dx ≠ 0$ таких, что
  \begin{align*}
    dg_j(\hat{x}) &= 0 \qquad \forall j: \lambda_j ≠ 0\\
    dg_j(\hat{x}) &\leq 0 \qquad \forall j: \lambda_j=0
  \end{align*}
\end{thm}

\begin{thm}[Достаточное условие экстремума первого порядка]
  \label{th:then-extr-1}
  Пусть $x^*$ является регулярной условно-стационарной точкой задачи
  \eqref{eq:cond-optim-problem-form}, а число активных в $x^*$
  ограничений равно числу переменных $n$. Тогда для наличия в $x^*$
  регулярного экстремума достаточно выполнения одного из следующих
  наборов неравенств при $\forall j: g_j(\hat{x}) = 0$:
  \begin{align*}
    \lambda_j &> 0 \qquad \text{для минимума} \\
    \lambda_j &< 0 \qquad \text{для максимума}
  \end{align*}
\end{thm}

\begin{thm}[Достаточное условие экстремума второго порядка]
  \label{th:then-extr-2}
  Пусть $x^*$ является регулярной условно-стационарной точкой задачи
  \eqref{eq:cond-optim-problem-form}. Если при $dx ≠ 0$ таких, что
    \begin{align*}
    dg_j(x^*) &= 0 \qquad \forall j: \lambda_j ≠ 0\\
    dg_j(x^*) &\leq 0 \qquad \forall j: \lambda_j=0
  \end{align*}
  выполняется неравенство $d^2\La(x^*) ≠ 0$, то $x^*$ — точка
  регулярного экстремума, причём
  \begin{align*}
    d^2\La(x^*) &> 0 \qquad \text{для минимума} \\
    d^2\La(x^*) &< 0 \qquad \text{для максимума}
  \end{align*}
\end{thm}

\clearpage
\subsubsection{Нахождение точного аналитического решения}

Рассмотрим методику применения изложенных в предыдущем разделе
теоретических соображений на практическом примере.

Требуется решить следующую задачу:
\begin{equation}
  \label{eq:cond-optim-problem-raw}
  \begin{cases}
    f(x) = (x_1+4)^2 + (x_2-4)^2 \to \extr \\
    2x_1 - x_2 \leq 2 \\
    x_1 \geq 0 \\
    x_2 \geq 0
  \end{cases}
\end{equation}

После приведения к каноническому виду она примет вид
\begin{equation}
  \label{eq:cond-optim-problem}
  \begin{cases}
    f(x) = (x_1+4)^2 + (x_2-4)^2 \to \extr \\
    g_1(x) = 2x_1 - x_2 - 2 \leq 0 \\
    g_2(x) = -x_1 \leq 0 \\
    g_3(x) = -x_2 \leq 0
  \end{cases}
\end{equation}

Заметим, что выполняется условие Слейтера \ref{th:slater}, так как в
качестве соответствующей точки $x_s$ можно взять, например, точку $(1,
1)$. Значит, в данной задаче достаточно рассмотреть лишь случай
классической функции Лагранжа с $\lambda_0=1$.

Составим функцию Лагранжа:
\begin{multline}
  \label{eq:lagrange}
  \La(x, \lambda) = (x_1+4)^2 + (x_2-4)^2 +\\
  + \lambda_1(2x_1-x_2-2)+\lambda_2(-x_1)+\lambda_3(-x_2)
\end{multline}

Запишем необходимые условия стационарности точки $x^*$ при
соответствующем векторе $\lambda$:
\begin{subequations}
  \renewcommand{\theequation}{\theparentequation\asbuk{equation}}
  \label{eq:kkt-conditions}
  \begin{equation}
    \label{eq:cond-stationary}
    \begin{cases}
      \pardiff{\La}{x_1} = 2(x^*_1+4)+2\lambda_1-\lambda_2=0\\
      \pardiff{\La}{x_2} = 2(x^*_2-4)-\lambda_1-\lambda_3=0
    \end{cases}
  \end{equation}
  \begin{equation}
    \label{eq:cond-sign}
    \sgn(\lambda_1) = \sgn(\lambda_2) = \sgn(\lambda_3)
  \end{equation}
  \begin{equation}
    \label{eq:cond-slackness}
    \begin{cases}
      \lambda_1\mul g_1(x^*) = \lambda_1\mul (2x^*_1-x^*_2-2) = 0\\
      \lambda_2\mul g_2(x^*) = \lambda_2\mul (-x^*_1) = 0\\
      \lambda_3\mul g_3(x^*) = \lambda_3\mul (-x^*_2) = 0
    \end{cases}
  \end{equation}
  \begin{equation}
    \begin{cases}
      \label{eq:cond-feasible}
      g_1(x^*) = 2x^*_1 - x^*_2 - 2 \leq 0 \\
      g_2(x^*) = -x^*_1 \leq 0 \\
      g_3(x^*) = -x^*_2 \leq 0
    \end{cases}
  \end{equation}
\end{subequations}

Рассмотрим $2³=8$ вариантов удовлетворения условий дополняющей
нежёсткости \eqref{eq:cond-slackness}.

\begin{enumerate}
\renewcommand{\labelenumi}{\Roman{enumi})}
\renewcommand{\labelenumiii}{\arabic{enumiii})}

\item $\lambda_1 = 0$
  
  В этом случае уравнения \eqref{eq:cond-stationary} принимают вид:
  \begin{equation}
    \label{eq:cond-stationary-l1=0}
    \begin{cases}
      2x^*_1+8-\lambda_2=0\\
      2x^*_2-8-\lambda_3=0
    \end{cases}
  \end{equation}
  \begin{enumerate}
  \item $\lambda_2 = 0$

    Из \eqref{eq:cond-stationary-l1=0} имеем $2x^*_1+8=0 \iff
    x^*_1=-4$, что не удовлетворяет условию $g_2(x^*) = -x^*_1 \leq 0$
    из \eqref{eq:cond-feasible}.
  \item $\lambda_2 ≠ 0$
    
    В этом случае из \eqref{eq:cond-slackness} следует, что
    $g_2(x^*)=0 \iff x^*_1 = 0$, откуда согласно
    \eqref{eq:cond-stationary-l1=0} получаем $\lambda_2=8$.
    \begin{enumerate}
    \item $\lambda_3 = 0$
      
      Из \eqref{eq:cond-stationary-l1=0} следует $x^*_2=4$. Получаем
      точку $A = (0, 4)$.
    \item $\lambda_3 ≠ 0$

      В данном случае согласно \eqref{eq:cond-slackness} находим
      $x^*_2=0$, поэтому из \eqref{eq:cond-stationary-l1=0} следует, что
      $\lambda_3=-8$. С учётом $\lambda_2=8$ заметим, что не
      выполняются условия \eqref{eq:cond-sign}.
    \end{enumerate}
  \end{enumerate}
\item $\lambda_1 ≠ 0$ 

  Согласно условию \eqref{eq:cond-slackness}, в
  данном случае
  \begin{equation}
    \label{eq:cond-slackness-l1n=0}
    2x^*_1-x^*_2-2=0
  \end{equation}
  \begin{enumerate}
  \item $\lambda_2 = 0$

    Из \eqref{eq:cond-stationary} получим
    \begin{equation}
      \label{eq:cond-stationary-l2=0}
      2x^*_1+8+2\lambda_1=0
    \end{equation}
    \begin{enumerate}
    \item $\lambda_3 = 0$

      Второе уравнение системы \eqref{eq:cond-stationary} даёт
      $2x^*_2-8-\lambda_1=0$. Сложим это уравнение с
      \eqref{eq:cond-stationary-l2=0} и рассмотрим его вместе с
      \eqref{eq:cond-slackness-l1n=0}, получив
      \begin{equation}
        \begin{cases}
          2x^*_1+4x^*_2-8=0\\
          2x^*_1-x^*_2-2=0
        \end{cases}
      \end{equation}
      
      Таким образом получим $4x^*_2-8=-x^*_2-2 \iff x^*_2=\frac{6}{5}$.

      Значение $x^*_1=\frac{8}{5}$ определяется из
      \eqref{eq:cond-slackness-l1n=0}. После этого из уравнения
      \eqref{eq:cond-stationary-l2=0} найдём значение $\lambda_1 =
      -\frac{28}{5}$. Итак, получена ещё одна точка $B =
      \left(\frac{8}{5}, \frac{6}{5}\right)$.
    \item $\lambda_3 ≠ 0$

      Согласно \eqref{eq:cond-slackness}, в данном случае $x^*_2=0$,
      поэтому из \eqref{eq:cond-slackness-l1n=0} следует $x^*_1=1$.
      Тогда из \eqref{eq:cond-stationary-l2=0} определим $\lambda_1 =
      -5$. Подставив найденные значения $x_2*$ и $\lambda_1$ в
      \eqref{eq:cond-stationary}, получим $\lambda_3 = -3$. Найдена
      очередная точка $C=(1, 0)$.
    \end{enumerate}
  \item $\lambda_2 ≠ 0$

    Из \eqref{eq:cond-slackness} получаем $x^*_1=0$, откуда с учётом
    \eqref{eq:cond-slackness-l1n=0} следует значение $x^*_2=-2$, не
    удовлетворяющее условию $g_3(x^*) = -x^*_2 \leq 0$ из
    \eqref{eq:cond-feasible}.
  \end{enumerate}
\end{enumerate}

Итак, найдены три точки, для которых выполнены необходимые условия
теоремы \ref{th:kuhn-tucker}. Тип возможного экстремума определяется
согласно знаку компонент $\lambda$.
\begin{itemize}
\item $A = (0, 4),\, \lambda=(0, 8, 0)$, минимум
\item $B = (\frac{8}{5}, \frac{6}{5}),\, \lambda=(-\frac{28}{5}, 0,
  0)$, максимум
\item $C = (1, 0),\, \lambda=(-5, 0, -3)$, максимум
\end{itemize}

\begin{figure}[!h]
  \centering
  \begin{tikzpicture}
    \begin{axis}[
      x=2.4 cm, y=1.3 cm,
      xmin=0, ymin=0, xmax=2.5,
      xlabel=$x_1$, ylabel=$x_2$,
      enlargelimits=0.05]
      \input{condextr-contours.tkz.tex}
      \addplot[very thick, red!50!black] coordinates{(0,5) (0,0) (1,0) (2.5,3)};
      \label{plot:boundaries}
      
      \node[circle,fill=black,scale=0.5,label=right:$A$] at (axis cs:0,4) {};
      \node[circle,fill=black,scale=0.5,label=below right:$B$] at (axis cs:1.6,1.2) {};
      \node[circle,fill=black,scale=0.5,label=right:$C$] at (axis cs:1,0) {};
    \end{axis}
  \end{tikzpicture}
  \caption{Границы $g_j(x)=0$, линии уровня
    целевой функции $f(x)$ и условно-стационарные точки задачи
    \eqref{eq:cond-optim-problem}}
  \label{fig:cond-optim}
\end{figure}

Поскольку целевая функция обладает положительно определённую матрицу
Гессе $\left( \begin{smallmatrix}2 & 0 \\ 0 & 2\end{smallmatrix}
\right)$, она является выпуклой в силу теоремы \ref{th:convex-f}.
Рассматриваемое множество (см. рис. \ref{fig:cond-optim}) также
является выпуклым, так как функции ограничений $g_j(x)$ выпуклы в силу
замечания \ref{rem:lin-f-convex}, а потому выполняются условия теоремы
\ref{th:convex-set}.

Таким образом, с учётом замечания \ref{rem:kt-cond}, для точки $A$
условия Куна—Таккера являются \emph{достаточными}, и она является
точкой \emph{локального минимума}. Линии уровня функции $f(x)$,
сходящиеся к $A$ на рисунке \ref{fig:cond-optim}, служат
дополнительным подтверждением этому.

В то же время, для $B$ и $C$ условия Куна—Таккера не являются
достаточными. Воспользуемся для их исследования условиями высших
порядков.

Второй дифференциал $d^2\La$ во всех точках одинаковый и имеет вид
\begin{equation}
  \label{eq:la-diff}
  d^2\La = 2dx_1^2 + 2 dx_2^2
\end{equation}

\begin{itemize}
\item $B = (\frac{8}{5}, \frac{6}{5})$

  В данной точке активно лишь ограничение $g_1$, так что
  воспользоваться достаточным условием первого порядка нельзя.
  Проверим необходимое условие второго порядка. Приравнивая к нулю
  дифференциал $dg_1$ активного ограничения и выбирая дифференциалы
  неактивных ограничений неположительными, получим
  \begin{align*}
    dg_1 &= 2dx_1-dx_2 = 0 \iff dx_2 = 2dx_1 \\
    dg_2 &= -dx_1 \leq 0 \\
    dg_3 &= -dx_2 \leq 0
  \end{align*}
  и с учётом этого рассмотрим \eqref{eq:la-diff} при условии $dx≠0$:
  \begin{equation*}
    d^2\La = 2dx_1^2 + 8dx_1^2 = 10dx_1^2
  \end{equation*}
  
  Полученная форма очевидно больше нуля. Поскольку в $B$ значения
  $\lambda < 0$, необходимое условие второго порядка теоремы
  \ref{th:if-extr-2} не выполняется, так что $B$ \emph{не} является
  точкой экстремума.

\item $C = (1, 0)$

  Активными являются ограничения $g_1$ и $g_3$, их количество равно
  числу переменных $n=2$, поэтому согласно теореме
  \ref{th:then-extr-1} выполняется достаточное условие экстремума
  первого порядка, и $C$ есть точка \emph{локального максимума}.

  Действительно, из рисунка \ref{fig:cond-optim-zoom} видно, что любое
  допустимое перемещение из точки $C$ приводит к переходу с линии
  уровня $f(C)=41$ на линии меньших уровней. Вектор градиента $f'(C) =
  \left(\begin{smallmatrix}\phantom{-}5\\-4\end{smallmatrix}\right)$ в
  данной точке образует с границами области тупые углы, так что
  возможные направления, вдоль которых функция растёт быстрее всего,
  будут направлениями убывания.
  
  \begin{figure}[!h]
    \centering
    \begin{tikzpicture}
      \begin{axis}[grid=both,x=15cm,y=15cm,
        xlabel=$x_1$, ylabel=$x_2$,
        enlargelimits=0.05]
        \input{condextr_zoom-contours.tkz.tex}
        
        \coordinate (C) at (axis cs:1,0);

        \draw[blue,thick,arrows=-triangle 45] (C) -- (axis cs:1.09375,-0.075);
        \addplot[very thick, red!50!black] coordinates{(0.85,0) (1,0) (1.1,0.2)};
        \node[circle,fill=black,scale=0.5,label=below:$C$] at (C) {};
      \end{axis}
    \end{tikzpicture}
    \caption{Направление вектора градиента $f'(C)$, линии уровня
      функции $f(x) = (x_1+4)^2+(x_2-4)^2$ и границы $g_j(x)=0$ вблизи
      точки $C$}
    \label{fig:cond-optim-zoom}
  \end{figure}
\end{itemize}

\clearpage
\subsection{Метод возможных направлений Зойтендейка}
\label{sec:zoutendijk}

Метод возможных направлений представляет собой итерационный метод
решения задачи \eqref{eq:cond-optim-problem-form} при линейных
ограничений. Поиск начинается в допустимой точке пространства решений
и продолжается по траектории, обеспечивающей улучшение значения
целевой функции, но не выходящей за границы допустимой области.

Пусть в \eqref{eq:cond-optim-problem-form} функции ограничений
$g_j(x)$ линейны или линеаризованы (с помощью разложения в ряд
Тейлора) так, что получена следующая задача (рассматривается случай
минимаизации):
\begin{equation}
  \label{eq:zoutendijk-problem-form}
  \begin{cases}
    f(x) \to \min \\
    \scalmult{a^j,x} \leq b_j,\, j=\overline{1,m} \\
    x_i \geq 0,\, i=\overline{1,n} \\
    x \in \set{R}^n
  \end{cases}
\end{equation}

Теперь сформулируем алгоритм метода возможных направлений.

Прежде всего отметим, что на каждом шаге очередное приближение к точке
минимума находится по формуле
\begin{equation}
  \label{eq:zoutendijk-iter}
  x^{k+1} = x^k + s_k \mul d^k
\end{equation}
причём начальное приближение $x^1$ выбирается из числа точек,
удовлетворяющих условиям \eqref{eq:zoutendijk-problem-form}.

\begin{enumerate}
  \renewcommand{\labelenumi}{\textbf{Шаг \arabic{enumi}.}}
\item Определяются множества номеров активных ограничений $I^k$ — для
  ограничений вида $x_i \geq 0$ и $J^k$ — для ограничений вида
  $\scalmult{a^j,x} \leq b_j$.
\item Если $I^k=J^k = \varnothing$, то $d^k=-f'(x^k)$. В противном
  случае в векторе $d^k$ каждую $i$-ю координату
  представляют\footnote{Верхние индексы опущены для компактности} как
  $d_i$ для $i \in I^k$ и $d_{i+}-d_{i-}$ для $i \notin I^k$, после
  чего решают следующую задачу линейного программирования:
  \begin{equation}
    \label{eq:zoutendijk-linprog}
    \begin{cases}
      \scalmult{f'(x^k), d^k} = 
      \suml_{i\in I^k}{\pardiff{f(x^k)}{x_i}d_i} + 
      \suml_{i \notin I^k}{\pardiff{f(x^k)}{x_i}(d_{i+}-d_{i-})}
      \to \min\\
      
      \scalmult{a^j, d^k} = 
      \suml_{i\in I^k}{a^j_id_i} +
      \suml_{i \notin I^k}{a^j_i(d_{i+}-d_{i-})} \leq 0,
      \, i\in J^k \\
      
      \suml_{i=1}^n\abs{d_i^k} = \suml_{i\in I^k}{d_i}+
      \suml_{i \notin I^k}{\left( d_{i+} + d_{i-}\right)} \leq 1\\
      
      d_{i+} \geq 0,\, d_{i-} \geq 0,\,d_{i+}\mul d_{i-}=0,\, i\notin I^k
    \end{cases}
  \end{equation}
\item Если выполняется условие
  \begin{equation}
    \label{eq:zoutendijk-halt}
    \scalmult{f'(x^k), d^k} = 0
  \end{equation}
  то дальнейшее улучшение решения невозможно и процесс прерывается.

  В противном случае определяют величину шага
  \begin{equation}
    \label{eq:zoutendijk-step}
    s_k =
    \min\{s_*,\hat{s}_1,\dotsc,\hat{s}_n,\bar{s}_1,\dotsc,\bar{s}_m\}
  \end{equation}
  где компоненты под $\min$ находятся по следующим правилам:
  \begin{itemize}
  \item $s_*$ — решение задачи безусловной одномерной оптимизации
    \begin{equation*} 
      f(x^k+s_* \mul d^k) \to \min
    \end{equation*}

  \item $\hat{s}_i$ — наибольший шаг, при котором ограничение
    \begin{equation*}
      x^k+\hat{s}_i d^k_i \geq 0
    \end{equation*}
    остаётся удовлетворённым. Если компонента найденного вектора
    направления $d^k_i \geq 0$, то $\hat{s}_i = +\infty$ (так что
    соответствующая компонента $\hat{s}_j$ согласно
    \eqref{eq:zoutendijk-step} никак не ограничивает выбор $s_k$),
    иначе
    \begin{equation*}
      \hat{s}_i = -\frac{x^k_i}{d^k_i}
    \end{equation*}

  \item $\bar{s}_j$ — наибольший шаг, при котором при движении из
    $x^k$ по направлению $d_k$ ограничение
    \begin{equation*}
      \scalmult{a^j, x^k+\bar{s}_j d^k} \leq b_j
    \end{equation*}
    остаётся выполненным. Если $\scalmult{a^j, d^k} \leq 0$, то
    ограничение будет удовлетворено при любом шаге, поэтому $\bar{s}_j
    = +\infty$. В противном случае, $\bar{s}_j$ определяется как
    \begin{equation*}
      \bar{s}_j = \frac{b_j-\scalmult{a^j, x^k}}{\scalmult{a^j, d^k}}
    \end{equation*}
  \end{itemize}
  Отметим, что $s_k>0$.
\item С учётом шага $s_k$ и направления $d^k$ согласно
  \eqref{eq:zoutendijk-iter} полагают новое приближение равным
  \begin{equation*}
    x^{k+1} = x^k + s_k \mul d^k
  \end{equation*}
  и переходят к следующей итерации.
\end{enumerate}

Таким образом, критерием останова процесса оптимизации $k$-й итерации
является выполнение равенства \eqref{eq:zoutendijk-halt}, что
равносильно невозможности дальнейшего уменьшения значения целевой
функции при заданных ограничениях.

\subsubsection{Применение метода Зойтендейка}

Рассмотрим работу метода возможных направлений на примере задачи
\eqref{eq:cond-optim-problem-raw}. Будем искать точку минимума.
Ограничения уже линейны, поэтому дополнительная линеаризация не
требуется. Задача уже приведена к виду
\eqref{eq:zoutendijk-problem-form}:
\begin{equation}
  \label{eq:zoutendijk-problem}
  \begin{cases}
    f(x) = (x_1+4)^2 + (x_2-4)^2 \to \min \\
    2x_1-x_2 \leq 2 \\
    x_1 \geq 0 \\
    x_2 \geq 0
  \end{cases}
\end{equation}

\begin{enumerate}
  \renewcommand{\labelenumi}{\textbf{Итерация \arabic{enumi}.}}
  \renewcommand{\labelenumii}{\textbf{Шаг \arabic{enumii}.}}
\item
  В качестве первого приближения $x^1$ возьмём точку $(1, 1)$, очевидно
  удовлетворяющую всем ограничениям.
  \begin{enumerate}
  \item Ни одно из ограничений не активно в данной точке, поэтому
    множества $I^k = J^k = \varnothing$.
  \item Направление движения определим как противоположное вектору
    градиента:
    \begin{equation*}
      d^1 = -f(x^1) = \begin{pmatrix} -10 \\ \phantom{-}6 \end{pmatrix}
    \end{equation*}
  \item Определим коэффициенты $s_*, \hat{s}_1, \hat{s}_2, \bar{s}_1$.
    \begin{itemize}
    \item Минимум функции $f(x^1+s_*\mul d^1) = (5-10s_*)^2+(6s_*-3)^2$
      достигается при $s_*=\frac{1}{2}$.
    \item $d^1_1 = -10 \ngeq 0 \implies \hat{s}_1 =
      -\frac{x^1_1}{d^1_1} = -\frac{1}{-10} = \frac{1}{10}$
    \item $d^1_2 = 6 \geq 0 \implies \hat{s}_2 = +\infty$
    \item $\scalmult{a^1, d^1} = -26 \leq
      0 \implies \bar{s}_1 = +\infty$
    \end{itemize}
    Находим величину шага: $s_1 = \min\{\frac{1}{2}, \frac{1}{10}\} =
    \frac{1}{10}$.
  \item Вычисляем новое приближение $x^2$:
    \begin{equation*}
      x^2 = \begin{pmatrix} 1 \\ 1 \end{pmatrix} +
      \frac{1}{10} \begin{pmatrix} -10 \\ \phantom{-}6 \end{pmatrix}
      = \begin{pmatrix} 0 \\ \frac{8}{5} \end{pmatrix}
    \end{equation*}
  \end{enumerate}
\item Продолжим поиск минимума из точки
  \begin{equation*}
    x^2 = \begin{pmatrix} 0 \\ \frac{8}{5} \end{pmatrix}
  \end{equation*}
  \begin{enumerate}
  \item В $x^2$ активно лишь ограничение $x_1\geq 0$, поэтому $I^2 =
    \varnothing, J^2 = \{1\}$.
  \item Составим задачу линейного программирования:
    \begin{equation*}
      \begin{cases}
        \scalmult{f'(x^2), d^2} = \tilde{f}_2 =
        8d_1 - \frac{3}{5} (d_{2+}-d_{2-}) \to \min \\
        d_1+(d_{2+}+d_{2-}) \leq 1 \\
        d_1, d_{2+}, d_{2-} \geq 0 \\
        d_{2+} \mul d_{2-} = 0
      \end{cases}
    \end{equation*}
    Приведём её к каноническому виду, получив в первом ограничении
    равенство путём добавления переменной $x_1$:
    \begin{equation*}
      \begin{cases}
        \tilde{f}_2 = 8d_1-\frac{3}{5}(d_{2+}-d_{2-}) \to \min \\
        d_1+(d_{2+}+d_{2-}) + x_1 = 1 \\
        d_1, d_{2+}, d_{2-}, x_1 \geq 0 \\
        d_{2+} \mul d_{2-} = 0
      \end{cases}
    \end{equation*}
    Решим задачу симплекс-методом. Составим матрицу коэффициентов
    ограничений и целевой функции, затем исключим $x_1$ из состава
    базисных переменных:
    \begin{equation}
      \begin{gmatrix}[b]
        \dagger & d_1 & d_{2+} & d_{2-} & x_1 & \diamond\\
        x_1 & 1 & 1 & 1 & 1 & \mathbf{1} \\
        \tilde{f}_2 & -8 & \mathbf{\frac{3}{5}} & -\frac{3}{5} & 0 & 0
        \rowops \add[-\frac{3}{5}]{1}{2}
      \end{gmatrix}
    \end{equation}
    Получим матрицу с одними отрицательными коэффициентами в нижнем
    ряду:
    \begin{equation*}
      \begin{gmatrix}[b]
        \dagger & d_1 & d_{2+} & d_{2-} & x_1 & \diamond\\
        x_1 & 1 & 1 & 1 & 1 & 1 \\
        \tilde{f}_2 & -\frac{43}{5} & 0 & -\frac{6}{5} & -\frac{3}{5}
        & -\frac{3}{5}
      \end{gmatrix}
    \end{equation*}
    При этом условие $d_{2+} \mul d_{2-} = 0$ выполнено. Итак, найдено
    оптимальное решение задачи линейного программирования:
    \begin{equation*}
      \begin{cases}
        d_{1\phantom{+}} = 0 \\
        d_{2+} = 1 \\
        d_{2-} = 0
      \end{cases}
    \end{equation*}
    с учётом которого построим вектор $d^2$:
    \begin{equation*}
      d^2 = \begin{pmatrix} 0 \\ 1 \end{pmatrix}
    \end{equation*}
  \item Определим коэффициенты $s_*, \hat{s}_1, \hat{s}_2, \bar{s}_1$.
    \begin{itemize}
    \item Минимум $f(x^2+s_*\mul d^2) = 16+(s_*-\frac{32}{5})^2$
      достигается при $s_* = \frac{32}{5}$.
    \item $d^2_1 = 0 \geq 0 \implies \hat{s}_1 = +\infty$
    \item $d^2_1 = 1 \geq 0 \implies \hat{s}_1 = +\infty$
    \item $\scalmult{a^1, d^2} = -1 \leq 0 \implies \bar{s}_1 =
      +\infty$ Итак, ограничения никак не влияют на величину шага,
      поэтому $s_k = \frac{32}{5}$.
    \end{itemize}
  \item Найдём приближение $x^3$:
    \begin{equation*}
      x^3 = \begin{pmatrix} 0 \\ \frac{8}{5} \end{pmatrix} +
      \frac{32}{5} \begin{pmatrix} 0 \\ 1 \end{pmatrix}
      = \begin{pmatrix} 0 \\ 4 \end{pmatrix}
    \end{equation*}
  \end{enumerate}
\item Как известно из точного аналитического решения (см. раздел
  \ref{sec:kuhn-tucker}), точка $(0, 4)$ является точным решением
  задачи \eqref{eq:zoutendijk-problem}. Значит, удовлетворение
  \eqref{eq:zoutendijk-halt} на этой итерации должно сигнализировать
  об этом. Выполним первые два шага алгоритма, что убедиться в этом.
  \begin{enumerate}
  \item Как и на прошлой итерации, $I^2 = \varnothing, J^2 = \{1\}$.
  \item Для нахождения компонент вектора $d^k$ решим задачу линейного
    программирования:
    \begin{equation*}
      \begin{cases}
        \scalmult{f'(x^3), d^3} = \tilde{f}_3 =
        8d_1 \to \min \\
        d_1+(d_{2+}+d_{2-}) + x_1 = 1 \\
        d_1, d_{2+}, d_{2-}, x_1 \geq 0 \\
        d_{2+} \mul d_{2-} = 0
      \end{cases}
    \end{equation*}
    оптимальным решением которой является тривиальный набор
    \begin{equation*}
      \begin{cases}
        d_{1\phantom{+}} = 0 \\
        d_{2+} = 0 \\
        d_{2-} = 0
      \end{cases}
    \end{equation*}

    Таким образом,
    \begin{equation*}
      d^3 = \begin{pmatrix} 0 \\ 0 \end{pmatrix}
    \end{equation*}

    Очевидно, что выполняется критерий останова
    \eqref{eq:zoutendijk-halt}. Значит, далее уменьшать значение
    $f(x)$ в \eqref{eq:zoutendijk-problem} не представляется
    возможным.
  \end{enumerate}
\end{enumerate}

Итак, с помощью метода возможных направлений Зойтендейка найдено
решением задачи условной минимизации \eqref{eq:zoutendijk-problem},
которое согласуется с решением, найденным в \ref{sec:kuhn-tucker}.

\begin{figure}[!h]
  \centering
  \begin{tikzpicture}
    \begin{axis}[
      x=2.4 cm, y=1.3 cm,
      xmin=0, ymin=0, xmax=2.5,
      xlabel=$x_1$, ylabel=$x_2$,
      enlargelimits=0.05]
      \input{condextr-contours.tkz.tex}
      \addplot[very thick, dashed, red!50!black] coordinates{(0,5) (0,0) (1,0) (2.5,3)};
      
      \addplot[mark=*, only marks,
               mark options={fill=blue!70!black}] coordinates{(1,1)
                 (0,1) (0, 4)};
      
      \addplot[blue, thick, mark=none, arrows=-triangle 45] coordinates{(1,1) (0,1)};
      \addplot[blue, thick, mark=none, arrows=-triangle 45] coordinates{(0,1) (0,4)};      
    \end{axis}
  \end{tikzpicture}
  \caption{Ход решения задачи \eqref{eq:zoutendijk-problem} методом
    Зойтендейка}
  \label{fig:cond-optim}
\end{figure}


\clearpage
\subsection{Метод штрафных функций}
\label{sec:penalty}

\subsubsection{Общие сведения}

В разделе \ref{sec:kuhn-tucker} рассматривалась задача условной
оптимизации \eqref{eq:cond-optim-problem-form}, были предложены
соображения по её \emph{аналитическому} решению с использованием
теоремы Куна—Таккера \ref{th:kuhn-tucker}. Реализация аналитических
машинных алгоритмов традиционно не является простой задачей, и в
большинстве случаев на практике оказывается более выгодным (во всех
отношениях) использование численных методов.

Методы, построенные на применении штрафных функций, позволяют решать
задачи на поиск минимума при наличии ограничений \emph{численно}.

Общая схема таких методов заключается в замене задачи условной
минимизации в некоторой области $D$, заданной ограничениями типа
неравенств
\begin{equation*}
  \begin{cases}
    f(x) \to \min \\
    g_j(x) \leq 0,\, j=\overline{1,m} \\
    x \in \set{R}^n
  \end{cases}
\end{equation*}
на задачу безусловной минимизации
\begin{equation}
  \label{eq:penalty-problem-form}
  P(x, r) = f(x) + \phi(x, p) \to \min
\end{equation}
где \neword{штраф} $\phi(x, p)$ удовлетворяет свойству
\begin{align*}
  \phi(x, p) &= 0 \quad x \in D\\
  \phi(x, p) &\gg 0 \quad x \notin D
\end{align*}
Таким образом, штрафная часть значительно возрастает при выходе за
пределы допустимого множества $D$.

Один из общих подходов к использованию штрафных функций заключается в
последовательном решении задач вида \ref{eq:penalty-problem-form},
параметризованных номером итерации $k$:
\begin{equation}
  \label{eq:penalty-iter}
  P_k(x, r) = f(x) + \phi(x, p_k) \to \min  
\end{equation}
где функция $\phi(x, p_k)$ масштабируется на каждой итерации в сторону
увеличения или уменьшения.

\subsubsection{Метод внешних штрафных функций}

Методами внешних штрафных функций называют методы, использующие схему
\eqref{eq:penalty-iter} таким образом, что $\phi(x, p_k)$ с каждой
итерацией увеличивается, и положения вне допустимой области с ростом
$k$ становятся всё менее «выгодными». При этом начальное приближение
$x^1$ к точке минимума выбирается \emph{вне} $D$.

Один из таких методов, изложенный в \cite{himmelblau75},
предусматривает построение штрафной функции для задачи
\eqref{eq:cond-optim-problem-form} в виде:
\begin{equation}
  \label{eq:weisman}
  \phi(x, p_k) = p_k \sum_{j=1}^m{ \left [ g_j^+(x) \right ]^2}
\end{equation}
где $g_j^+(x) = \max\{0, g_j(x)\}$, так что штрафная часть отлична от
нуля вне границ, заданных ограничениями $g_j(x)$, а
$\lim\limits_{k\to\infty}{p_k} = +\infty$.

В качестве критерия останова процесса \eqref{eq:penalty-iter} можно
выбрать условие
\begin{equation*}
  \phi(x^k, p_k) < \epsilon
\end{equation*}
где $x^k$ — приближение к точке минимума на $k$-ом шаге.

\subsubsection{Применение метода}
\label{sec:penalty-usage}

В разделе \ref{sec:kuhn-tucker} была решена задача условной
оптимизации \eqref{eq:cond-optim-problem}:
\begin{equation*}
  \begin{cases}
    f(x) = (x_1+4)^2 + (x_2-4)^2 \to \extr \\
    g_1(x) = 2x_1 - x_2 - 2 \leq 0 \\
    g_2(x) = -x_1 \leq 0 \\
    g_3(x) = -x_2 \leq 0
  \end{cases}
\end{equation*}
Среди условных экстремумов функции был найден минимум в точке $A = (0,
4)$. Найдём его с помощью штрафных функций. Преобразуем задачу к виду
\label{eq:penalty-iter}, выбирая штрафную часть в виде
\eqref{eq:weisman}:
\begin{equation}
  \label{eq:penalty-problem}
  P(x) = f(x) + p \sum_{j=1}^3{ \left [ g_j^+(x) \right ]^2} \to \min
\end{equation}

На рисунках \ref{fig:pen-contours1}-\ref{fig:pen-contours4}
представлены линии уровня функции $P(x)$ задачи
\eqref{eq:penalty-problem} при различных значениях $p$ (при $p=0$
получается исходная функция $f(x)$ задачи
\eqref{eq:cond-optim-problem}). Видно, что с ростом $p$ влияние
штрафной части вне $D$ (\ref{plot:pen-boundaries}) значительно
возрастает, тогда как в допустимой области оптимизируемая функция не
претерпевает никаких изменений.

\pgfplotsset{every axis/.append style={xmin=-5,xmax=5,ymin=-5,ymax=5}}
\begin{figure}[!thb]
  \centering
  \begin{tikzpicture}
    \begin{axis}[xlabel=$x_1$, ylabel=$x_2$]
      \input{condextr_full-contours.tkz.tex}
      \addplot[thick,densely dashed, red!50!black] coordinates{(0,5)
        (0,0) (1,0) (3.5,5)};
      \label{plot:pen-boundaries}
    \end{axis}      
  \end{tikzpicture}
  \caption{Линии уровня функции $P(x)$ при $p=0$}
  \label{fig:pen-contours1}
\end{figure}

\begin{figure}[!thb]
  \centering
  \begin{tikzpicture}
    \begin{axis}[xlabel=$x_1$, ylabel=$x_2$]
      \input{penalty_mini-contours.tkz.tex}
      \addplot[thick,densely dashed, red!50!black] coordinates{(0,5)
        (0,0) (1,0) (3.5,5)};
    \end{axis}      
  \end{tikzpicture}
  \caption{Линии уровня функции $P(x)$ при $p=5$}
    \label{fig:pen-contours2}
\end{figure}

\begin{figure}[!thb]
  \centering
  \begin{tikzpicture}
    \begin{axis}[xlabel=$x_1$, ylabel=$x_2$]
      \input{penalty-contours.tkz.tex}
      \addplot[thick,densely dashed, red!50!black] coordinates{(0,5)
        (0,0) (1,0) (3.5,5)};
    \end{axis}      
  \end{tikzpicture}
  \caption{Линии уровня функции $P(x)$ при $p=50$}
  \label{fig:pen-contours3}
\end{figure}

\begin{figure}[!thb]
  \centering
  \begin{tikzpicture}
    \begin{axis}[xlabel=$x_1$, ylabel=$x_2$]
      \input{penalty_maxi-contours.tkz.tex}
      \addplot[thick,densely dashed, red!50!black] coordinates{(0,5)
        (0,0) (1,0) (3.5,5)};
    \end{axis}      
  \end{tikzpicture}
  \caption{Линии уровня функции $P(x)$ при $p=150$}
  \label{fig:pen-contours4}
\end{figure}

Для численного решения задачи \eqref{eq:penalty-problem} теперь уже
безусловной минимизации воспользуемся описанным ранее градиентным
методом с чебышёвскими функциями релаксации, выбирая в качестве
начального приближения точку вне области $D$.

Индекс $k$ в \eqref{eq:penalty-problem} опущен не случайно: опыт
показывает, что при достаточно большом начальном значении параметра
$p_0$ использование итерационного процесса \eqref{eq:penalty-iter}
может и не понадобиться, поскольку точка минимума $A$ с достаточной
точностью локализуется уже на первом шаге. Это проиллюстрировано на
рисунке \ref{fig:penalty}.

\begin{figure}[!thb]
  \centering
  \begin{tikzpicture}
    \begin{axis}[xlabel=$x_1$, ylabel=$x_2$, x=1 cm, y=1 cm]
      \addplot[thick,densely dashed, red!50!black] coordinates{(0,5)
        (0,0) (1,0) (3.5,5)};
      \input{penalty-contours.tkz.tex}

      \input{penalty_relch_-2,-3_20_14-trace.tkz.tex}
      \input{penalty_relch_2,-4_20_10-trace.tkz.tex}

      \node[circle,fill=black,scale=0.5,label=right:$A$] at (axis cs:0,4) {};
    \end{axis}      
  \end{tikzpicture}
  \caption{Ход процесса минимизации \eqref{eq:penalty-problem} при
    $p=50$ и различном выборе начальной точки}
  \label{fig:penalty}
\end{figure}
