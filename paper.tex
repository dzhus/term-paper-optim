\documentclass{article}
\usepackage[utf8x]{inputenc}
\usepackage[english,russian]{babel}
\usepackage{amsmath,amssymb}

% Rich title
\usepackage{titling}

% Matrix operations
\usepackage{gauss}

% Russian traditions
\renewcommand{\epsilon}{\varepsilon}
\renewcommand{\phi}{\varphi}
\renewcommand{\leq}{\leqslant}
\renewcommand{\geq}{\geqslant}
\newcommand{\intl}{\int\limits}
\usepackage{misccorr}

% Bib in TOC
\usepackage[numbib,nottoc]{tocbibind}

\usepackage{tocbibind}

% Custom commands
\providecommand{\program}[1]{{\tt #1}}

\numberwithin{equation}{section}

\usepackage[pdftex,unicode]{hyperref}

\begin{document}
\author{Дмитрий Джус}
\title{Курсовая работа по теме \\
  \Huge{«Методы оптимизации»}}
\pretitle{\begin{center}\LARGE}
\posttitle{\par\end{center}\vskip 3pc}
\date{}
\maketitle
\thispagestyle{empty}

\clearpage
\tableofcontents

\clearpage
\section{Введение}
Настоящая курсовая работа посвящена методам оптимизации и
экстремальным задачам.

В первой части рассмотрено решение задачи условной минимизации
линейной функции с помощью симплекс-метода.

Кроме того, изложено описание градиентного метод многопараметрической
оптимизации с чебышёвской функцией релаксации, предложена реализация
метода на алгоритмическом языке и представлены результаты работы на
тестовых функциях.

\section{Симплекс-метод}

Симплекс-метод позволяет решать задачи линейного программирования,
заключающиеся в минимизации целевого линейного функционала при
заданных линейных ограничениях. Основной идеей симплекс-метода
является перебор вершин выпуклого многогранника в многомерном
пространстве.

К решению предлагается следующая задача:
\begin{equation}
  \label{eq:lp-initial}
  \begin{cases}
    f(x) = -3x_1+2x_2-2x_3+2x_4-x_5 \to \min \\
    -x_1+x_2-x_3=1 \\
    -x_2+x_3+x_4=1 \\
    \phantom{-}x_2+x_3+x_5=2 \\
    \phantom{-}x_i \geq 0,\, i = \overline{1, 5}
  \end{cases}
\end{equation}

Для того, чтобы матрица коэффициентов ограничений содержала в себе
единичную, воспользуемся методом искусственных переменных, добавив в
первое ограничение переменную $x_6$ так, что оно примет вид
\begin{equation*}
  -x_1+x_2-x_3+x_6=1 \\
\end{equation*}

Заменим минимизируемую функцию на $\tilde{f}(x) = x_6$ и выразим её
через свободные переменные $x_1, x_2, x_3$, получив
\mbox{$\tilde{f}(x) = x_1-x_2+x_3+1$}. Перейдём к рассмотрению
вспомогательной задачи
\begin{equation}
  \begin{cases}
    \tilde{f}(x) = x_1-x_2+x_3+1 \to \min \\
    -x_1+x_2-x_3+x_6=1 \\
    -x_2+x_3+x_4=1 \\
    \phantom{-}x_2+x_3+x_5=2 \\
    \phantom{-}x_i \geq 0,\, i = \overline{1, 6}
  \end{cases}
\end{equation}

Решим её симплекс-методом. Составим расширенную матрицу из
коэффициентов ограничений и целевой функции:
\begin{equation}
  \begin{gmatrix}[b]
    \dagger & x_1 & x_2 & x_3 & x_4 & x_5 & x_6 & \diamond\\
    x_6 & -1 &  1 & -1 & 0 & 0 & 1 & \mathbf{1} \\
    x_4 &  0 & -1 &  1 & 1 & 0 & 0 & 1 \\
    x_5 &  0 &  1 &  1 & 0 & 1 & 0 & 2 \\
    \tilde{f}    & -1 &  \mathbf{1} & -1 & 0 & 0 & 0 & 1
  \end{gmatrix}
\end{equation}

Базисными переменными являются $x_4, x_5, x_6$. Для дальнейшего
решения необходимо исключить искусственную переменную $x_6$ из состава
базисных. 

Выберем в качестве ведущего столбец с максимальным элементом в
последней строке — это столбец переменной $x_2$, поскольку $\max\{1,
-1, 0\} = 1$. С помощью симплексного отношения выберем первую строку в
качестве ведущей, поскольку $\min\left\{\frac{1}{1},
  \frac{2}{1}\right\} = 1$.

Учитывая выбранный ведущий элемент, переведём переменную $x_2$ в
состав базисных взамен $x_6$. Сложим со второй строкой первую, а с
третьей и четвёртой — первую, умноженную на $-1$:
\begin{equation*}
  \begin{gmatrix}[b]
    \dagger & x_1 & x_2 & x_3 & x_4 & x_5 & x_6 & \diamond\\
    x_6 & -1 &  \mathbf{1} & -1 & 0 & 0 & 1 & 1 \\
    x_4 &  0 & -1 &  1 & 1 & 0 & 0 & 1 \\
    x_5 &  0 &  1 &  1 & 0 & 1 & 0 & 2 \\
    \tilde{f} & -1 &  1 & -1 & 0 & 0 & 0 & 1
    \rowops
    \add{1}{2}
    \add[-1]{1}{3}
    \add[-1]{1}{4}
  \end{gmatrix}
\end{equation*}

Получим матрицу с базисными переменными $x_2, x_4, x_5$:
\begin{equation}
  \label{eq:lp-artif}
  \begin{bmatrix}
    \dagger & x_1 & x_2 & x_3 & x_4 & x_5 & x_6 & \diamond\\
    x_2 & -1 &  1 & -1 & 0 & 0 & 1 &  1\\
    x_4 & -1 &  0 &  0 & 1 & 0 & 0 &  2\\
    x_5 &  1 &  0 &  2 & 0 & 1 & 0 &  1\\
    \tilde{f} & 0 &  0 & 0 & 0 & 0 & -1 & 0
  \end{bmatrix}
\end{equation}

При этом значение функции $\tilde{f}(x)$ равно нулю, что свидетельствует
о том, что найденное решение вспомогательной задачи является
допустимым для исходной.

Отбросим искусственную переменную $x_6$ и вернёмся к начальной задаче
\eqref{eq:lp-initial}. Исключим из исходной целевой функции $f(x)$
базисные переменные $x_2, x_4$ и $x_5$, используя ограничения из
матрицы \eqref{eq:lp-artif}:
\begin{equation*}
  \begin{cases}
    -x_1+x_2-x_3=1\\
    -x_1+x_4=2\\
    \phantom{-}x_1+2x_3+x_5 = 1
  \end{cases}
\end{equation*}

Тогда $f(x)$ принимает вид
\begin{equation*}
  f(x) = 2x_1+2x_3+5
\end{equation*}

Используем эту целевую функцию в матрице ограничений \eqref{eq:lp-artif}:
\begin{equation*}
  \begin{bmatrix}
    \dagger & x_1 & x_2 & x_3 & x_4 & x_5 & \diamond\\
    x_2 & -1 & 1 & -1 & 0 & 0 & 1\\
    x_4 & -1 & 0 &  0 & 1 & 0 & 2\\
    x_5 &  1 & 0 &  2 & 0 & 1 & 1\\
    \tilde{f} & -2 & 0 & -2 & 0 & 0 & 5
  \end{bmatrix}
\end{equation*}

В нижнем ряду под столбцами переменных все коэффициенты отрицательны,
что свидетельствует о том, что найдено оптимальное решение:

\begin{equation}
  \begin{cases}
    x_1 = 0\\
    x_2 = 1\\
    x_3 = 0\\
    x_4 = 2\\
    x_5 = 1
  \end{cases}
\end{equation}

\end{document}
