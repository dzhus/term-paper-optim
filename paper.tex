\documentclass[titlepage]{article}
\usepackage[utf8x]{inputenc}
\usepackage[english,russian]{babel}
\usepackage{amsmath,amsthm,amssymb}
\usepackage[mathscr]{eucal}

% Save trees
\usepackage[top=2.5cm, bottom=3cm, left=6cm, right=3.5cm]{geometry}

% Rich title
\usepackage{titling}

% Contours for plot labels
\usepackage[auto]{contour}
\contourlength{1pt}
\contournumber{8}

% Matrix operations
\usepackage{gauss}

% PGF/TikZ and pgfplots
\usepackage{tikz}
\usetikzlibrary{calc}
\usepackage{pgfplots}
\usetikzlibrary{patterns,arrows}
\pgfplotsset{every axis/.append style={enlargelimits=0.05}}
\pgfplotsset{every axis/.append style={grid=both}}
\pgfplotsset{every axis grid/.append style={densely dotted}}
\pgfplotsset{every tick/.append style={black}}

% Tables
\usepackage{pgfplotstable}
\pgfkeys{/pgf/number format/.cd, set thousands separator={\,}}
\usepackage{booktabs}

% Listings
\usepackage{listings}
\lstdefinelanguage[plt]{Scheme}
        {morekeywords={define,let,map,lambda,if,cond,or,and,else},
         morecomment=[l]{;},
         morestring=[b]{"}}
\lstset{mathescape=true}
\lstset{language=[plt]Scheme}
\lstset{basicstyle=\footnotesize\sffamily}
\lstset{breakautoindent=true}

% Russian traditions
\renewcommand{\epsilon}{\varepsilon}
\renewcommand{\phi}{\varphi}
\renewcommand{\leq}{\leqslant}
\renewcommand{\geq}{\geqslant}
\renewcommand{\nleq}{\nleqslant}
\renewcommand{\ngeq}{\ngeqslant}
\newcommand{\intl}{\int\limits}
\newcommand{\suml}{\sum\limits}
\usepackage{misccorr}

% Some running headers
\usepackage{fancyhdr}
\pagestyle{fancy}
\renewcommand{\sectionmark}[1]{\markboth{\thesection.\ #1}{}}
\renewcommand{\subsectionmark}[1]{\markright{#1}}
\fancyhead{}\fancyfoot{}
\lhead{{\slshape\leftmark}}
\rhead{{\slshape\rightmark}\quad{\rm \thepage}}
% Switching of all headers and footers for «front matter»-like pages
% such as TOC or LOF
\pagestyle{empty}

% Better \paragraph command with small caps and line skip
\makeatletter
\renewcommand{\paragraph}{\@startsection{paragraph}{4}{\z@}{-\baselineskip}{0.25\baselineskip}{\scshape}}
\makeatother

\usepackage{wrapfig}

% Subfigures
\usepackage{subfigure}
\renewcommand{\thesubfigure}{\asbuk{subfigure})~}

% Nice captions
\usepackage[format=hang]{caption}

% Bib in TOC
\usepackage[numbib,nottoc,notlof]{tocbibind}

% Dashing out
\usepackage{cancel}

% Custom commands
\newcommand{\program}[1]{{\tt #1}}
\newcommand{\filename}[1]{{\tt #1}}
\newcommand{\relch}{\textsc{relch}}
\newcommand{\gdrelch}{\textsc{gdrelch}}
\newcommand{\xrelch}{\textsc{xrelch2}}
\newcommand{\gd}{\textsc{пгс}}
\newcommand{\rgd}{\textsc{пгс}$_2$}
\newcommand{\name}{\textsc}
\newcommand{\La}{\mathscr{L}}
\newcommand{\neword}{\emph}
\newcommand{\pardiff}[2]{\frac{\partial{#1}}{\partial{#2}}}
\newcommand{\dpardiff}[3]{\frac{\partial^2{#1}}{\partial{#2}\partial{#3}}}
\newcommand{\abs}[1]{\left \lvert{#1}\right \rvert}
\newcommand{\norm}[1]{\left \lVert{#1}\right \rVert}
\newcommand{\set}[1]{\mathbb{#1}}
\newcommand{\mul}{\cdot}
\newcommand{\scalmult}[1]{{\left \langle #1 \right \rangle}}
\newcommand{\at}[2]{\left. {#1}\right\vert_{#2}}
% Minus for matrices
\newcommand{\mm}{\llap{$-$}}
\newcommand{\phm}{\phantom{-}}
\DeclareMathOperator{\rank}{rg}
\DeclareMathOperator{\extr}{extr}
\DeclareMathOperator{\sgn}{sgn}
\DeclareMathOperator{\Sp}{Sp}

% Using small caps to typeset theorem titles
\newtheoremstyle{ruthm}{3pt}{3pt}{\itshape}{}{\scshape}{.}{.5em}{}
\newtheoremstyle{rudfn}{3pt}{3pt}{}{}{\scshape}{.}{.5em}{}
\newtheoremstyle{rurem}{}{}{}{}{\scshape}{.}{.5em}{}

% Theorems, remarks and definition all share the same numbering
% sequence, which is also subordinate to sections
\theoremstyle{ruthm}
\newtheorem{thm}[]{Теорема}
\numberwithin{thm}{section}

\theoremstyle{rurem}
\newtheorem{rem}[thm]{Замечание}

\theoremstyle{rudfn}
\newtheorem{dfn}[thm]{Определение}

\numberwithin{equation}{section}

\usepackage[unicode,
pdftex, colorlinks, linkcolor=black, citecolor=black,
pdfauthor=Dmitry Dzhus]{hyperref}

% New environment to mark up solutions or methods
\newenvironment{steps}%
{\begin{enumerate}%
    \renewcommand{\labelenumi}{\textsf{Шаг \arabic{enumi}.}}}%
{\end{enumerate}}
\newenvironment{ssteps}%
{\begin{enumerate}%
    \renewcommand{\labelenumii}{\textsf{Шаг \arabic{enumii}.}}}%
{\end{enumerate}}

\begin{document}

\author{Дмитрий Джус}
\title{Курсовая работа по теме \\
  \Huge{«Методы оптимизации»}}
\pretitle{\begin{center}\LARGE}
\posttitle{\par\end{center}\vskip 3pc}
\date{2008}
\maketitle

\clearpage
\tableofcontents
\listoffigures

\clearpage
\section*{Предмет работы}

Настоящая курсовая работа посвящена методам оптимизации и
экстремальным задачам.

В первой части рассмотрено решение задачи условной минимизации
линейной функции с помощью симплекс-метода.

Кроме того, дано описание градиентного метода многопараметрической
оптимизации с чебышёвским функциями релаксации, предложена реализация
метода на алгоритмическом языке и представлены результаты работы на
тестовых функциях.

Последний раздел посвящён поиску экстремальных значений целевой
функции при наличии ограничений с использованием теоремы Куна—Таккера,
метода штрафных функций и метода возможных направлений Зойтендейка.

% Poorman's replacement for `\mainmatter` follows
\pagestyle{fancy}
\pagenumbering{arabic}
\clearpage
\section{Предварительные сведения}
В данном разделе изложены теоретические сведения общего характера,
которые будут использоваться в дальнейших частях работы.

\subsection{Выпуклость и вогнутость}
\label{sec:convexity}
Интуитивно ясное понятие выпуклости опирается на наглядные
геометрические представления, но вместе с тем имеет и чисто
аналитическую формулировку.

Более подробное рассмотрение выпуклых множеств и функций представлено
в \cite{polovinkin04}, \cite{kolmogorov72} и \cite{fikhtengolz03}.

\begin{dfn}
  \label{dfn:convex-set}
  Множество $D \subset \set{R}^n$ называется \neword{выпуклым}, если
  для $\forall x, y \in D, \forall \alpha \in [0;1]$ выполняется
  \begin{equation*}
    \alpha x + (1-\alpha) y \in D
  \end{equation*}
\end{dfn}
\begin{dfn}
  Если в определении \ref{dfn:convex-set} для $\forall \alpha \in
  (0;1)$ условие выполняется в смысле принадлежности ко
  \emph{внутренности} $D$, то есть
  \begin{equation*}
    \alpha x + (1-\alpha) y \in D \setminus \partial D,
  \end{equation*}
  то множество $D$ называется \neword{строго выпуклым}.
\end{dfn}
\begin{dfn}
  Множество называется \neword{вогнутым}, если оно не является
  выпукло.
\end{dfn}
Таким образом, отрезок, соединяющий две точки выпуклого множества,
целиком принадлежит этому множеству, а две точки строго выпуклого
множества соединяются отрезком, внутренние точки которого не
принадлежат границе множества (рис. \ref{fig:convex-sets}).

\begin{figure}[!thb]
  \centering
  \begin{tikzpicture}
    \coordinate (A) at (-2, 0);
    \coordinate (B) at (2, -0.5);
    \coordinate (C) at (0, -2.5);

    % Convex
    \draw[thick] (A) -- ++(1,1.2) -- ++(1.3,-1) -- ++(-0.7, -2) -- ++(-1.5, 0.2) -- cycle;
    \draw ($ (A) + (1.25,0.75) $) -- ++(0.5, -1.5);
    \node[] at ($ (A) + (0.5, -1) $) {$A$};
    
    % Strictly convex
    \draw[thick] (B) to[out=80, in=170] ($ (B) + (1.5, 1.5) $)
                     to[out=350, in=70] ($ (B) + (2.5, -0.5) $)
                     to[out=250, in=260] (B);
    \draw ($ (B) + (0.7, 1) $) -- ++(1, -1);
    \node[] at ($ (B) + (1, -0.5) $) {$B$};                     

    % Concave
    \draw[densely dashed] ($ (C) + (1,-0.1) $) -- ++(+0.85, -2);
    % Concave set is drawn twice because of clipping
    \draw[thick] (C) -- ++(2.5,0.25) --++(-2,-1.25) -- ++(3.5, -1.8) to[out=175,in=-70]
    ($ (C) - (0.1, 2) $) to[out=110, in=-100] (C);
    \clip (C) -- ++(2.5,0.25) --++(-2,-1.25) -- ++(3.5, -1.8) --
    ($ (C) - (0.1, 2) $) -- cycle;
    \draw ($ (C) + (1,-0.1) $) -- ++(+0.85, -2);
    \node[] at ($ (C) + (0.5, -1.5) $) {$C$};
  \end{tikzpicture}
  \caption[Выпуклость множеств]{Выпуклое множество $A$, строго
    выпуклое множество $B$ и вогнутое множество $C$}
  \label{fig:convex-sets}
\end{figure}


\begin{dfn}
  \label{dfn:convex-f}
  Функция $f(x), x \in \set{R}^n$ называется \neword{выпуклой}, если
  для $\forall x, y \in \set{R}^n, \forall \alpha \in [0;1]$
  выполняется неравенство
  \begin{equation*}
    f(\alpha x + (1-\alpha)y) \leq \alpha f(x) + (1-\alpha) f(y)
  \end{equation*}
\end{dfn}
\begin{dfn}
  \label{dfn:strictly-convex-f}
  Если в определении \ref{dfn:convex-f} для $\forall \alpha \in (0;1)$
  неравенство выполняется строго, функция $f(x)$ называется
  \neword{строго выпуклой}.
\end{dfn}
\begin{rem}
  \label{rem:lin-f-convex}
  Линейные функции выпуклы, но не строго выпуклы.
\end{rem}
\begin{dfn}
  Функция $f(x)$ называется \neword{вогнутой}, если $-f(x)$ является
  выпуклой функцией.
\end{dfn}

Таким образом, геометрический смысл определений \ref{dfn:convex-f} и
\ref{dfn:strictly-convex-f} ясен: функция выпукла, если выпукла
область над её графиком, и строго выпукла, если область над её
графиком строго выпукла (рис. \ref{fig:convex-function}).

\begin{figure}[!h]
  \centering
  \begin{tikzpicture}
    \begin{axis}[
      xlabel=$x$, ylabel=$f(x)$,
      xtick={-2, -1, 1, 2},
      xmin=-2.5, xmax=2.5,
      ymin=-0.9, ymax=4.9,
      axis x line=middle,
      axis y line=middle,
      axis on top,
      pattern color=gray!50]
      
      
      \draw[draw=none, pattern=north east lines] (axis cs:-1.9, 3.61) 
                       parabola bend (axis cs:0, 0)
                       (axis cs:1.9, 3.61) -- cycle;
      \addplot[thick,mark=none] plot[domain=-2:2] function{x**2};
      \addplot[mark=none] plot[domain=0.25:2] function{2*x-1};
      \addplot[mark=none] plot[domain=-1.5:-0.1] function{-x-0.25};
    \end{axis}
  \end{tikzpicture}
  \caption[Выпуклая функция]{Выпуклая функция $f(x)=x^2$ и касательные
    прямые к ней в точках $\left(-\frac{1}{2}, \frac{1}{4}\right)$ и
    $(1, 1)$}
  \label{fig:convex-function}
\end{figure}


Следущие две теоремы оказываются полезными при доказательстве
выпуклости функций и множеств.

\begin{thm}
  \label{th:convex-f}
  Если матрица Гессе квадратичной функции положительно определена, то
  эта функция выпукла.
\end{thm}

\begin{thm}
  \label{th:convex-set}
  Если $g_j(x), j=\overline{1,m}$ — выпуклые функции, то множество
  $D$, заданное условиями $g_j(x) \leq b_j$, является выпуклым.
\end{thm}


\clearpage
\section{Симплекс-метод}
\label{sec:simplex}

Симплекс-метод позволяет решать задачи линейного программирования,
заключающиеся в минимизации целевого линейного функционала при
заданных линейных ограничениях. Основной идеей симплекс-метода
является перебор вершин выпуклого многогранника в многомерном
пространстве.

\subsection{Алгоритм симплекс-метода}

Приведём описание общей схемы метода. Более подробное изложение
предложено, например, в \cite{taha05}.

Рассмотрим следующую \neword{задачу линейного программирования}:
\begin{equation}
  \label{eq:lp-problem-form-initial}
  \begin{cases}
    f(x) = \suml_{i=1}^n{\! c_i x_i} + c_{n+1} \to \min \\
    Ax = b,\, \underset{m×n}{A} = (a_{ij}),\rank{A} = m,\,b \in \set{E}^n \\
    x_i \geq 0,\, i = \overline{1, n}
  \end{cases}
\end{equation}
Как видим, она заключается в поиске таких неотрицательных значений
$x_i$, что достигается минимум линейной функции $f(x)$ при выполнении
$n$ линейных ограничений на набор $x_i$.

\begin{rem}
  \label{rem:slack}
  Рассмотрение задачи с ограничениями лишь типа равенств не
  ограничивает общности рассуждений. Действительно, ограничение типа
  неравенства со знаком ``$\leq$''
  \begin{equation*}
    \suml_{i=1}^n a_{ij} x_i \leq b_j
  \end{equation*}
  может быть сведено к ограничению типа равенства путём добавления
  дополнительной неотрицательной переменной $s_k$, называемой
  \neword{остаточной}:
  \begin{equation*}
    \suml_{i=1}^n a_{ij} x_i + s_k = b_j
  \end{equation*}

  В случае знака ``$\geq$'' ограничение типа неравенства
  \begin{equation*}
    \suml_{i=1}^n a_{ij} x_i \geq b_j
  \end{equation*}
  приводится к равенству путем вычитания \neword{избыточной
    переменной}:
  \begin{equation*}
    \suml_{i=1}^n a_{ij} x_i - s_k = b_j
  \end{equation*}

  В обоих случаях условие $s_k \geq 0$ добавляется к набору
  ограничений исходной задачи.
\end{rem}

Пусть исходная задача может быть приведена к следующему виду:
\begin{equation}
  \label{eq:lp-problem-form}
  \begin{cases}
    f(x) = -{\!\!\!\!\suml_{i={m+1}}^n{\!\!\!\!\Delta_{i} x_i}} + \hat{c}_{n+1} \to \min \\
    x_j + \!\! \suml_{i=m+1}^n {\!\!\!\!\alpha_{ji} x_i} = \beta_j,\,
    \beta_j \geq 0, j = \overline{1, m},\\
    x_i \geq 0,\, i = \overline{1, n}
  \end{cases}
\end{equation}

\begin{dfn}
  В задаче \eqref{eq:lp-problem-form} переменные $x_1, \dotsc, x_m$,
  которым соответствуют единичные столбцы в новой матрице
  коэффициентов ограничений, называются \neword{базисными}, а
  $x_{m+1}, \dotsc, x_n$ — \neword{свободными}.
\end{dfn}

\begin{dfn}
  Решение задачи \eqref{eq:lp-problem-form} при нулевых свободных
  переменных называется \neword{базисным решением}. Если при этом все
  переменные принимают неотрицательные значения, решение называется
  \neword{допустимым}, а противном случае — \neword{недопустимым}.
\end{dfn}

Итерационный алгоритм симплекс-метода для поиска оптимального решения
перебирает ограниченное число допустимых базисных решений. В ходе
работы алгоритма на каждой итерации определённая переменная
исключается из состава базисных, а другая, наоборот, вводится в их
число, что соответствует уменьшению значения целевой функции с помощью
изменения значения одной из свободных переменных.

Для формулировки алгоритма удобно на каждой итерации использовать
следующее матричное представление задачи:

\begin{equation}
  \label{eq:lp-matrix}
  \begin{bmatrix}
    \circ & \bar{x}_1 & \bar{x}_2 & \hdots & \bar{x}_m & \bar{x}_{m+1} & \hdots & \bar{x}_n & \diamond\\
    \bar{x}_1 & 1 & 0 & \hdots & 0  & \bar{\alpha}_{1,{m+1}} & \hdots & \bar{\alpha}_{1,n} & \bar{\beta}_1 \\
    \bar{x}_2 & 0 & 1 & & \vdots & \bar{\alpha}_{2,{m+1}} & \hdots & \bar{\alpha}_{2,n} & \bar{\beta}_2 \\
    \vdots & \vdots & &\vspace{2pt}\ddots  & 0 & \vdots & \ddots & \vdots & \vdots \\
    \bar{x}_m & 0 & \hdots & 0 & 1 & \bar{\alpha}_{m,{m+1}} & \hdots & \bar{\alpha}_{m,n} & \bar{\beta}_m \\
    f & 0 & \hdots & 0 & 0 & \bar{\Delta}_{m+1} & \hdots & \bar{\Delta}_n & \hat{\bar{c}}_{n+1}
  \end{bmatrix}
\end{equation}
\begin{rem}
  Поскольку наборы базисных и свободных переменных меняются с каждой
  итерацией, в данной формулировке с помощью
  $\bar{x}_1,\dotsc,\bar{x}_m$ и $\bar{x}_{m+1}, \dotsc, \bar{x}_n$
  обозначены базисные и свободные переменные \emph{на текущей
    итерации}. Аналогичные обозначения используются и для элементов
  матрицы \eqref{eq:lp-matrix}. На практике подобное
  переупорядочивание переменных обычно не используется.
\end{rem}

Приведём последовательность действий, выполняемых для решения задачи
симплекс-методом.
\begin{enumerate}
  \renewcommand{\labelenumi}{\textbf{Шаг \arabic{enumi}.}}
\item Если все коэффициенты $\bar{\Delta}_{m+1}, \dotsc,
  \bar{\Delta}_n$ неположительны, то найдено оптимальное решение
  $\bar{x}_1 = \bar{\beta}_1, \dotsc, \bar{x}_m = \bar{\beta}_m,
  \bar{x}_{m+1} = \bar{x}_n = 0$. Если найдётся $\bar{\Delta}_s > 0$
  такое, что в соответствующем столбце матрицы \eqref{eq:lp-matrix}
  все коэффициенты $\bar{\alpha}_{js} \leq 0$, решений нет.
\item В качестве вводимой в состав базиса свободной переменной
  $\bar{x}_s$ выбирается та, которой отвечает наибольший положительный
  коэффициент $\bar{\Delta}_s$. Соответствующий столбец называется
  ведущим.
\item Выбор исключаемой из базиса переменной $\bar{x}_r$ производится
  таким образом, что достигается минимум симплексного отношения
  $\Theta = \frac{\bar{\beta}_r}{\bar{\alpha}_{rs}}$ для
  $\bar{\alpha}_{rs} > 0$. Соответствующая строка называется ведущей.
\item Осуществляется переход к новому базису — ведущая строка делится
  на $\bar{\alpha}_{rs}$ и ведущий столбец путём элементарных
  преобразований приводится единичному виду.
\item Процесс повторяется с первого шага для нового базиса
\end{enumerate}

\begin{rem}
  После того, как переменная была исключена из числа базисных,
  дальнейшее её рассмотрение в ходе работы алгоритма не имеет
  практического значения. В представленных далее примерах столбцы
  матрицы \eqref{eq:lp-matrix}, соответствующие исключённым ранее
  переменным, будут опускаться.
\end{rem}

Алгоритм симплекс-метода гарантирует допустимость базисного решения,
получаемого на каждой итерации. При этом проблему может составлять
построение \emph{начального} допустимого базисного решения. Структура
целевой функции и ограничений могут быть таковы, что сразу привести
задачу \eqref{eq:lp-problem-form-initial} к виду
\eqref{eq:lp-problem-form} не удастся. Для этого пользуются различными
методами для поиска начального решения, один из которых — метод
искусственных переменных, изложен в следующем разделе.

\begin{rem}
  \label{rem:slack-solution}
  Задачи линейного программирования, в которых все ограничения имеют
  вид неравенств с ``$\leq$'' с неотрицательной правой частью,
  приводятся к виду \eqref{eq:lp-problem-form} за счёт добавления
  остаточных переменных (см. замечание \ref{rem:slack}), поэтому для
  них начальное допустимое решение получается сразу.
\end{rem}

\subsubsection{Метод искусственных переменных}

В данном методе сначала рассматривается вспомогательная задача,
решение которой затем используется в качестве начального для исходной
задачи. Алгоритм метода имеет следующую схему:

\begin{enumerate}
  \renewcommand{\labelenumi}{\textbf{Шаг \arabic{enumi}.}}
\item Пусть все ограничения исходной задачи
  \eqref{eq:lp-problem-form-initial} приведены к виду равенств. Для
  каждого ограничения, в котором не содержится остаточной переменной
  (см. замечание \ref{rem:slack}), вводится искусственная остаточная
  переменная.
\item Решается задача линейного программирования \emph{минимизации}
  суммы $\tilde{f}$ введённых искусственных переменных при полученных
  ограничениях. Если минимальное значение новой целевой функции
  $\tilde{f}$ больше нуля, задача решения не имеет. В противном случае
  полученное оптимальное решение используется в качестве начального
  допустимого базисного решения в исходной задаче.
\end{enumerate}

\subsubsection{Графическая интерпретация}

Рассмотрим следующую простую задачу линейного программирования:

\begin{equation}
  \label{eq:lp-simple}
  \begin{cases}
    f(x) = -x_1-2x_2 \to \min \\
    -3x+4y \leq 8 \\
    \phm 7x+6y \leq 35 \\
    \phm x_1, x_2 \geq 0
  \end{cases}
\end{equation}

Проиллюстрируем на ней алгоритм симплекс-метода.

Согласно замечанию \ref{rem:slack} преобразуем ограничения к
равенствам, добавив остаточные переменные, и приведём задачу к виду
\eqref{eq:lp-problem-form}:
\begin{equation*}
  \begin{cases}
    f(x) = -x_1-2x_2 \to \min \\
    -3x+4y+s_1 = 8 \\
    \phm 7x+6y+s_2 = 35 \\
    \phm x_1, x_2, s_1, s_2 \geq 0
  \end{cases}
\end{equation*}
Как и говорилось в замечании \ref{rem:slack-solution}, в данном случае
использование метода искусственных переменных не нужно, так как
начальное допустимое решение находится сразу.

Перейдём к матричному представление задачи.

\begin{equation*}
  \begin{bmatrix}
    \circ & s_1 & s_2 & x_1 & x_2 & \diamond\\
    s_1 & 1 & 0 & -3 & 4 & \mathbf{8} \\
    s_2 & 0 & 1 &  7 & 6 & 35 \\
    f & 0 & 0 & 1 & \mathbf{2} & 0
  \end{bmatrix}
\end{equation*}
Максимальный положительный элемент в последней строке соответствует
свободной переменной $x_2$, поэтому её выберем в качестве вводимой.
Минимум соотношения $\Theta = \frac{\bar{\beta}_r}{\bar{\alpha}_{rs}}$
достигается на первой строке (в ней $\Theta = \frac{8}{4} = 2$),
поэтому первая базисная переменная $s_1$ выбирается в качестве
исключаемой. Теперь путём элементарных преобразований приведём ведущий
столбец к единичному виду:
\begin{equation*}
  \begin{gmatrix}[b]
    \circ & s_1 & s_2 & x_1 & x_2 & \diamond\\
    s_1 & 1 & 0 & \mm{3} & \mathbf{4} & 8 \\
    s_2 & 0 & 1 &  7 & 6 & 35 \\
    f & 0 & 0 & 1 & 2 & 0
    \rowops
    \add[-\frac{3}{2}]{1}{2}
    \add[-\frac{1}{2}]{1}{3}
    \mult{1}{\mul \frac{1}{4}}
  \end{gmatrix}
\end{equation*}
Таким образом переменная $x_2$ вводится в базис, а $s_1$ — исключается
из него. На следующей итерации матрица задачи имеет вид:
\begin{equation*}
  \begin{bmatrix}
    \circ & s_2 & x_1 & x_2 & \diamond \vspace{1pt}\\
    x_2 & 0 & \mm{\frac{3}{4}} & 1 & 2 \vspace{3pt}\\
    s_2 & 1 & \frac{23}{2} & 0 & \mathbf{23} \vspace{3pt}\\
    f & 0 & \mathbf{\frac{5}{2}} & 0 & \mm{4}
  \end{bmatrix}
\end{equation*}
На этот раз ведущим является третий столбец, соответствующий свободной
переменной $x_1$. Строка базисной переменной $s_1$ является ведущей.
Введём выбранную переменную $x_1$ в состав базисных путём элементарных
преобразований:
\begin{equation*}
  \begin{gmatrix}[b]
    \circ & s_2 & x_1 & x_2 & \diamond \vspace{1pt}\\
    x_2 & 0 & \mm{\frac{3}{4}} & 1 & 2 \vspace{3pt}\\
    s_2 & 1 & \mathbf{\frac{23}{2}} & 0 & 23 \vspace{3pt}\\
    f & 0 & \frac{5}{2} & 0 & \mm{4}
    \rowops
    \add[\frac{3}{46}]{2}{1}
    \add[-\frac{5}{23}]{2}{3}
    \mult{2}{\mul \frac{2}{23}}
  \end{gmatrix}
\end{equation*}

В результате элементы рассматриваемой матрицы приобретут следующие
значения:
\begin{equation*}
  \begin{bmatrix}
    \circ & x_1 & x_2 & \diamond \vspace{1pt}\\
    x_2 & 0 & 1 & \frac{7}{2} \vspace{3pt}\\
    x_1 & 1 & 0 & 2 \vspace{3pt}\\
    f & 0 & 0 & \mm{9}
  \end{bmatrix}
\end{equation*}

Отрицательность всех коэффициентов в последней строке свидетельствует
о том, что найдено оптимальное допустимое решение. Соответствующие
значения базисных переменных находятся в правом столбце матрицы: $x_1
= 2, x_2 = \frac{7}{2}$, свободные переменные приравниваются к нулю.

\begin{figure}[!thb]
  \centering
  \begin{tikzpicture}
    \begin{axis}[      
      xlabel=$x_1$,
      ylabel=$x_2$,
      xtick={1, 5}, ytick={1, 2, 5},
      xmin=-0.5, ymin=-0.5,
      xmax=7, ymax=7,
      axis x line=center,
      axis y line=center,
      pattern color=gray]

      \addplot[draw=none,mark=none,pattern=north east lines] coordinates{
        (0,0) (0,2) (2,3.5) (5,0)} \closedcycle;
      
      \addplot[thick,mark=none] coordinates{(4, 5) (-1, 1.25)};
      \addplot[thick,mark=none] coordinates{(5.5, -0.5833) (1.5, 4.0833)};
      \draw node[circle, fill=black, scale=0.5, 
                 pin={above:$(2, 3.5)$}] at (axis cs:2,3.5) {};

      \addplot[ultra thick, arrows=-triangle 45, blue]  coordinates{(0, 0) (0, 2)};
      \addplot[ultra thick, arrows=-triangle 45, blue] coordinates{(0, 2) (2,3.5)};
      
    \end{axis}
  \end{tikzpicture}
  \caption[Симплек-метод]{Ход решения задачи \eqref{eq:lp-simple}
    симплекс-методом}
\end{figure}
\clearpage
\subsection{Практический пример}

К решению предлагается следующая задача:
\begin{equation}
  \label{eq:lp-initial}
  \begin{cases}
    f(x) = -3x_1+2x_2-2x_3+2x_4-x_5 \to \min \\
    -x_1+x_2-x_3=1 \\
    -x_2+x_3+x_4=1 \\
    \phm x_2+x_3+x_5=2 \\
    \phm x_i \geq 0,\, i = \overline{1, 5}
  \end{cases}
\end{equation}

Воспользуемся методом искусственных переменных для построения
начального допустимого базисного решения. Добавим в первое ограничение
переменную $x_6$ так, что оно примет вид
\begin{equation*}
  -x_1+x_2-x_3+x_6=1 \\
\end{equation*}

Заменим минимизируемую функцию на $\tilde{f}(x) = x_6$ и выразим её
через свободные переменные $x_1, x_2, x_3$, получив
\mbox{$\tilde{f}(x) = x_1-x_2+x_3+1$}. Перейдём к рассмотрению
вспомогательной задачи
\begin{equation}
  \begin{cases}
    \tilde{f}(x) = x_1-x_2+x_3+1 \to \min \\
    -x_1+x_2-x_3+x_6=1 \\
    -x_2+x_3+x_4=1 \\
    \phm x_2+x_3+x_5=2 \\
    \phm x_i \geq 0,\, i = \overline{1, 6}
  \end{cases}
\end{equation}

Решим её симплекс-методом. Составим расширенную матрицу из
коэффициентов ограничений и целевой функции:
\begin{equation}
  \begin{gmatrix}[b]
    \dagger &  x_1 &  x_2 &  x_3 & x_4 & x_5 &
    x_6 & \diamond\\
    x_6 & \mm{1} &  1 & \mm{1} & 0 & 0 & 1 & \mathbf{1} \\
    x_4 &  0 & \mm{1} &  1 & 1 & 0 & 0 & 1 \\
    x_5 &  0 &  1 &  1 & 0 & 1 & 0 & 2 \\
    \tilde{f}    & \mm{1} &  \mathbf{1} & \mm{1} & 0 & 0 & 0 & 1
  \end{gmatrix}
\end{equation}

Базисными переменными являются $x_4, x_5, x_6$. Для дальнейшего
решения необходимо исключить искусственную переменную $x_6$ из состава
базисных. 

Выберем в качестве ведущего столбец с максимальным элементом в
последней строке — это столбец переменной $x_2$, поскольку $\max\{1,
-1, 0\} = 1$. С помощью симплексного отношения выберем первую строку в
качестве ведущей.

Учитывая выбранный ведущий элемент, переведём переменную $x_2$ в
состав базисных взамен $x_6$:
\begin{equation*}
  \begin{gmatrix}[b]
    \dagger & x_1 & x_2 & x_3 & x_4 & x_5 & x_6 & \diamond\\
    x_6 & \mm{1} &  \mathbf{1} &\mm{1}& 0 & 0 & 1 & 1 \\
    x_4 &  0 &\mm{1}&  1 & 1 & 0 & 0 & 1 \\
    x_5 &  0 &  1 &  1 & 0 & 1 & 0 & 2 \\
    \tilde{f} &  1&  1 &\mm{1}& 0 & 0 & 0 & 1
    \rowops
    \add{1}{2}
    \add[-1]{1}{3}
    \add[-1]{1}{4}
  \end{gmatrix}
\end{equation*}

Получим матрицу с базисными переменными $x_2, x_4, x_5$:
\begin{equation}
  \label{eq:lp-artif}
  \begin{bmatrix}
    \dagger & x_1 & x_2 & x_3 & x_4 & x_5 & \diamond\\
    x_2 &\mm{1}&  1 &\mm{1}& 0 & 0 & 1\\
    x_4 &\mm{1}&  0 &  0 & 1 & 0 & 2\\
    x_5 &  1 &  0 &  2 & 0 & 1 & 1\\
    \tilde{f} & 0 &  0 & 0 & 0 & 0 & 0
  \end{bmatrix}
\end{equation}

При этом значение функции $\tilde{f}(x)$ равно нулю, что свидетельствует
о том, что найденное решение вспомогательной задачи является
допустимым для исходной.

Отбросим искусственную переменную $x_6$ и вернёмся к начальной задаче
\eqref{eq:lp-initial}. Исключим из исходной целевой функции $f(x)$
базисные переменные $x_2, x_4$ и $x_5$, используя ограничения из
матрицы \eqref{eq:lp-artif}:
\begin{equation*}
  \begin{cases}
    -x_1+x_2-x_3=1\\
    -x_1+x_4=2\\
    \phm x_1+2x_3+x_5 = 1
  \end{cases}
\end{equation*}

Тогда $f(x)$ принимает вид
\begin{equation*}
  f(x) = 2x_1+2x_3+5
\end{equation*}

Используем эту целевую функцию в матрице ограничений \eqref{eq:lp-artif}:
\begin{equation*}
  \begin{bmatrix}
    \dagger & x_1 & x_2 & x_3 & x_4 & x_5 & \diamond\\
    x_2 &\mm{1}& 1 &\mm{1}& 0 & 0 & 1\\
    x_4 &\mm{1}& 0 &  0 & 1 & 0 & 2\\
    x_5 &  1 & 0 &  2 & 0 & 1 & 1\\
    \tilde{f} & \mm{2} & 0 & \mm{2} & 0 & 0 & 5
  \end{bmatrix}
\end{equation*}

В нижнем ряду под столбцами переменных все коэффициенты отрицательны,
что свидетельствует о том, что найдено оптимальное решение:
\begin{equation}
  \begin{cases}
    x_1 = 0\\
    x_2 = 1\\
    x_3 = 0\\
    x_4 = 2\\
    x_5 = 1
  \end{cases}
\end{equation}


\clearpage
\section{Многопараметрическая оптимизация\\
  с чебышёвскими функциями релаксации}
\label{sec:relch}

В данном разделе представлена общая схема градиентных методов,
рассмотрено понятие функции релаксации и описан метод оптимизации с
чебышёвскими функциями релаксации, предложенный в
\cite{chernorutsky04}.

\subsection{Теоретические сведения}

\subsubsection{Общая схема градиентных методов}

Рассмотрим задачу безусловной минимизации:
\begin{equation}
  \label{eq:optim-problem-form}
  f(x) \to \min,\quad x \in \set{R}^n,\, J \in C^2(\set{R}^n)
\end{equation}

\begin{dfn}
  \neword{Градиентными} называются итерационные методы оптимизации со
  следующей рабочей формулой, которая определяет способ перехода к
  новому приближению $x^{k+1}$ на очередном шаге итерации:
  \begin{equation}
    \label{eq:grad-methods}
    x^{k+1} = x^k - H_k\left(G_k, h_k\right) g_k
  \end{equation}
  здесь $H_k$ — некоторая функция от матрицы Гессе $G_k = G(x^k) =
  f''(x^k)$ и параметра $h_k$, а $g_k = g(x^k) = f'(x^k)$ — градиент
  функции в точке $x_k$.
\end{dfn}

При постоянных значениях параметров на различных шагах итерации
соответствующие индексы $k$ в формуле \eqref{eq:grad-methods}
опускаются.

Предполагается, что в некоторой $\epsilon_k$-окрестности $\{x \in
\set{R}^n | \norm{x-x^k} < \epsilon_k\}$ точки $x^k$ функция $f(x)$
приближается гиперболоидом:
\begin{equation}
  \label{eq:sqr-approx}
  f(x) \approx \frac{1}{2}\scalmult{G_k x, x} - \scalmult{a_k,x} + b_k \approx \frac{1}{2}\scalmult{G_k x, x}
\end{equation}

Ставится задача построения таких матричных функций $H_k$, при которых
выполняется \neword{условие релаксации} процесса
\begin{equation}
  \label{eq:relax-cond}
  f(x^{k+1}) < f(x^k)
\end{equation}
При этом требуется, чтобы величина нормы $\norm{x^{k+1}-x^{k}}$ была
ограничена сверху лишь параметром $\epsilon_k$, который характеризует
область справедливости локальной квадратичной модели
\eqref{eq:sqr-approx}.

\subsubsection{Функция релаксации}

\begin{dfn}
  \neword{Функцией релаксации} называется скалярная функция
  \begin{equation}
    \label{eq:relax-fun}
    R_h(\lambda) = 1 - H(\lambda, h)\lambda,\quad \lambda,h \in \set{R}
  \end{equation}
  где $H(\lambda, h)$ — скалярный аналог матричной функции $H(G, h)$
  из формулы \eqref{eq:grad-methods}.
\end{dfn}
В дальнейшем индекс $h$ у функции релаксации $R_h(\lambda)$ иногда
будем опускать.

\begin{dfn}
  \neword{Множителями релаксации} для точки $x^k$ называются значения
  функции релаксации на спектре матрицы Гессе:
  \begin{equation}
    \label{eq:relax-fac}
    R_h(\lambda_i),\, \lambda_i \in \Sp{G_k}
  \end{equation}
\end{dfn}

Благодаря следующей теореме, функция релаксации используется для
анализа различных градиентных методов.

\begin{thm}
  \label{thm:relax-thm}
  При любых $x^k$ для выполнения условия релаксации
  \eqref{eq:relax-cond} необходимо и достаточно, чтобы
  \begin{equation}
    \label{eq:relax-thm}
    \begin{aligned}
      & \abs{R(\lambda_i)} \geq 1 & \lambda_i& < 0 \\
      & \abs{R(\lambda_i)} \leq 1 & \lambda_i& > 0\\
      &&i& = \overline{1, n}
    \end{aligned}
  \end{equation}
\end{thm}

\begin{figure}[tb]
  \centering
  \label{fig:relax-thm}
  \begin{tikzpicture}
    \begin{axis}[
      xlabel=$\lambda$,xtick=\empty,
      ylabel=$R(\lambda)$,ytick={-1,1},
      y tick label style={anchor=north east},
      y tick style={draw=none},
      xmin=-3.2,xmax=3.2,ymin=-2.5,ymax=2.5,
      axis x line=middle,
      axis y line=middle,
      axis on top,
      pattern color=gray!50,
      ]

      % Forbidden areas
      \addplot[draw=none,mark=none,pattern=north west lines] coordinates{
        (-3,1) (0,1) (0,-1) (-3,-1)} \closedcycle;
      \addplot[draw=none,mark=none,pattern=north west lines] coordinates{
        (0,1) (3,1) (3,2) (0,2)} \closedcycle;
      \addplot[draw=none,mark=none,pattern=north west lines] coordinates{
        (0,-1) (3,-1) (3,-2) (0,-2)} \closedcycle;

      % Function
      \addplot[mark=none,smooth,thick] coordinates{
        (-2.5, 2)
        (-2.2, 1.1)
        (-1.8, 1.6)
        (-1.1, 1.3)
        (-0.5, 1.4)
        (0, 1)
        (0.5, -0.8)
        (1.7, 0.1)
        (2.5, 0.2)};
      
      % Dashed lines at forbidden area boundaries
      \addplot[dashed,mark=none] coordinates{
        (-3,1)
        (3,1)};
      \addplot[dashed,mark=none] coordinates{
        (-3,-1)
        (3,-1)};
    \end{axis}
  \end{tikzpicture}
  \caption[Подходящая функция релаксации]{Функция релаксации,
    удовлетворяющая условиям теоремы \ref{thm:relax-thm}}
  \label{fig:relax-thm}
\end{figure}


Скорость релаксации может быть оценена с использованием следующего
соотношения:
\begin{equation}
  \label{eq:relax-speed}
  2\abs{f(x^{k+1})-f(x^k)}=\sum_{\lambda_i^+>0} \left\{\xi_{i,k}^2
    \lambda_i^+ [1-R^2(\lambda_i^+)]\right\} + \sum_{\lambda_i^-<0} \left\{\xi_{i,k}^2
    \abs{\lambda_i^-} [R^2(\lambda_i^-)-1]\right\}
\end{equation}
здесь $\lambda_i^+$ и $\lambda_i^-$ — положительные и отрицательные
собственные значения матрицы $G_k$. Коэффициенты $\xi_{i,k}$
происходят из разложения $x^k=\sum_i{\xi_{i,k}u^i}$ по собственным
векторам $u^i$ матрицы Гессе.

Таким образом, эффективными оказываются методы, функция релаксации
которых в положительной области значений $\lambda$ как можно
\emph{меньше} уклоняется от нуля, а при отрицательных $\lambda$ —
становится как можно большей по модулю.

\paragraph{Пример: метод ПГС}

В качестве примера рассмотрим метод \neword{простого градиентного
  спуска} (ПГС), рабочая формула которого имеет вид:
\begin{equation*}
  x^{k+1}=x^k-hg_k,\, h\in\set{R}
\end{equation*}
Рассмотрим функцию релаксации метода ПГС:
\begin{equation}
  \label{eq:gd-relax}
  R(\lambda) = 1 - \lambda h
\end{equation}

\begin{figure}[!h]
  \centering
  \label{fig:relax-thm}
  \begin{tikzpicture}
    \begin{axis}[
      xlabel=$\lambda$, ylabel=$R(\lambda)$,
      xtick={2}, xticklabels={$\frac{1}{h}$},
      ytick=\empty,
      x tick style={black},
      x tick label style={anchor=south west},
      extra x ticks={4},
      extra x tick labels={$M^*$},
      extra x tick style={grid=major},
      xmin=-3.2, xmax=6.5, ymin=-2.5, ymax=3.5,
      axis x line=middle,
      axis y line=middle,
      axis on top,
      mark=none,
      pattern color=gray!50,
      ]

      % Forbidden areas
      \addplot[draw=none,pattern=north west lines] coordinates{
        (-3,1) (0,1) (0,-1) (-3,-1)} \closedcycle;
      \addplot[draw=none,pattern=north west lines] coordinates{
        (0,1) (6,1) (6,2) (0,2)} \closedcycle;
      \addplot[draw=none,pattern=north west lines] coordinates{
        (0,-1) (6,-1) (6,-2) (0,-2)} \closedcycle;

      % Function
      \addplot[mark=none] coordinates{
        (6, -2)
        (-2, 2)};
      
      % Dashed lines at forbidden area boundaries
      \addplot[dashed] coordinates{
        (-3,1)
        (6,1)};
      \addplot[dashed] coordinates{
        (-3,-1)
        (6,-1)};
    \end{axis}
  \end{tikzpicture}
  \caption{Функция релаксации $R(\lambda)=1-\lambda h$ метода простого
    градиентного спуска}
  \label{fig:relax-thm}
\end{figure}


Из её графика на рисунке \ref{fig:gd-relax} видно, что метод применим
лишь в том случае, когда собственные значения матрицы Гессе
оптимизируемой функции не превосходят некоторого критического значения
$M*=\frac{2}{h}$. Кроме того, в области значений $\lambda$, близких к
нулю или $M*$, согласно \eqref{eq:relax-speed} скорость релаксации
сильно снижается.

\subsection{Описание метода}

\subsubsection{Использование полиномов Чебышёва\\
  в функции релаксации}

Рассмотрим смещённые полиномы Чебышёва второго рода, которые
определяются согласно рекуррентному соотношению:
\begin{align}
  \label{eq:chebyshev}
  P_0(\lambda) &= 0 \\
  P_1(\lambda) &= 1 \\
  P_k(\lambda) &= 2(1-2\lambda)P_{k-1}(\lambda) - P_{k-2}(\lambda),\, k
  \geq 2
\end{align}

Функциональная последовательность таких полиномов обладает важным
свойством, а именно: при $s\to\infty$, $\frac{P_s(\lambda)}{s}$
равномерно стремится к нулю на $(0;1)$.

Предположим, что собственные числа матрицы Гессе целевой функции
$f(x)$ задачи \eqref{eq:optim-problem-form} в положительной области
спектра не превосходят 1 (для этого достаточно в рассмотрении считать
градиент и матрицу Гессе нормированными). Тогда использование
следующей функции в качестве релаксационной:
\begin{equation}
  \label{eq:cheb-relax}
  R_s(\lambda) = \frac{P_s(\lambda)}{s}
\end{equation}
позволяет обеспечить, согласно \eqref{eq:relax-speed}, сколь угодно
быструю релаксацию. На иллюстрации \ref{fig:cheb-relax} приведены
графики функции $R_s(\lambda)$ вблизи $[0;1]$ для нескольких значений
$s$.

\begin{figure}[!h]
  \centering
  \begin{tikzpicture}
    \begin{axis}[x=7cm, y=2cm,
      xlabel=$\lambda$, ylabel=$R_s(\lambda)$,
      xtick=\empty, ytick={-1, 1},
      y tick label style={anchor=north east},
      y tick style={draw=none},
      extra x ticks={1},
      extra x tick style={grid=major},
      xmin=-0.2, xmax=1.4, ymin=-2, ymax=2,
      axis x line=middle,
      axis y line=middle,
      axis on top,
      pattern color=gray!50
      ]

      \addplot[mark=o] plot[domain=-0.2:1.1] function{(16*x**2-16*x+3)/3};
      \addplot[mark=square] plot[domain=-0.2:1.1] function{(-64*x**3+96*x**2-40*x+4)/4};
      \addplot[mark=triangle] plot[domain=-0.2:1.1] function{(256*x**4-512*x**3+336*x**2-80*x+5)/5};
      \legend{$L=3$, $L=4$, $L=5$}


      % Forbidden areas
      \addplot[draw=none,pattern=north west lines] coordinates{
        (-0.2,1) (0,1) (0,-1) (-0.2,-1)} \closedcycle;
      \addplot[draw=none,pattern=north west lines] coordinates{
        (0,1) (1.2,1) (1.2,1.4) (0,1.4)} \closedcycle;
      \addplot[draw=none,pattern=north west lines] coordinates{
        (0,-1) (1.2,-1) (1.2,-1.4) (0,-1.4)} \closedcycle;
      
      % Dashed lines at forbidden area boundaries
      \addplot[dashed] coordinates{
        (-0.2,1)
        (1.2,1)};
      \addplot[dashed] coordinates{
        (-0.2,-1)
        (1.2,-1)};
    \end{axis}
  \end{tikzpicture}
  \caption{Функции релаксации на основе полиномов Чебышёва}
  \label{fig:cheb-relax}
\end{figure}


Уже при $s=8$ значение $R_s(\lambda)$ не превосходит по модулю $0.23$
на отрезке $[0.025; 0.975]$, обеспечивая хорошее подавление слагаемых
\eqref{eq:relax-speed}, соответствующих большому диапазону
положительных собственных чисел.

При этом стремление $R_s(\lambda)$ к $+\infty$ в
отрицательной части спектра также соответствует требованиям к хорошей
функции релаксации.

\subsubsection{Реализация метода}
Согласно \eqref{eq:relax-fun}, выбранной функции релаксации
соответствует зависимость
\begin{equation*}
  H(\lambda) = \frac{1-\frac{P_s(\lambda)}{s}}{\lambda}
\end{equation*}
откуда с учётом \eqref{eq:chebyshev} получим следующие рекуррентные
соотношения уже для функции $H(\lambda)$:
\begin{equation}
  \label{eq:cheb-scalarfun}
  \begin{aligned}
    H_1 &= 0 \\
    H_2 &= 2 \\
    H_{s+1} \mul (s+1) &= 2s\mul(1-2\lambda)\mul H_s-(s-1)\mul
    H_{s-1}+4s
  \end{aligned}
\end{equation}
где $s$ — параметр метода, равный степени используемых полиномов
Чебышёва. Соображения по выбору $s$ приведены ниже.

Подстановка полученного соотношения \eqref{eq:cheb-scalarfun}
в \eqref{eq:grad-methods} даёт следующую рабочую форумулу
\begin{multline}
  x^{k+1} = x^k - H_{s+1}g_k =\\=
  x^k-\frac{2s}{s+1}(E-2G_k)H_sg_k+\frac{s-1}{s+1}H_{s-1}g_k-\frac{4s}{s+1}g_k
\end{multline}

Таким образом, вектор смещения $d_{s+1} = x^{k+1} - x^k$ при выбранном
значении параметра $s$ на каждом шаге $k$ вычисляется по следующей
рекуррентной формуле:
\begin{equation}
  \label{eq:cheb-workhorse}
  \begin{aligned}
    d_1 &= 0\\
    d_2 &= -2g_k \\
    d_{s+1} &=
    \frac{2s}{s+1}(E-2G_k)d_{s}-\frac{s-1}{s+1}d_{s-1}-\frac{4s}{s+1}g_k
  \end{aligned}
\end{equation}


\clearpage
\subsection{Примеры работы}

В данном разделе приведены результаты работы градиентного метода
оптимизации с чебышёвскими функциями релаксации с рядом типовых
тестов.

\subsubsection{Функция Розенброка}

Одним из классических тестов для различных алгоритмов оптимизации
является тест Розенброка, заключающийся в минимизации следующей
функции:
\begin{equation}
  \label{eq:rosenbrock}
  f(x, y) = 100(y - x²)² + (1 - x)²
\end{equation}
Функция Розенброка имеет глубокую впадину вдоль кривой $y=x^2$, что
обуславливает плохую сходимость простых методов. Минимум этой функции
находится в точке $A=(1,1)$.

Матрица Гессе этой функции имеет вид:
\begin{equation*}
  \begin{pmatrix}
    1200x^2-400y+2 & -400x \\
    -400x & \phm200\ph{x}
  \end{pmatrix}
\end{equation*}
и в точке минимума функции $(1, 1)$ равна
\begin{equation*}
  \begin{pmatrix}
    \phm 802 & -400 \\ -400 & \phm 200
  \end{pmatrix}
\end{equation*}
Собственные значения матрицы Гессе в этой точке равны ${\approx}0.3994$ и
${\approx}1001.6006$, так что овражность целевой функции равна
\begin{equation*}
  \eta = \frac{1001.6006}{0.3994} \approx 2507.7631
\end{equation*}

\pgfplotsset{every axis/.append style={xlabel=$x$, ylabel=$y$,grid}}
\begin{figure}[!thb]
  \centering
  \begin{tikzpicture}
    \begin{axis}[x=2.7cm, y=2.2cm]
      \input{rosenbrock-contours.tkz.tex}
      \input{rosenbrock_relch_-1.2,1_40_12-trace.tkz.tex}
      \node[circle,fill=black,scale=0.25,label=right:$A$] at (axis cs:1,1) {};
    \end{axis}
  \end{tikzpicture}
  \caption[Функция Розенброка]{Работа алгоритма с функцией Розенброка
    \eqref{eq:rosenbrock}}
\end{figure}
\clearpage

\subsubsection{Экспоненциальная функция}

Теперь рассмотрим функцию
\begin{equation}
  \label{eq:exptest}
  f(x, y) = \sum\limits_{a=\overline{0.1, 1.0}}\left [
    e^{-xa}-e^{-ya}-(e^{-a}-e^{-10a})\right ]^2
\end{equation}
Здесь суммирование происходит по значениям $a = 0.1, 0.2, \dotsc, 1$.

\begin{figure}[!thb]
  \centering
  \begin{tikzpicture}
    \begin{axis}[x=1cm,y=1cm]
      \input{exptest-contours.tkz.tex}
      \input{exptest_relch_-0.5,1.5_20-trace.tkz.tex}
      \input{exptest_relch_5,5_20-trace.tkz.tex}
      \input{exptest_relch_3,-1_25_12-trace.tkz.tex}
      \node[circle,fill=black,scale=0.5,label=right:$A$] at (axis cs:1,10) {};
    \end{axis}
  \end{tikzpicture}
  \caption[Экспоненциальная функция]{Линии уровня и трассировка процесса минимизации функции
    \eqref{eq:exptest}}
\end{figure}

Алгоритм стабильно определяет минимум в точке $(1, 10)$.

\subsubsection{Функция Химмельблау}

В следующем примере рассмотрим функцию Химмельблау, которая задаётся
следующим образом:
\begin{equation}
  \label{eq:himmelblau}
  f(x, y) = (x² + y - 11)² + (x + y² - 7)²
\end{equation}
Известно, что локальные минимумы функции \eqref{eq:himmelblau}
расположены в точках $A=(-3.78, -3.28)$, $B=(-2.80, 3.13)$, $C=(3.58,
-1.85)$, $D=(3.00, 2.00)$, и в каждой из них достигается значение $0$.
Этим обусловлена следующая особенность работы алгоритма — результат
значительно зависит от выбора начального приближения, что
продемонстрировано на рисунке \ref{fig:himmelblau}.

\begin{figure}[!thb]
  \centering
  \begin{tikzpicture}
    \begin{axis}[x=.8cm,y=.8cm]
      \input{himmelblau-contours.tkz.tex}
      \input{himmelblau_relch_0.1,0.1_10-trace.tkz.tex}
      \input{himmelblau_relch_0.5,-0.5_10-trace.tkz.tex}
      \input{himmelblau_relch_-0.5,0.5_10-trace.tkz.tex}
      \input{himmelblau_relch_-0.5,-0.5_10-trace.tkz.tex}

      \node[circle,fill=black,scale=0.5,label=above left:$A$] at (axis cs:-3.78,-3.28) {};
      \node[circle,fill=black,scale=0.5,label=below left:$B$] at (axis cs:-2.8,3.13) {};
      \node[circle,fill=black,scale=0.5,label=below right:$C$] at (axis cs:3.58,-1.85) {};
      \node[circle,fill=black,scale=0.5,label=above right:$D$] at (axis cs:3,2) {};
    \end{axis}      
  \end{tikzpicture}
  \caption[Функция Химмельблау]{Результаты работы алгоритма с функцией
    Химмельблау \eqref{eq:himmelblau} в зависимости от выбора
    начальной точки}
  \label{fig:himmelblau}
\end{figure}


\clearpage
\section{Условная оптимизация}

Задачами условной оптимизации называются задачи на поиск экстремума
целевой функции $f(x)$ в области $D$, заданной ограничениями типа
неравенств:
\begin{equation}
  \label{eq:cond-optim-problem-form}
  \begin{cases}
    f(x) \to \extr \\
    g_j(x) \leq 0,\, j=\overline{1,m} \\
    x \in \set{R}^n
  \end{cases}
\end{equation}

\begin{rem}
  \label{rem:eq-noneq}
  Ограничение типа равенства
  \begin{equation*}
    g(x) = 0
  \end{equation*}
  эквивалентно паре ограничений типа неравенств
  \begin{align*}
    g(x) \leq 0 \\
    g(x) \geq 0
  \end{align*}
\end{rem}

Далее рассматриваются различные методы решения задач типа
\eqref{eq:cond-optim-problem-form}. Теоретическое описание каждого
метода иллюстрируется практическим решением модельной задачи
\eqref{eq:cond-optim-problem-raw}.

\subsection{Теорема Куна—Таккера}
\label{sec:kuhn-tucker}

На основе теоремы Куна—Таккера построен \emph{аналитический} метод
решения задачи \eqref{eq:cond-optim-problem-form}, позволяющий найти
её решения точно. 

В данной главе формулируется ряд определений и теорем, которые
используются в алгоритме метода, после чего рассматривается решение
модельной задачи с его помощью.

Более подробно тема освещена в \cite{alekseev05} и \cite{taha05}.

\subsubsection{Теоретические сведения}

\begin{dfn} \neword{Функцией Лагранжа} задачи
  \eqref{eq:cond-optim-problem-form} называют функцию
  \begin{equation}
    \label{eq:lagrange-form}
    \La(x, \lambda_0, \lambda) = \lambda_0 f(x) + \sum_{j=1}^m {\lambda_j g_j(x)}
  \end{equation}
  где $\lambda$ — вектор $\lambda_1, \dotsc, \lambda_m$. При
  $\lambda_0=1$ функция Лагранжа называется \neword{классической}.
\end{dfn}

\begin{thm}[Куна—Таккера]
  \label{th:kuhn-tucker}
  Пусть точка $\hat{x}$ является решением задачи
  \eqref{eq:cond-optim-problem-form} с функцией Лагранжа
  \eqref{eq:lagrange-form} при соответствующих $\lambda_0$ и
  $\lambda$. Тогда выполнены следующие условия:
  \begin{enumerate}
    \renewcommand{\labelenumi}{\emph{\asbuk{enumi})}}
  \item $\pardiff{\La}{x_i}=0,\, i=\overline{1,n}$ (условие
    стационарности функции Лагранжа)
  \item $\lambda_j \geq 0,\, j=\overline{1,m}$, если $\hat{x}$ — точка
    минимума, и $\lambda_j \leq 0$, если это точка максимума.
  \item $\lambda_j \mul g_j(x) = 0,\, j=\overline{1,m}$ (условие
    дополняющей нежёсткости)
  \item $g_j(x) \leq 0,\, j=\overline{1,m}$
  \item $\lambda_0^2 + \norm{\lambda}^2 > 0$ (условие нетривиальности решения)
  \end{enumerate}
\end{thm}

\begin{dfn}
  Точка $x^*$, удовлетворяющая условиям теоремы \ref{th:kuhn-tucker},
  называется \neword{условно-стационарной}. Если при этом $\lambda ≠
  0$, то $x^*$ называется \neword{регулярной} условно-стационарной
  точкой.
\end{dfn}

При использовании условий теоремы \ref{th:kuhn-tucker} для поиска
условно-стационарных точек рассматривают два варианта значений
$\lambda_0$ в функции Лагранажа: $\lambda_0=0$ и $\lambda_0 \neq 0$. В
последнем случае без ограничения общности обычно полагают $\lambda_0 =
1$.

\begin{thm}[Условие регулярности]
  \label{th:regular}
  Для того, чтобы в условно-стационарной точке $x^*$ выполнялось
  неравенство $\lambda_0 \neq 0$, достаточно линейной независимости
  градиентов активных в этой точке ограничений
  \begin{equation*}
    g_k'(x^*), g_{k+1}'(x^*), \dotsc, g_{k+l}'(x^*)
  \end{equation*}
\end{thm}

Если удаётся показать, что во \emph{всех} допустимых точках задачи
\eqref{eq:cond-optim-problem-form} выполнено условие регулярности,
случай $\lambda_0=0$ можно исключить из рассмотрения. Для этого также
удобно использовать следующее условие.

\begin{thm}[Условие Слейтера]
  \label{th:slater}
  Для $\lambda_0 \neq 0$ в условиях \ref{th:kuhn-tucker} достаточно
  существования такой точки $x_s$, в которой все неравенства
  ограничений выполняются строго: $g_j(x_s)<0, \, j=\overline{1,m}$.
\end{thm}

При использовании теоремы Куна—Таккера \ref{th:kuhn-tucker} в решении
задачи \eqref{eq:cond-optim-problem-form} важную роль могут сыграть
определённые свойства целевой функции $f(x)$ и допустимого множества
$D$. Рассмотрим их.

\begin{dfn}
  \label{dfn:convex-f}
  Функцию $f(x), x \in \set{R}^n$ называется \neword{выпуклой}, если
  для $\forall x, y \in \set{R}^n, \forall \alpha \in [0;1]$
  выполняется неравенство
  \begin{equation*}
    f(\alpha x + (1-\alpha)y) \leq \alpha f(x) + (1-\alpha) f(y)
  \end{equation*}
  Когда это неравенство $\forall \alpha \in (0;1)$ выполняется строго,
  функцию $f(x)$ называют \neword{строго выпуклой}.
\end{dfn}
\begin{rem}
  \label{rem:lin-f-convex}
  Линейные функции выпуклы, но не строго выпуклы.
\end{rem}

\begin{dfn}
  Функция $f(x)$ называется \neword{вогнутой}, если $-f(x)$ является
  выпуклой функцией.
\end{dfn}

\begin{dfn}
  Множество $D \subset \set{R}^n$ называется \neword{выпуклым}, если
  для $\forall x, y \in D, \forall \alpha \in [0;1]$ выполняется
  \begin{equation*}
    \alpha x + (1-\alpha) y \in D
  \end{equation*}
\end{dfn}
Таким образом, отрезок, соединяющий две точки выпуклого множества,
целиком принадлежит этому множеству.

Следущие две теоремы оказываются полезными при доказательстве
выпуклости функций и множеств.

\begin{thm}
  \label{th:convex-f}
  Если матрица Гессе квадратичной функции положительно определена, то
  эта функция выпукла.
\end{thm}

\begin{thm}
  \label{th:convex-set}
  Если $g_j(x), j=\overline{1,m}$ — выпуклые функции, то множество
  $D$, заданное условиями $g_j(x) \leq b_j$, является выпуклым.
\end{thm}

Понятия выпуклости и вогнутости функций и множеств были введены для
следующей теоремы, которая о случае \emph{достаточности} условий
теоремы \ref{th:kuhn-tucker}.

\begin{thm}
  \label{th:kt-cond}
  Для точек минимума необходимые условия Куна—Таккера становятся
  достаточными в случае, когда целевая функция $f(x)$ и ограниченное
  неравенствами $g_j(x)$ множество \emph{выпуклы}.

  В случае точек максимума достаточность достигается при
  \emph{вогнутости} функции $f(x)$ и \emph{выпуклости} множества
  допустимых решений задачи.
\end{thm}

Может оказаться, что наличие в определённой точке экстремума не
удаётся доказать, опираясь лишь на теоремы \ref{th:kuhn-tucker} и
\ref{th:kt-cond}. В таком случае задачу исследуют с применением
условий высших порядков, которые приведены далее.

В следующих теоремах используются полные дифференциалы функции
Лагранжа $d^2\La$ и ограничений $dg_j$, определяемые следующим
образом:
\begin{align*}
  d^2\La &= \sum_{p=1}^n\sum_{r=1}^n{\dpardiff{\La}{x_p}{x_r}\,dx_pdx_r} \\
  dg_j &= \sum_{p=1}^n{\pardiff{g_j}{x_p}\,dx_p}
\end{align*}

\begin{thm}[Необходимое условие экстремума второго порядка]
  \label{th:if-extr-2}
  Пусть точка $\hat{x}$ — регулярный экстремум в задаче
  \eqref{eq:cond-optim-problem-form}, удовлетворяющий условиям теоремы
  \ref{th:kuhn-tucker} при соответствующем $\lambda$. Тогда
  \begin{align*}
    d^2\La(\hat{x}) &\geq 0 \qquad \text{для минимума} \\
    d^2\La(\hat{x}) &\leq 0 \qquad \text{для максимума}
  \end{align*}
  при $dx ≠ 0$ таких, что
  \begin{align*}
    dg_j(\hat{x}) &= 0 \qquad \forall j: \lambda_j ≠ 0\\
    dg_j(\hat{x}) &\leq 0 \qquad \forall j: \lambda_j=0
  \end{align*}
\end{thm}

\begin{thm}[Достаточное условие экстремума первого порядка]
  \label{th:then-extr-1}
  Пусть $x^*$ является регулярной условно-стационарной точкой задачи
  \eqref{eq:cond-optim-problem-form}, а число активных в $x^*$
  ограничений равно числу переменных $n$. Тогда для наличия в $x^*$
  регулярного экстремума достаточно выполнения одного из следующих
  наборов неравенств при $\forall j: g_j(\hat{x}) = 0$:
  \begin{align*}
    \lambda_j &> 0 \qquad \text{для минимума} \\
    \lambda_j &< 0 \qquad \text{для максимума}
  \end{align*}
\end{thm}

\begin{thm}[Достаточное условие экстремума второго порядка]
  \label{th:then-extr-2}
  Пусть $x^*$ является регулярной условно-стационарной точкой задачи
  \eqref{eq:cond-optim-problem-form}. Если при $dx ≠ 0$ таких, что
    \begin{align*}
    dg_j(x^*) &= 0 \qquad \forall j: \lambda_j ≠ 0\\
    dg_j(x^*) &\leq 0 \qquad \forall j: \lambda_j=0
  \end{align*}
  выполняется неравенство $d^2\La(x^*) ≠ 0$, то $x^*$ — точка
  регулярного экстремума, причём
  \begin{align*}
    d^2\La(x^*) &> 0 \qquad \text{для минимума} \\
    d^2\La(x^*) &< 0 \qquad \text{для максимума}
  \end{align*}
\end{thm}

\subsubsection{Алгоритм метода}

Пользуясь введёнными в предыдущем разделе определениям и теоремами,
сформулируем алгоритм решения задачи
\eqref{eq:cond-optim-problem-form}. В нём после применения теоремы
Куна—Таккера точки добавляются к решению или исключаются из
дальнейшего рассмотрения по мере выполнения для них различных
достаточных условий или невыполнения необходимых, соответственно.

\begin{enumerate}
  \renewcommand{\labelenumi}{\textbf{Шаг \arabic{enumi}.}}
\item Составить функцию Лагранжа \eqref{eq:lagrange-form}. При
  выполнении условий теорем \ref{th:regular} или \ref{th:slater}
  положить $\lambda_0 = 0$.
\item Определить условно-стационарные точки из необходимых условий
  Куна—Таккера теоремы \ref{th:kuhn-tucker}.
\item Добавить к решению точки условного минимума или максимума, если
  для них условия Куна—Таккера оказываются достаточными согласно
  теореме \ref{th:kt-cond}.
\item Для оставшихся точек проверить достаточное условие первого
  порядка (теорема \ref{th:then-extr-1}). Точки, для которых оно
  выполняется, добавляются к решению. Затем к решению добавляются
  точки, для которых выполняется достаточное условие второго порядка
  (теорема \ref{th:then-extr-2}).
\item Для оставшихся точек проверить необходимое условие второго
  порядка (теорема \ref{th:if-extr-2}). Точки, для которых оно не
  выполняется, решениями задачи быть не могут и исключаются из
  рассмотрения.
\item Для оставшихся точек (то есть тех, для которых выполнены все
  необходимые, но ни одно из достаточных условий) требуется дальнейшее
  исследование.
\end{enumerate}

\clearpage
\subsubsection{Нахождение точного аналитического решения}

Рассмотрим методику применения изложенных в предыдущем разделе
теоретических соображений на практическом примере.

К решению предлагается следующая задача:
\begin{equation}
  \label{eq:cond-optim-problem-raw}
  \begin{cases}
    f(x) = (x_1+4)^2 + (x_2-4)^2 \to \extr \\
    2x_1 - x_2 \leq 2 \\
    x_1 \geq 0 \\
    x_2 \geq 0
  \end{cases}
\end{equation}

После приведения к каноническому виду она примет вид
\begin{equation}
  \label{eq:cond-optim-problem}
  \begin{cases}
    f(x) = (x_1+4)^2 + (x_2-4)^2 \to \extr \\
    g_1(x) = 2x_1 - x_2 - 2 \leq 0 \\
    g_2(x) = -x_1 \leq 0 \\
    g_3(x) = -x_2 \leq 0
  \end{cases}
\end{equation}

Заметим, что выполняется условие Слейтера \ref{th:slater}, так как в
качестве соответствующей точки $x_s$ можно взять, например, точку $(1,
1)$. Значит, в данной задаче достаточно рассмотреть лишь случай
классической функции Лагранжа с $\lambda_0=1$.

Составим функцию Лагранжа:
\begin{multline}
  \label{eq:lagrange}
  \La(x, \lambda) = (x_1+4)^2 + (x_2-4)^2 +\\
  + \lambda_1(2x_1-x_2-2)+\lambda_2(-x_1)+\lambda_3(-x_2)
\end{multline}

Запишем необходимые условия стационарности точки $x^*$ при
соответствующем векторе $\lambda$:
\begin{subequations}
  \renewcommand{\theequation}{\theparentequation\asbuk{equation}}
  \label{eq:kkt-conditions}
  \begin{equation}
    \label{eq:cond-stationary}
    \begin{cases}
      \pardiff{\La}{x_1} = 2(x^*_1+4)+2\lambda_1-\lambda_2=0\\
      \pardiff{\La}{x_2} = 2(x^*_2-4)-\lambda_1-\lambda_3=0
    \end{cases}
  \end{equation}
  \begin{equation}
    \label{eq:cond-sign}
    \sgn(\lambda_1) = \sgn(\lambda_2) = \sgn(\lambda_3)
  \end{equation}
  \begin{equation}
    \label{eq:cond-slackness}
    \begin{cases}
      \lambda_1\mul g_1(x^*) = \lambda_1\mul (2x^*_1-x^*_2-2) = 0\\
      \lambda_2\mul g_2(x^*) = \lambda_2\mul (-x^*_1) = 0\\
      \lambda_3\mul g_3(x^*) = \lambda_3\mul (-x^*_2) = 0
    \end{cases}
  \end{equation}
  \begin{equation}
    \begin{cases}
      \label{eq:cond-feasible}
      g_1(x^*) = 2x^*_1 - x^*_2 - 2 \leq 0 \\
      g_2(x^*) = -x^*_1 \leq 0 \\
      g_3(x^*) = -x^*_2 \leq 0
    \end{cases}
  \end{equation}
\end{subequations}

Рассмотрим $2³=8$ вариантов удовлетворения условий дополняющей
нежёсткости \eqref{eq:cond-slackness}.

\begin{enumerate}
\renewcommand{\labelenumi}{\Roman{enumi})}
\renewcommand{\labelenumiii}{\arabic{enumiii})}

\item $\lambda_1 = 0$
  
  В этом случае уравнения \eqref{eq:cond-stationary} принимают вид:
  \begin{equation}
    \label{eq:cond-stationary-l1=0}
    \begin{cases}
      2x^*_1+8-\lambda_2=0\\
      2x^*_2-8-\lambda_3=0
    \end{cases}
  \end{equation}
  \begin{enumerate}
  \item $\lambda_2 = 0$

    Из \eqref{eq:cond-stationary-l1=0} имеем $2x^*_1+8=0 \iff
    x^*_1=-4$, что не удовлетворяет условию $g_2(x^*) = -x^*_1 \leq 0$
    из \eqref{eq:cond-feasible}.
  \item $\lambda_2 ≠ 0$
    
    В этом случае из \eqref{eq:cond-slackness} следует, что
    $g_2(x^*)=0 \iff x^*_1 = 0$, откуда согласно
    \eqref{eq:cond-stationary-l1=0} получаем $\lambda_2=8$.
    \begin{enumerate}
    \item $\lambda_3 = 0$
      
      Из \eqref{eq:cond-stationary-l1=0} следует $x^*_2=4$. Получаем
      точку $A = (0, 4)$.
    \item $\lambda_3 ≠ 0$

      В данном случае согласно \eqref{eq:cond-slackness} находим
      $x^*_2=0$, поэтому из \eqref{eq:cond-stationary-l1=0} следует, что
      $\lambda_3=-8$. С учётом $\lambda_2=8$ заметим, что не
      выполняются условия \eqref{eq:cond-sign}.
    \end{enumerate}
  \end{enumerate}
\item $\lambda_1 ≠ 0$ 

  Согласно условию \eqref{eq:cond-slackness}, в
  данном случае
  \begin{equation}
    \label{eq:cond-slackness-l1n=0}
    2x^*_1-x^*_2-2=0
  \end{equation}
  \begin{enumerate}
  \item $\lambda_2 = 0$

    Из \eqref{eq:cond-stationary} получим
    \begin{equation}
      \label{eq:cond-stationary-l2=0}
      2x^*_1+8+2\lambda_1=0
    \end{equation}
    \begin{enumerate}
    \item $\lambda_3 = 0$

      Второе уравнение системы \eqref{eq:cond-stationary} даёт
      $2x^*_2-8-\lambda_1=0$. Сложим это уравнение с
      \eqref{eq:cond-stationary-l2=0} и рассмотрим его вместе с
      \eqref{eq:cond-slackness-l1n=0}, получив
      \begin{equation}
        \begin{cases}
          2x^*_1+4x^*_2-8=0\\
          2x^*_1-x^*_2-2=0
        \end{cases}
      \end{equation}
      
      Таким образом получим $4x^*_2-8=-x^*_2-2 \iff x^*_2=\frac{6}{5}$.

      Значение $x^*_1=\frac{8}{5}$ определяется из
      \eqref{eq:cond-slackness-l1n=0}. После этого из уравнения
      \eqref{eq:cond-stationary-l2=0} найдём значение $\lambda_1 =
      -\frac{28}{5}$. Итак, получена ещё одна точка $B =
      \left(\frac{8}{5}, \frac{6}{5}\right)$.
    \item $\lambda_3 ≠ 0$

      Согласно \eqref{eq:cond-slackness}, в данном случае $x^*_2=0$,
      поэтому из \eqref{eq:cond-slackness-l1n=0} следует $x^*_1=1$.
      Тогда из \eqref{eq:cond-stationary-l2=0} определим $\lambda_1 =
      -5$. Подставив найденные значения $x_2*$ и $\lambda_1$ в
      \eqref{eq:cond-stationary}, получим $\lambda_3 = -3$. Найдена
      очередная точка $C=(1, 0)$.
    \end{enumerate}
  \item $\lambda_2 ≠ 0$

    Из \eqref{eq:cond-slackness} получаем $x^*_1=0$, откуда с учётом
    \eqref{eq:cond-slackness-l1n=0} следует значение $x^*_2=-2$, не
    удовлетворяющее условию $g_3(x^*) = -x^*_2 \leq 0$ из
    \eqref{eq:cond-feasible}.
  \end{enumerate}
\end{enumerate}

Итак, найдены три точки, для которых выполнены необходимые условия
теоремы \ref{th:kuhn-tucker}. Тип возможного экстремума определяется
согласно знаку компонент $\lambda$.
\begin{itemize}
\item $A = (0, 4),\, \lambda=(0, 8, 0)$, минимум
\item $B = (\frac{8}{5}, \frac{6}{5}),\, \lambda=(-\frac{28}{5}, 0,
  0)$, максимум
\item $C = (1, 0),\, \lambda=(-5, 0, -3)$, максимум
\end{itemize}

\begin{figure}[!h]
  \centering
  \begin{tikzpicture}
    \begin{axis}[
      x=2.4 cm, y=1.3 cm,
      xmin=0, ymin=0, xmax=2.5,
      xlabel=$x_1$, ylabel=$x_2$,
      enlargelimits=0.05]
      \input{condextr-contours.tkz.tex}
      \addplot[very thick, red!50!black] coordinates{(0,5) (0,0) (1,0) (2.5,3)};
      \label{plot:boundaries}
      
      \node[circle,fill=black,scale=0.5,label=right:$A$] at (axis cs:0,4) {};
      \node[circle,fill=black,scale=0.5,label=below right:$B$] at (axis cs:1.6,1.2) {};
      \node[circle,fill=black,scale=0.5,label=right:$C$] at (axis cs:1,0) {};
    \end{axis}
  \end{tikzpicture}
  \caption[Границы допустимого множества, линии уровня целевой функции
  и условно-стационарные точки задачи условной оптимизации]{Границы
    допустимого множества, линии уровня целевой функции и
    условно-стационарные точки задачи \eqref{eq:cond-optim-problem}}
  \label{fig:cond-optim}
\end{figure}

Поскольку целевая функция обладает положительно определённую матрицу
Гессе $\left( \begin{smallmatrix}2 & 0 \\ 0 & 2\end{smallmatrix}
\right)$, она является выпуклой в силу теоремы \ref{th:convex-f}.
Рассматриваемое множество (см. рис. \ref{fig:cond-optim}) также
является выпуклым, так как функции ограничений $g_j(x)$ выпуклы в силу
замечания \ref{rem:lin-f-convex}, а потому выполняются условия теоремы
\ref{th:convex-set}.

Таким образом, согласно теореме \ref{th:kt-cond}, для точки $A$
условия Куна—Таккера являются \emph{достаточными}, и она является
точкой \emph{локального минимума}. Линии уровня функции $f(x)$,
сходящиеся к $A$ на рисунке \ref{fig:cond-optim}, служат
дополнительным подтверждением этому.

В то же время, для $B$ и $C$ условия Куна—Таккера не являются
достаточными. Воспользуемся для их исследования условиями высших
порядков.

Второй дифференциал $d^2\La$ во всех точках одинаковый и имеет вид
\begin{equation}
  \label{eq:la-diff}
  d^2\La = 2dx_1^2 + 2 dx_2^2
\end{equation}

\begin{itemize}
\item $B = (\frac{8}{5}, \frac{6}{5})$

  В данной точке активно лишь ограничение $g_1$, так что
  воспользоваться достаточным условием первого порядка нельзя.
  Проверим необходимое условие второго порядка. Приравнивая к нулю
  дифференциал $dg_1$ активного ограничения и выбирая дифференциалы
  неактивных ограничений неположительными, получим
  \begin{align*}
    dg_1 &= 2dx_1-dx_2 = 0 \iff dx_2 = 2dx_1 \\
    dg_2 &= -dx_1 \leq 0 \\
    dg_3 &= -dx_2 \leq 0
  \end{align*}
  и с учётом этого рассмотрим \eqref{eq:la-diff} при условии $dx≠0$:
  \begin{equation*}
    d^2\La = 2dx_1^2 + 8dx_1^2 = 10dx_1^2
  \end{equation*}
  
  Полученная форма очевидно больше нуля. Поскольку в $B$ значения
  $\lambda < 0$, необходимое условие второго порядка теоремы
  \ref{th:if-extr-2} не выполняется, так что $B$ \emph{не} является
  точкой экстремума.

\item $C = (1, 0)$

  Активными являются ограничения $g_1$ и $g_3$, их количество равно
  числу переменных $n=2$, поэтому согласно теореме
  \ref{th:then-extr-1} выполняется достаточное условие экстремума
  первого порядка, и $C$ есть точка \emph{локального максимума}.

  Действительно, из рисунка \ref{fig:cond-optim-zoom} видно, что любое
  допустимое перемещение из точки $C$ приводит к переходу с линии
  уровня $f(C)=41$ на линии меньших уровней. Вектор градиента $f'(C) =
  \left(\begin{smallmatrix}\phm 5\\-4\end{smallmatrix}\right)$ в
  данной точке образует с границами области тупые углы, так что
  возможные направления, вдоль которых функция растёт быстрее всего,
  будут направлениями убывания.
  
  \begin{figure}[!h]
    \centering
    \begin{tikzpicture}
      \begin{axis}[grid=both,x=15cm,y=15cm,
        xlabel=$x_1$, ylabel=$x_2$,
        enlargelimits=0.05]
        \input{condextr_zoom-contours.tkz.tex}
        
        \coordinate (C) at (axis cs:1,0);

        \draw[blue,thick,arrows=-triangle 45] (C) -- (axis cs:1.09375,-0.075);
        \addplot[very thick, red!50!black] coordinates{(0.85,0) (1,0) (1.1,0.2)};
        \node[circle,fill=black,scale=0.5,label=below:$C$] at (C) {};
      \end{axis}
    \end{tikzpicture}
    \caption[Задача \eqref{eq:cond-optim-problem} вблизи
    $C$]{Направление вектора градиента $f'(C)$, границы допустимого
      множества и линии уровня целевой функции $f(x)$ вблизи точки
      $C$}
    \label{fig:cond-optim-zoom}
  \end{figure}
\end{itemize}

\subsubsection{Использование метода}

Аналитический метод решения, построенный на использовании теоремы
Куна—Таккера \ref{th:kuhn-tucker}, позволяет точно определить условные
экстремумы.

В то же время, реализация данного метода на алгоритмическом языке
сопряжена с рядом проблем, основной из которых является концептуальная
сложность алгоритмов логического вывода, которые придётся использовать
при проверке условий теорем необходимости и достаточности. Кроме того,
в общем случае метод также требует решения систем нелинейных
уравнений.

Реализация аналитических машинных алгоритмов традиционно не является
простой задачей, и в большинстве случаев на практике оказывается более
выгодным (во всех отношениях) использование численных методов.

\clearpage
\subsection{Метод возможных направлений Зойтендейка}
\label{sec:zoutendijk}

\subsubsection{Общая схема методов возможных направлений}

Итерационные методы возможных направлений, к которым относится и метод
Зойтендейка, реализуют решение задачи
\eqref{eq:cond-optim-problem-form} по общей схеме, согласно которой
поиск начинается в допустимой точке пространства решений и
продолжается по траектории, обеспечивающей улучшение значения целевой
функции, но не выходящей за границы допустимой области.

На каждом шаге очередное приближение к точке минимума находится по
формуле
\begin{equation}
  \label{eq:pd-iter}
  x^{k+1} = x^k + s_k \mul d^k
\end{equation}

Различие между методами возможных направлений заключается в стратегии
выбора направления $d^k$ и шага $s_k$.

\subsubsection{Алгоритм метода Зойтендейка}

В методе Зойтендейка на каждой итерации направление и шаг выбираются
так, чтобы обеспечить наименьшую минимизирующую поправку к целевой
функции без нарушения какого-либо ограничения.

Пусть в \eqref{eq:cond-optim-problem-form} функции ограничений
$g_j(x)$ линейны или линеаризованы (с помощью разложения в ряд
Тейлора) так, что получена следующая задача (рассматривается случай
минимизации):
\begin{equation}
  \label{eq:zoutendijk-problem-form}
  \begin{cases}
    f(x) \to \min \\
    \scalmult{a^j,x} \leq b_j,\, j=\overline{1,m} \\
    x_i \geq 0,\, i=\overline{1,n} \\
    x \in \set{R}^n
  \end{cases}
\end{equation}

Теперь сформулируем алгоритм метода Зойтендейка.

Начальное приближение $x^1$ выбирается из числа точек, удовлетворяющих
ограничениям задачи \eqref{eq:zoutendijk-problem-form}.

\begin{enumerate}
  \renewcommand{\labelenumi}{\textbf{Шаг \arabic{enumi}.}}
\item Определяются множества номеров активных ограничений $I^k$ — для
  ограничений вида $x_i \geq 0$ и $J^k$ — для ограничений вида
  $\scalmult{a^j,x} \leq b_j$.
\item Если $I^k=J^k = \varnothing$, то $d^k=-f'(x^k)$. В противном
  случае решают задачу линейного программирования по минимизации
  скалярного произведения
  \begin{equation*}
    \scalmult{f'(x^k), d^k} \to \min
  \end{equation*}
  при наличии ограничений
  \begin{equation*}
    \scalmult{a^j, d^k} \leq 0
  \end{equation*}
  что соответствует поиску такого допустимого направления, вдоль
  которого целевая функция убывает быстрее всего.

  Для этого в векторе $d^k$ каждую $i$-ю координату
  представляют\footnote{Верхние индексы у $d_i, d_{i+}, d_{i-}$
    опущены для компактности} как $d_i$ для $i \in I^k$ и
  $d_{i+}-d_{i-}$ для $i \notin I^k$, так что решаемая задача
  приобретёт следующий вид:
  \begin{equation}
    \label{eq:zoutendijk-linprog}
    \begin{cases}
      \scalmult{f'(x^k), d^k} = 
      \suml_{i\in I^k}{\pardiff{f(x^k)}{x_i}d_i} + 
      \suml_{i \notin I^k}{\pardiff{f(x^k)}{x_i}(d_{i+}-d_{i-})}
      \to \min\\
      
      \scalmult{a^j, d^k} = 
      \suml_{i\in I^k}{a^j_id_i} +
      \suml_{i \notin I^k}{a^j_i(d_{i+}-d_{i-})} \leq 0,
      \, i\in J^k \\
      
      \suml_{i=1}^n\abs{d_i^k} = \suml_{i\in I^k}{d_i}+
      \suml_{i \notin I^k}{\left( d_{i+} + d_{i-}\right)} \leq 1\\
      
      d_{i+} \geq 0,\, d_{i-} \geq 0,\,d_{i+}\mul d_{i-}=0,\, i\notin I^k
    \end{cases}
  \end{equation}
\item Если выполняется условие
  \begin{equation}
    \label{eq:zoutendijk-halt}
    \scalmult{f'(x^k), d^k} = 0
  \end{equation}
  то дальнейшее улучшение решения невозможно и процесс прерывается.

  В противном случае определяют величину шага
  \begin{equation}
    \label{eq:zoutendijk-step}
    s_k =
    \min\{s_*,\hat{s}_1,\dotsc,\hat{s}_n,\bar{s}_1,\dotsc,\bar{s}_m\}
  \end{equation}
  где компоненты под $\min$ находятся по следующим правилам:
  \begin{itemize}
  \item $s_*$ — решение задачи безусловной одномерной оптимизации
    \begin{equation} 
      \label{eq:zoutendijk-1optim}
      f(x^k+s_* \mul d^k) \to \min
    \end{equation}

  \item $\hat{s}_i$ — наибольший шаг, при котором ограничение
    \begin{equation*}
      x^k+\hat{s}_i d^k_i \geq 0
    \end{equation*}
    остаётся удовлетворённым. Если компонента найденного вектора
    направления $d^k_i \geq 0$, то $\hat{s}_i = +\infty$ (так что
    соответствующая компонента $\hat{s}_j$ согласно
    \eqref{eq:zoutendijk-step} никак не ограничивает выбор $s_k$),
    иначе
    \begin{equation*}
      \hat{s}_i = -\frac{x^k_i}{d^k_i}
    \end{equation*}

  \item $\bar{s}_j$ — наибольший шаг, при котором при движении из
    $x^k$ по направлению $d_k$ ограничение
    \begin{equation*}
      \scalmult{a^j, x^k+\bar{s}_j d^k} \leq b_j
    \end{equation*}
    остаётся выполненным. Если $\scalmult{a^j, d^k} \leq 0$, то
    ограничение будет удовлетворено при любом шаге, поэтому $\bar{s}_j
    = +\infty$. В противном случае, $\bar{s}_j$ определяется как
    \begin{equation*}
      \bar{s}_j = \frac{b_j-\scalmult{a^j, x^k}}{\scalmult{a^j, d^k}}
    \end{equation*}
  \end{itemize}
  Отметим, что $s_k>0$.
\item С учётом шага $s_k$ и направления $d^k$ согласно
  \eqref{eq:pd-iter} полагают новое приближение равным
  \begin{equation*}
    x^{k+1} = x^k + s_k \mul d^k
  \end{equation*}
  и переходят к следующей итерации.
\end{enumerate}

Таким образом, критерием останова процесса оптимизации $k$-й итерации
является выполнение равенства \eqref{eq:zoutendijk-halt}, что
равносильно невозможности дальнейшего уменьшения значения целевой
функции при заданных ограничениях.

\subsubsection{Применение метода Зойтендейка}

Рассмотрим работу метода возможных направлений на примере задачи
\eqref{eq:cond-optim-problem-raw}. Будем искать точку минимума.
Ограничения уже линейны, поэтому дополнительная линеаризация не
требуется. Задача уже приведена к виду
\eqref{eq:zoutendijk-problem-form}:
\begin{equation}
  \label{eq:zoutendijk-problem}
  \begin{cases}
    f(x) = (x_1+4)^2 + (x_2-4)^2 \to \min \\
    2x_1-x_2 \leq 2 \\
    x_1 \geq 0 \\
    x_2 \geq 0
  \end{cases}
\end{equation}

\begin{enumerate}
  \renewcommand{\labelenumi}{\textbf{Итерация \arabic{enumi}.}}
  \renewcommand{\labelenumii}{\textbf{Шаг \arabic{enumii}.}}
\item
  В качестве первого приближения $x^1$ возьмём точку $(1, 1)$, очевидно
  удовлетворяющую всем ограничениям.
  \begin{enumerate}
  \item Ни одно из ограничений не активно в данной точке, поэтому
    множества $I^k = J^k = \varnothing$.
  \item Направление движения определим как противоположное вектору
    градиента:
    \begin{equation*}
      d^1 = -f(x^1) = \begin{pmatrix} -10 \\ \phm 6 \end{pmatrix}
    \end{equation*}
  \item Определим коэффициенты $s_*, \hat{s}_1, \hat{s}_2, \bar{s}_1$.
    \begin{itemize}
    \item Минимум функции $f(x^1+s_*\mul d^1) = (5-10s_*)^2+(6s_*-3)^2$
      достигается при $s_*=\frac{1}{2}$.
    \item $d^1_1 = -10 \ngeq 0 \implies \hat{s}_1 =
      -\frac{x^1_1}{d^1_1} = -\frac{1}{-10} = \frac{1}{10}$
    \item $d^1_2 = 6 \geq 0 \implies \hat{s}_2 = +\infty$
    \item $\scalmult{a^1, d^1} = -26 \leq
      0 \implies \bar{s}_1 = +\infty$
    \end{itemize}
    Находим величину шага: $s_1 = \min\{\frac{1}{2}, \frac{1}{10}\} =
    \frac{1}{10}$.
  \item Вычисляем новое приближение $x^2$:
    \begin{equation*}
      x^2 = \begin{pmatrix} 1 \\ 1 \end{pmatrix} +
      \frac{1}{10} \begin{pmatrix} -10 \\ \phm 6 \end{pmatrix}
      = \begin{pmatrix} 0 \\ \frac{8}{5} \end{pmatrix}
    \end{equation*}
  \end{enumerate}
\item Продолжим поиск минимума из точки
  \begin{equation*}
    x^2 = \begin{pmatrix} 0 \\ \frac{8}{5} \end{pmatrix}
  \end{equation*}
  \begin{enumerate}
  \item В $x^2$ активно лишь ограничение $x_1\geq 0$, поэтому $I^2 =
    \varnothing, J^2 = \{1\}$.
  \item Составим задачу линейного программирования:
    \begin{equation*}
      \begin{cases}
        \scalmult{f'(x^2), d^2} = \tilde{f}_2 =
        8d_1 - \frac{3}{5} (d_{2+}-d_{2-}) \to \min \\
        d_1+(d_{2+}+d_{2-}) \leq 1 \\
        d_1, d_{2+}, d_{2-} \geq 0 \\
        d_{2+} \mul d_{2-} = 0
      \end{cases}
    \end{equation*}
    Приведём её к каноническому виду, получив в первом ограничении
    равенство путём добавления переменной $x_1$:
    \begin{equation*}
      \begin{cases}
        \tilde{f}_2 = 8d_1-\frac{3}{5}(d_{2+}-d_{2-}) \to \min \\
        d_1+(d_{2+}+d_{2-}) + x_1 = 1 \\
        d_1, d_{2+}, d_{2-}, x_1 \geq 0 \\
        d_{2+} \mul d_{2-} = 0
      \end{cases}
    \end{equation*}
    Решим задачу симплекс-методом. Составим матрицу коэффициентов
    ограничений и целевой функции, затем исключим $x_1$ из состава
    базисных переменных:
    \begin{equation}
      \begin{gmatrix}[b]
        \dagger & d_1 & d_{2+} & d_{2-} & x_1 & \diamond\\
        x_1 & 1 & 1 & 1 & 1 & \mathbf{1} \\
        \tilde{f}_2 & -8 & \mathbf{\frac{3}{5}} & -\frac{3}{5} & 0 & 0
        \rowops \add[-\frac{3}{5}]{1}{2}
      \end{gmatrix}
    \end{equation}
    Получим матрицу с одними отрицательными коэффициентами в нижнем
    ряду:
    \begin{equation*}
      \begin{gmatrix}[b]
        \dagger & d_1 & d_{2+} & d_{2-} & x_1 & \diamond\\
        x_1 & 1 & 1 & 1 & 1 & 1 \\
        \tilde{f}_2 & -\frac{43}{5} & 0 & -\frac{6}{5} & -\frac{3}{5}
        & -\frac{3}{5}
      \end{gmatrix}
    \end{equation*}
    При этом условие $d_{2+} \mul d_{2-} = 0$ выполнено. Итак, найдено
    оптимальное решение задачи линейного программирования:
    \begin{equation*}
      \begin{cases}
        d_{1\phantom{+}} = 0 \\
        d_{2+} = 1 \\
        d_{2-} = 0
      \end{cases}
    \end{equation*}
    с учётом которого построим вектор $d^2$:
    \begin{equation*}
      d^2 = \begin{pmatrix} 0 \\ 1 \end{pmatrix}
    \end{equation*}
  \item Определим коэффициенты $s_*, \hat{s}_1, \hat{s}_2, \bar{s}_1$.
    \begin{itemize}
    \item Минимум $f(x^2+s_*\mul d^2) = 16+(s_*-\frac{32}{5})^2$
      достигается при $s_* = \frac{32}{5}$.
    \item $d^2_1 = 0 \geq 0 \implies \hat{s}_1 = +\infty$
    \item $d^2_1 = 1 \geq 0 \implies \hat{s}_1 = +\infty$
    \item $\scalmult{a^1, d^2} = -1 \leq 0 \implies \bar{s}_1 =
      +\infty$ Итак, ограничения никак не влияют на величину шага,
      поэтому $s_k = \frac{32}{5}$.
    \end{itemize}
  \item Найдём приближение $x^3$:
    \begin{equation*}
      x^3 = \begin{pmatrix} 0 \\ \frac{8}{5} \end{pmatrix} +
      \frac{32}{5} \begin{pmatrix} 0 \\ 1 \end{pmatrix}
      = \begin{pmatrix} 0 \\ 4 \end{pmatrix}
    \end{equation*}
  \end{enumerate}
\item Как известно из точного аналитического решения (см. раздел
  \ref{sec:kuhn-tucker}), точка $(0, 4)$ является точным решением
  задачи \eqref{eq:zoutendijk-problem}. Значит, удовлетворение
  \eqref{eq:zoutendijk-halt} на этой итерации должно сигнализировать
  об этом. Выполним первые два шага алгоритма, что убедиться в этом.
  \begin{enumerate}
  \item Как и на прошлой итерации, $I^2 = \varnothing, J^2 = \{1\}$.
  \item Для нахождения компонент вектора $d^k$ решим задачу линейного
    программирования:
    \begin{equation*}
      \begin{cases}
        \scalmult{f'(x^3), d^3} = \tilde{f}_3 =
        8d_1 \to \min \\
        d_1+(d_{2+}+d_{2-}) + x_1 = 1 \\
        d_1, d_{2+}, d_{2-}, x_1 \geq 0 \\
        d_{2+} \mul d_{2-} = 0
      \end{cases}
    \end{equation*}
    оптимальным решением которой является тривиальный набор
    \begin{equation*}
      \begin{cases}
        d_{1\phantom{+}} = 0 \\
        d_{2+} = 0 \\
        d_{2-} = 0
      \end{cases}
    \end{equation*}

    Таким образом,
    \begin{equation*}
      d^3 = \begin{pmatrix} 0 \\ 0 \end{pmatrix}
    \end{equation*}

    Очевидно, что выполняется критерий останова
    \eqref{eq:zoutendijk-halt}. Значит, далее уменьшать значение
    $f(x)$ в \eqref{eq:zoutendijk-problem} не представляется
    возможным.
  \end{enumerate}
\end{enumerate}

Итак, с помощью метода возможных направлений Зойтендейка найдено
решением задачи условной минимизации \eqref{eq:zoutendijk-problem},
которое согласуется с решением, найденным в \ref{sec:kuhn-tucker}.

\begin{figure}[!h]
  \centering
  \begin{tikzpicture}
    \begin{axis}[
      x=2.4 cm, y=1.3 cm,
      xmin=0, ymin=0, xmax=2.5,
      xlabel=$x_1$, ylabel=$x_2$,
      enlargelimits=0.05]
      \input{condextr-contours.tkz.tex}
      \addplot[ultra thick, dashed, red!50!black] coordinates{(0,5) (0,0) (1,0) (2.5,3)};
      
      \addplot[mark=*, only marks,
               mark options={fill=blue!70!black}] coordinates{(1,1)
                 (0,1) (0, 4)};
      
      \addplot[blue, thick, mark=none, arrows=-triangle 45] coordinates{(1,1) (0,1)};
      \addplot[blue, thick, mark=none, arrows=-triangle 45] coordinates{(0,1) (0,4)};      
    \end{axis}
  \end{tikzpicture}
  \caption[Метод Зойтендейка]{Ход решения задачи
    \eqref{eq:zoutendijk-problem} методом Зойтендейка}
  \label{fig:cond-optim}
\end{figure}

\subsubsection{Использование метода}

Метод Зойтендейка подходит для реализации на компьютере, поскольку не
содержит принципиально сложных шагов. Единственным проблемным этапом
может быть одномерная минимизация \eqref{eq:zoutendijk-1optim} для
нахождения значения $s_*$.

Линеаризация ограничений может вносить определённую погрешность в
решение.

Основным же ограничением методом Зойтендейка является невозможность
решения задач с ограничениями типа равенств.

\clearpage
\subsection{Метод штрафных функций}
\label{sec:penalty}

\subsubsection{Общие сведения}

Методы, построенные на применении штрафных функций, позволяют решать
задачи на поиск минимума при наличии ограничений \emph{численно}.

Общая схема таких методов заключается в замене задачи условной
минимизации в некоторой области $D$, заданной ограничениями типа
неравенств
\begin{equation*}
  \begin{cases}
    f(x) \to \min \\
    g_j(x) \leq 0,\, j=\overline{1,m} \\
    x \in \set{R}^n
  \end{cases}
\end{equation*}
на задачу безусловной минимизации
\begin{equation}
  \label{eq:penalty-problem-form}
  P(x, r) = f(x) + \phi(x, p) \to \min
\end{equation}
где \neword{штраф} $\phi(x, p)$ удовлетворяет свойству
\begin{align*}
  \phi(x, p) &= 0 \quad x \in D\\
  \phi(x, p) &\gg 0 \quad x \notin D
\end{align*}
Таким образом, штрафная часть значительно возрастает при выходе за
пределы допустимого множества $D$.

Один из подходов к использованию штрафных функций заключается в
последовательном решении задач вида \ref{eq:penalty-problem-form},
параметризованных номером итерации $k$:
\begin{equation}
  \label{eq:penalty-iter}
  P_k(x, r) = f(x) + \phi(x, p_k) \to \min  
\end{equation}
где функция $\phi(x, p_k)$ масштабируется на каждой итерации в сторону
увеличения или уменьшения.

\subsubsection{Метод внешних штрафных функций}

Методами внешних штрафных функций называют методы, использующие схему
\eqref{eq:penalty-iter} таким образом, что $\phi(x, p_k)$ с каждой
итерацией увеличивается, и положения вне допустимой области с ростом
$k$ становятся всё менее «выгодными». При этом начальное приближение
$x^1$ к точке минимума выбирается \emph{вне} $D$.

Один из таких методов, изложенный в \cite{himmelblau75},
предусматривает построение штрафной функции для задачи
\eqref{eq:cond-optim-problem-form} в виде:
\begin{equation}
  \label{eq:weisman}
  \phi(x, p_k) = p_k \sum_{j=1}^m{ \left [ g_j^+(x) \right ]^2}
\end{equation}
где $g_j^+(x) = \max\{0, g_j(x)\}$, так что штрафная часть отлична от
нуля вне границ, заданных ограничениями $g_j(x)$, а
$\lim\limits_{k\to\infty}{p_k} = +\infty$.

В качестве критерия останова процесса \eqref{eq:penalty-iter} можно
выбрать условие
\begin{equation*}
  \phi(x^k, p_k) < \epsilon
\end{equation*}
где $x^k$ — приближение к точке минимума на $k$-ом шаге.

\subsubsection{Пример}
\label{sec:penalty-usage}

В разделе \ref{sec:kuhn-tucker} была решена задача условной
оптимизации \eqref{eq:cond-optim-problem}:
\begin{equation*}
  \begin{cases}
    f(x) = (x_1+4)^2 + (x_2-4)^2 \to \extr \\
    g_1(x) = 2x_1 - x_2 - 2 \leq 0 \\
    g_2(x) = -x_1 \leq 0 \\
    g_3(x) = -x_2 \leq 0
  \end{cases}
\end{equation*}
Среди условных экстремумов функции был найден минимум в точке $A = (0,
4)$. Найдём его с помощью штрафных функций. Преобразуем задачу к виду
\label{eq:penalty-iter}, выбирая штрафную часть в виде
\eqref{eq:weisman}:
\begin{equation}
  \label{eq:penalty-problem}
  P(x) = f(x) + p \sum_{j=1}^3{ \left [ g_j^+(x) \right ]^2} \to \min
\end{equation}

На рисунках \ref{fig:pen-contours1}--\ref{fig:pen-contours4}
представлены линии уровня функции $P(x)$ задачи
\eqref{eq:penalty-problem} при различных значениях $p$ (при $p=0$
получается исходная функция $f(x)$ задачи
\eqref{eq:cond-optim-problem}). Видно, что с ростом $p$ влияние
штрафной части вне $D$ (\ref{plot:pen-boundaries}) значительно
возрастает, тогда как в допустимой области оптимизируемая функция не
претерпевает никаких изменений.

\pgfplotsset{every axis/.append style={xmin=-5,xmax=5,ymin=-5,ymax=5}}
\begin{figure}[!thb]
  \centering
  \begin{tikzpicture}
    \begin{axis}[xlabel=$x_1$, ylabel=$x_2$]
      \input{condextr_full-contours.tkz.tex}
      \addplot[thick,densely dashed, red!50!black] coordinates{(0,5)
        (0,0) (1,0) (3.5,5)};
      \label{plot:pen-boundaries}
    \end{axis}      
  \end{tikzpicture}
  \caption{Линии уровня функции $P(x)$ при $p=0$}
  \label{fig:pen-contours1}
\end{figure}

\begin{figure}[!thb]
  \centering
  \begin{tikzpicture}
    \begin{axis}[xlabel=$x_1$, ylabel=$x_2$]
      \input{penalty_mini-contours.tkz.tex}
      \addplot[thick,densely dashed, red!50!black] coordinates{(0,5)
        (0,0) (1,0) (3.5,5)};
    \end{axis}      
  \end{tikzpicture}
  \caption{Линии уровня функции $P(x)$ при $p=5$}
    \label{fig:pen-contours2}
\end{figure}

\begin{figure}[!thb]
  \centering
  \begin{tikzpicture}
    \begin{axis}[xlabel=$x_1$, ylabel=$x_2$]
      \input{penalty-contours.tkz.tex}
      \addplot[thick,densely dashed, red!50!black] coordinates{(0,5)
        (0,0) (1,0) (3.5,5)};
    \end{axis}      
  \end{tikzpicture}
  \caption{Линии уровня функции $P(x)$ при $p=50$}
  \label{fig:pen-contours3}
\end{figure}

\begin{figure}[!thb]
  \centering
  \begin{tikzpicture}
    \begin{axis}[xlabel=$x_1$, ylabel=$x_2$]
      \input{penalty_maxi-contours.tkz.tex}
      \addplot[thick,densely dashed, red!50!black] coordinates{(0,5)
        (0,0) (1,0) (3.5,5)};
    \end{axis}      
  \end{tikzpicture}
  \caption{Линии уровня функции $P(x)$ при $p=150$}
  \label{fig:pen-contours4}
\end{figure}

Для численного решения задачи \eqref{eq:penalty-problem} теперь уже
безусловной минимизации воспользуемся описанным ранее градиентным
методом с чебышёвскими функциями релаксации, выбирая в качестве
начального приближения точку вне области $D$.

Индекс $k$ в \eqref{eq:penalty-problem} опущен не случайно: опыт
показывает, что при достаточно большом начальном значении параметра
$p_0$ использование итерационного процесса \eqref{eq:penalty-iter}
может и не понадобиться, поскольку точка минимума $A$ с достаточной
точностью локализуется уже на первом шаге. Это проиллюстрировано на
рисунке \ref{fig:penalty}.

\begin{figure}[!thb]
  \centering
  \begin{tikzpicture}
    \begin{axis}[xlabel=$x_1$, ylabel=$x_2$, x=1 cm, y=1 cm]
      \addplot[thick,densely dashed, red!50!black] coordinates{(0,5)
        (0,0) (1,0) (3.5,5)};
      \input{penalty-contours.tkz.tex}

      \input{penalty_relch_-2,-3_20_14-trace.tkz.tex}
      \input{penalty_relch_2,-4_20_10-trace.tkz.tex}

      \node[circle,fill=black,scale=0.5,label=right:$A$] at (axis cs:0,4) {};
    \end{axis}      
  \end{tikzpicture}
  \caption[Метод штрафных функций]{Ход процесса минимизации
    \eqref{eq:penalty-problem} при $p=50$ и различном выборе начальной
    точки}
  \label{fig:penalty}
\end{figure}

\subsubsection{Использование метода}

Метод штрафных функций позволяет сводить задачи оптимизации при
наличии ограничений к задачам безусловной оптимизации, снижая таким
образом структурную сложность решаемой задачи. При этом, однако,
оптимизация функций \eqref{eq:penalty-iter} может стать проблематичной
ввиду высокой степени овражности вспомогательного функционала. В таком
случае целесообразным может оказаться использование методов типа
описанных в разделе \ref{sec:relch}.


\clearpage
\appendix
\section{Исходные тексты}
\label{sec:sources}

\subsection{Общая схема градиентных методов}
\input{gradient-methods.ss-full-listing.tex}

\subsection{Алгоритмы \relch{}, \gdrelch{}, \gd{} и \rgd{}}
\input{relch.ss-full-listing.tex}

\clearpage
\section{Информация о документе}

Данный документ был подготовлен с использованием \LaTeX{}. В качестве
реализации языка Scheme использовалась \program{PLT Scheme}.
Иллюстрации созданы с помощью пакета \program{pgfplots} и
\program{gnuplot}.

Автоматизация процесса сборки обеспечивалась утилитами
\program{GNU Make} и \program{texdepend}.

Представленная работа выполнена в рамках программы пятого семестра
обучения по специальности «Вычислительная математика и математическая
физика» в МГТУ им. Н. Э. Баумана.

Дата компиляции настоящего документа: \today

\newcommand{\BibEmph}{\name}
\bibliographystyle{gost71s}
\bibliography{paper}

\end{document}
