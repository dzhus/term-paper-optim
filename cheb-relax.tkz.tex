\begin{figure}[!h]
  \centering
  \begin{tikzpicture}
    \begin{axis}[x=7cm, y=2cm,
      xlabel=$\lambda$, ylabel=$R_s(\lambda)$,
      xtick=\empty, ytick={-1, 1},
      y tick label style={anchor=north east},
      y tick style={draw=none},
      extra x ticks={1},
      extra x tick style={grid=major},
      xmin=-0.2, xmax=1.4, ymin=-2, ymax=2,
      axis x line=middle,
      axis y line=middle,
      axis on top,
      mark size=3pt,
      pattern color=gray!50
      ]

      \addplot[mark=o] plot[domain=-0.2:1.1] function{(16*x**2-16*x+3)/3};
      \addplot[mark=square] plot[domain=-0.2:1.1] function{(-64*x**3+96*x**2-40*x+4)/4};
      \addplot[mark=triangle] plot[domain=-0.2:1.1] function{(256*x**4-512*x**3+336*x**2-80*x+5)/5};
      \legend{$L=3$, $L=4$, $L=5$}


      % Forbidden areas
      \addplot[draw=none,pattern=north west lines] coordinates{
        (-0.2,1) (0,1) (0,-1) (-0.2,-1)} \closedcycle;
      \addplot[draw=none,pattern=north west lines] coordinates{
        (0,1) (1.2,1) (1.2,1.4) (0,1.4)} \closedcycle;
      \addplot[draw=none,pattern=north west lines] coordinates{
        (0,-1) (1.2,-1) (1.2,-1.4) (0,-1.4)} \closedcycle;
      
      % Dashed lines at forbidden area boundaries
      \addplot[dashed] coordinates{
        (-0.2,1)
        (1.2,1)};
      \addplot[dashed] coordinates{
        (-0.2,-1)
        (1.2,-1)};
    \end{axis}
  \end{tikzpicture}
  \caption{Функции релаксации на основе полиномов Чебышёва}
  \label{fig:cheb-relax}
\end{figure}
