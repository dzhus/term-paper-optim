\subsection{Теорема Куна—Таккера}
\label{sec:kuhn-tucker}

На основе теоремы Куна—Таккера построен \emph{аналитический} метод
решения задачи \eqref{eq:cond-optim-problem-form}, позволяющий найти
её решения точно. 

В данной главе формулируется ряд определений и теорем, которые
используются в алгоритме метода, после чего рассматривается решение
модельной задачи с его помощью.

Более подробно тема освещена в \cite{alekseev05}, \cite{taha05} и
\cite{polovinkin04}.

\subsubsection{Теоретические сведения}

\begin{dfn} \neword{Функцией Лагранжа} задачи
  \eqref{eq:cond-optim-problem-form} называют функцию
  \begin{equation}
    \label{eq:lagrange-form}
    \La(x, \lambda_0, \lambda) = \lambda_0 f(x) + \sum_{j=1}^m {\lambda_j g_j(x)}
  \end{equation}
  где $\lambda$ — вектор $\lambda_1, \dotsc, \lambda_m$. При
  $\lambda_0=1$ функция Лагранжа называется \neword{классической}.
\end{dfn}

\begin{dfn}
  \label{dfn:active-constraint}
  Ограничение $g_k(x)$ задачи \eqref{eq:cond-optim-problem-form}
  называется \neword{активным} в точке $x_0$, если $g_k(x_0)=0$.
\end{dfn}

\begin{thm}[Куна—Таккера]
  \label{th:kuhn-tucker}
  Пусть точка $\hat{x}$ является решением задачи
  \eqref{eq:cond-optim-problem-form} с функцией Лагранжа
  \eqref{eq:lagrange-form} при соответствующих $\lambda_0$ и
  $\lambda$. Тогда выполнены следующие условия:
  \begin{enumerate}
    \renewcommand{\labelenumi}{\emph{\asbuk{enumi})}}
  \item $\pardiff{\La}{x_i}=0,\, i=\overline{1,n}$ (условие
    стационарности функции Лагранжа)
  \item $\lambda_j \geq 0,\, j=\overline{1,m}$, если $\hat{x}$ — точка
    минимума, и $\lambda_j \leq 0$, если это точка максимума.
  \item $\lambda_j \mul g_j(x) = 0,\, j=\overline{1,m}$ (условие
    дополняющей нежёсткости)
  \item $g_j(x) \leq 0,\, j=\overline{1,m}$
  \item $\lambda_0^2 + \norm{\lambda}^2 > 0$ (условие нетривиальности решения)
  \end{enumerate}
\end{thm}

\begin{dfn}
  Точка $x^*$, удовлетворяющая условиям теоремы \ref{th:kuhn-tucker},
  называется \neword{условно-стационарной}. Если при этом $\lambda ≠
  0$, то $x^*$ называется \neword{регулярной} условно-стационарной
  точкой.
\end{dfn}

При использовании условий теоремы \ref{th:kuhn-tucker} для поиска
условно-стационарных точек рассматривают два варианта значений
$\lambda_0$ в функции Лагранажа: $\lambda_0=0$ и $\lambda_0 \neq 0$. В
последнем случае без ограничения общности обычно полагают $\lambda_0 =
1$.

\begin{thm}[Условие регулярности]
  \label{th:regular}
  Для того, чтобы в условно-стационарной точке $x^*$ выполнялось
  неравенство $\lambda_0 \neq 0$, достаточно линейной независимости
  градиентов активных в этой точке ограничений
  \begin{equation*}
    g_k'(x^*), g_{k+1}'(x^*), \dotsc, g_{k+l}'(x^*)
  \end{equation*}
\end{thm}

Если удаётся показать, что во \emph{всех} допустимых точках задачи
\eqref{eq:cond-optim-problem-form} выполнено условие регулярности,
случай $\lambda_0=0$ можно исключить из рассмотрения. Для этого также
удобно использовать следующее условие.

\begin{thm}[Условие Слейтера]
  \label{th:slater}
  Для $\lambda_0 \neq 0$ в условиях теоремы \ref{th:kuhn-tucker}
  достаточно существования такой точки $x_s$, в которой все
  неравенства ограничений выполняются строго: $g_j(x_s)<0, \,
  j=\overline{1,m}$.
\end{thm}

Введённые в разделе \ref{sec:convexity} понятия выпуклости и
вогнутости функций и множеств использутся в следующей теореме, в
которой говорится о случае \emph{достаточности} условий теоремы
\ref{th:kuhn-tucker}.

\begin{thm}
  \label{th:kt-cond}
  Для точек минимума необходимые условия Куна—Таккера становятся
  достаточными в случае, когда целевая функция $f(x)$ и ограниченное
  неравенствами $g_j(x)$ множество \emph{выпуклы}.

  В случае точек максимума достаточность достигается при
  \emph{вогнутости} функции $f(x)$ и \emph{выпуклости} множества
  допустимых решений задачи.
\end{thm}

Может оказаться, что наличие в определённой точке экстремума не
удаётся доказать, опираясь лишь на теоремы \ref{th:kuhn-tucker} и
\ref{th:kt-cond}. В таком случае задачу исследуют с применением
условий высших порядков, которые приведены далее.

В следующих теоремах используются полные дифференциалы функции
Лагранжа $d^2\La$ и ограничений $dg_j$, определяемые следующим
образом:
\begin{align*}
  d^2\La &= \sum_{p=1}^n\sum_{r=1}^n{\dpardiff{\La}{x_p}{x_r}\,dx_pdx_r} \\
  dg_j &= \sum_{p=1}^n{\pardiff{g_j}{x_p}\,dx_p}
\end{align*}

\begin{thm}[Необходимое условие экстремума второго порядка]
  \label{th:if-extr-2}
  Пусть точка $\hat{x}$ — регулярный экстремум в задаче
  \eqref{eq:cond-optim-problem-form}, удовлетворяющий условиям теоремы
  \ref{th:kuhn-tucker} при соответствующем $\lambda$. Тогда
  \begin{align*}
    d^2\La(\hat{x}) &\geq 0 \qquad \text{для минимума} \\
    d^2\La(\hat{x}) &\leq 0 \qquad \text{для максимума}
  \end{align*}
  при $dx ≠ 0$ таких, что
  \begin{align*}
    dg_j(\hat{x}) &= 0 \qquad \forall j: \lambda_j ≠ 0\\
    dg_j(\hat{x}) &\leq 0 \qquad \forall j: \lambda_j=0
  \end{align*}
\end{thm}

\begin{thm}[Достаточное условие экстремума первого порядка]
  \label{th:then-extr-1}
  Пусть $x^*$ является регулярной условно-стационарной точкой задачи
  \eqref{eq:cond-optim-problem-form}, а число активных в $x^*$
  ограничений равно числу переменных $n$. Тогда для наличия в $x^*$
  регулярного экстремума достаточно выполнения одного из следующих
  наборов неравенств при $\forall j: g_j(\hat{x}) = 0$:
  \begin{align*}
    \lambda_j &> 0 \qquad \text{для минимума} \\
    \lambda_j &< 0 \qquad \text{для максимума}
  \end{align*}
\end{thm}

\begin{thm}[Достаточное условие экстремума второго порядка]
  \label{th:then-extr-2}
  Пусть $x^*$ является регулярной условно-стационарной точкой задачи
  \eqref{eq:cond-optim-problem-form}. Если при $dx ≠ 0$ таких, что
    \begin{align*}
    dg_j(x^*) &= 0 \qquad \forall j: \lambda_j ≠ 0\\
    dg_j(x^*) &\leq 0 \qquad \forall j: \lambda_j=0
  \end{align*}
  выполняется неравенство $d^2\La(x^*) ≠ 0$, то $x^*$ — точка
  регулярного экстремума, причём
  \begin{align*}
    d^2\La(x^*) &> 0 \qquad \text{для минимума} \\
    d^2\La(x^*) &< 0 \qquad \text{для максимума}
  \end{align*}
\end{thm}

\subsubsection{Алгоритм метода}

Пользуясь введёнными в предыдущем разделе определениям и теоремами,
сформулируем алгоритм решения задачи
\eqref{eq:cond-optim-problem-form}. В нём после применения теоремы
Куна—Таккера точки добавляются к решению или исключаются из
дальнейшего рассмотрения по мере выполнения для них различных
достаточных условий или невыполнения необходимых, соответственно.

\begin{enumerate}
  \renewcommand{\labelenumi}{\textbf{Шаг \arabic{enumi}.}}
\item Составить функцию Лагранжа \eqref{eq:lagrange-form}. При
  выполнении условий теорем \ref{th:regular} или \ref{th:slater}
  положить $\lambda_0 = 0$.
\item Определить условно-стационарные точки из необходимых условий
  Куна—Таккера теоремы \ref{th:kuhn-tucker}.
\item Добавить к решению точки условного минимума или максимума, если
  для них условия Куна—Таккера оказываются достаточными согласно
  теореме \ref{th:kt-cond}.
\item Для оставшихся точек проверить достаточное условие первого
  порядка (теорема \ref{th:then-extr-1}). Точки, для которых оно
  выполняется, добавляются к решению. Затем к решению добавляются
  точки, для которых выполняется достаточное условие второго порядка
  (теорема \ref{th:then-extr-2}).
\item Для оставшихся точек проверить необходимое условие второго
  порядка (теорема \ref{th:if-extr-2}). Точки, для которых оно не
  выполняется, решениями задачи быть не могут и исключаются из
  рассмотрения.
\item Для оставшихся точек (то есть тех, для которых выполнены все
  необходимые, но ни одно из достаточных условий) требуется дальнейшее
  исследование.
\end{enumerate}

\clearpage
\subsubsection{Нахождение точного аналитического решения}
\label{sec:kuhn-tucker-solution}

Рассмотрим методику применения изложенных в предыдущем разделе
теоретических соображений на практическом примере.

К решению предлагается следующая задача:
\begin{equation}
  \label{eq:cond-optim-problem-raw}
  \begin{cases}
    f(x) = (x_1+4)^2 + (x_2-4)^2 \to \extr \\
    2x_1 - x_2 \leq 2 \\
    x_1 \geq 0 \\
    x_2 \geq 0
  \end{cases}
\end{equation}

После приведения к каноническому виду она примет вид
\begin{equation}
  \label{eq:cond-optim-problem}
  \begin{cases}
    f(x) = (x_1+4)^2 + (x_2-4)^2 \to \extr \\
    g_1(x) = 2x_1 - x_2 - 2 \leq 0 \\
    g_2(x) = -x_1 \leq 0 \\
    g_3(x) = -x_2 \leq 0
  \end{cases}
\end{equation}

Заметим, что выполняется условие Слейтера \ref{th:slater}, так как в
качестве соответствующей точки $x_s$ можно взять, например, точку $(1,
1)$. Значит, в данной задаче достаточно рассмотреть лишь случай
классической функции Лагранжа с $\lambda_0=1$.

Составим функцию Лагранжа:
\begin{multline}
  \label{eq:lagrange}
  \La(x, \lambda) = (x_1+4)^2 + (x_2-4)^2 +\\
  + \lambda_1(2x_1-x_2-2)+\lambda_2(-x_1)+\lambda_3(-x_2)
\end{multline}

Запишем необходимые условия стационарности точки $x^*$ при
соответствующем векторе $\lambda$:
\begin{subequations}
  \renewcommand{\theequation}{\theparentequation\asbuk{equation}}
  \label{eq:kkt-conditions}
  \begin{equation}
    \label{eq:cond-stationary}
    \begin{cases}
      \pardiff{\La}{x_1} = 2(x^*_1+4)+2\lambda_1-\lambda_2=0\\
      \pardiff{\La}{x_2} = 2(x^*_2-4)-\lambda_1-\lambda_3=0
    \end{cases}
  \end{equation}
  \begin{equation}
    \label{eq:cond-sign}
    \sgn(\lambda_1) = \sgn(\lambda_2) = \sgn(\lambda_3)
  \end{equation}
  \begin{equation}
    \label{eq:cond-slackness}
    \begin{cases}
      \lambda_1\mul g_1(x^*) = \lambda_1\mul (2x^*_1-x^*_2-2) = 0\\
      \lambda_2\mul g_2(x^*) = \lambda_2\mul (-x^*_1) = 0\\
      \lambda_3\mul g_3(x^*) = \lambda_3\mul (-x^*_2) = 0
    \end{cases}
  \end{equation}
  \begin{equation}
    \begin{cases}
      \label{eq:cond-feasible}
      g_1(x^*) = 2x^*_1 - x^*_2 - 2 \leq 0 \\
      g_2(x^*) = -x^*_1 \leq 0 \\
      g_3(x^*) = -x^*_2 \leq 0
    \end{cases}
  \end{equation}
\end{subequations}

Рассмотрим $2³=8$ вариантов удовлетворения условий дополняющей
нежёсткости \eqref{eq:cond-slackness}.

\begin{enumerate}
\renewcommand{\labelenumi}{\Roman{enumi})}
\renewcommand{\labelenumiii}{\arabic{enumiii})}

\item $\lambda_1 = 0$
  
  В этом случае уравнения \eqref{eq:cond-stationary} принимают вид:
  \begin{equation}
    \label{eq:cond-stationary-l1=0}
    \begin{cases}
      2x^*_1+8-\lambda_2=0\\
      2x^*_2-8-\lambda_3=0
    \end{cases}
  \end{equation}
  \begin{enumerate}
  \item $\lambda_2 = 0$

    Из \eqref{eq:cond-stationary-l1=0} имеем $2x^*_1+8=0 \iff
    x^*_1=-4$, что не удовлетворяет условию $g_2(x^*) = -x^*_1 \leq 0$
    из \eqref{eq:cond-feasible}.
  \item $\lambda_2 ≠ 0$
    
    В этом случае из \eqref{eq:cond-slackness} следует, что
    $g_2(x^*)=0 \iff x^*_1 = 0$, откуда согласно
    \eqref{eq:cond-stationary-l1=0} получаем $\lambda_2=8$.
    \begin{enumerate}
    \item $\lambda_3 = 0$
      
      Из \eqref{eq:cond-stationary-l1=0} следует $x^*_2=4$. Получаем
      точку $A = (0, 4)$.
    \item $\lambda_3 ≠ 0$

      В данном случае согласно \eqref{eq:cond-slackness} находим
      $x^*_2=0$, поэтому из \eqref{eq:cond-stationary-l1=0} следует, что
      $\lambda_3=-8$. С учётом $\lambda_2=8$ заметим, что не
      выполняются условия \eqref{eq:cond-sign}.
    \end{enumerate}
  \end{enumerate}
\item $\lambda_1 ≠ 0$ 

  Согласно условию \eqref{eq:cond-slackness}, в
  данном случае
  \begin{equation}
    \label{eq:cond-slackness-l1n=0}
    2x^*_1-x^*_2-2=0
  \end{equation}
  \begin{enumerate}
  \item $\lambda_2 = 0$

    Из \eqref{eq:cond-stationary} получим
    \begin{equation}
      \label{eq:cond-stationary-l2=0}
      2x^*_1+8+2\lambda_1=0
    \end{equation}
    \begin{enumerate}
    \item $\lambda_3 = 0$

      Второе уравнение системы \eqref{eq:cond-stationary} даёт
      $2x^*_2-8-\lambda_1=0$. Сложим это уравнение с
      \eqref{eq:cond-stationary-l2=0} и рассмотрим его вместе с
      \eqref{eq:cond-slackness-l1n=0}, получив
      \begin{equation}
        \begin{cases}
          2x^*_1+4x^*_2-8=0\\
          2x^*_1-x^*_2-2=0
        \end{cases}
      \end{equation}
      
      Таким образом получим $4x^*_2-8=-x^*_2-2 \iff x^*_2=\frac{6}{5}$.

      Значение $x^*_1=\frac{8}{5}$ определяется из
      \eqref{eq:cond-slackness-l1n=0}. После этого из уравнения
      \eqref{eq:cond-stationary-l2=0} найдём значение $\lambda_1 =
      -\frac{28}{5}$. Итак, получена ещё одна точка $B =
      \left(\frac{8}{5}, \frac{6}{5}\right)$.
    \item $\lambda_3 ≠ 0$

      Согласно \eqref{eq:cond-slackness}, в данном случае $x^*_2=0$,
      поэтому из \eqref{eq:cond-slackness-l1n=0} следует $x^*_1=1$.
      Тогда из \eqref{eq:cond-stationary-l2=0} определим $\lambda_1 =
      -5$. Подставив найденные значения $x_2*$ и $\lambda_1$ в
      \eqref{eq:cond-stationary}, получим $\lambda_3 = -3$. Найдена
      очередная точка $C=(1, 0)$.
    \end{enumerate}
  \item $\lambda_2 ≠ 0$

    Из \eqref{eq:cond-slackness} получаем $x^*_1=0$, откуда с учётом
    \eqref{eq:cond-slackness-l1n=0} следует значение $x^*_2=-2$, не
    удовлетворяющее условию $g_3(x^*) = -x^*_2 \leq 0$ из
    \eqref{eq:cond-feasible}.
  \end{enumerate}
\end{enumerate}

Итак, найдены три точки, для которых выполнены необходимые условия
теоремы \ref{th:kuhn-tucker}. Тип возможного экстремума определяется
согласно знаку компонент $\lambda$.
\begin{itemize}
\item $A = (0, 4),\, \lambda=(0, 8, 0)$, минимум
\item $B = (\frac{8}{5}, \frac{6}{5}),\, \lambda=(-\frac{28}{5}, 0,
  0)$, максимум
\item $C = (1, 0),\, \lambda=(-5, 0, -3)$, максимум
\end{itemize}

\begin{figure}[!h]
  \centering
  \begin{tikzpicture}
    \begin{axis}[
      x=2.4 cm, y=1.3 cm,
      xmin=0, ymin=0, xmax=2.5,
      xlabel=$x_1$, ylabel=$x_2$,
      enlargelimits=0.05]
      \input{condextr-contours.tkz.tex}
      \addplot[very thick, red!50!black] coordinates{(0,5) (0,0) (1,0) (2.5,3)};
      \label{plot:boundaries}
      
      \node[circle,fill=black,scale=0.5,label=right:$A$] at (axis cs:0,4) {};
      \node[circle,fill=black,scale=0.5,label=below right:$B$] at (axis cs:1.6,1.2) {};
      \node[circle,fill=black,scale=0.5,label=right:$C$] at (axis cs:1,0) {};
    \end{axis}
  \end{tikzpicture}
  \caption[Границы допустимого множества, линии уровня целевой функции
  и условно-стационарные точки задачи условной оптимизации]{Границы
    допустимого множества, линии уровня целевой функции и
    условно-стационарные точки задачи \eqref{eq:cond-optim-problem}}
  \label{fig:cond-optim}
\end{figure}

Поскольку целевая функция обладает положительно определённой матрицей
Гессе
\begin{equation}
  \label{eq:cond-optim-hess}
  \begin{pmatrix}
    2 & 0 \\
    0 & 2
  \end{pmatrix}
\end{equation}
она является выпуклой в силу теоремы \ref{th:convex-f-hess}.
Рассматриваемое множество (см. рис. \ref{fig:cond-optim}) также
является выпуклым, так как функции ограничений $g_j(x)$ выпуклы в силу
замечания \ref{rem:lin-f-convex}, а потому выполняются условия теоремы
\ref{th:convex-set}.

Таким образом, согласно теореме \ref{th:kt-cond}, для точки $A$
условия Куна—Таккера являются \emph{достаточными}, и она является
точкой \emph{локального минимума}. Линии уровня функции $f(x)$,
сходящиеся к $A$ на рисунке \ref{fig:cond-optim}, служат
дополнительным подтверждением этому.

В то же время, для $B$ и $C$ условия Куна—Таккера не являются
достаточными. Воспользуемся для их исследования условиями высших
порядков.

Второй дифференциал $d^2\La$ во всех точках одинаковый и имеет вид
\begin{equation}
  \label{eq:la-diff}
  d^2\La = 2dx_1^2 + 2 dx_2^2
\end{equation}

\begin{itemize}
\item $B = (\frac{8}{5}, \frac{6}{5})$

  В данной точке активно лишь ограничение $g_1$, так что
  воспользоваться достаточным условием первого порядка нельзя.
  Проверим необходимое условие второго порядка. Приравнивая к нулю
  дифференциал $dg_1$ активного ограничения и выбирая дифференциалы
  неактивных ограничений неположительными, получим
  \begin{align*}
    dg_1 &= 2dx_1-dx_2 = 0 \iff dx_2 = 2dx_1 \\
    dg_2 &= -dx_1 \leq 0 \\
    dg_3 &= -dx_2 \leq 0
  \end{align*}
  и с учётом этого рассмотрим \eqref{eq:la-diff} при условии $dx≠0$:
  \begin{equation*}
    d^2\La = 2dx_1^2 + 8dx_1^2 = 10dx_1^2
  \end{equation*}
  
  Полученная форма очевидно больше нуля. Поскольку в $B$ значения
  $\lambda < 0$, необходимое условие второго порядка теоремы
  \ref{th:if-extr-2} не выполняется, так что $B$ \emph{не} является
  точкой экстремума.

\item $C = (1, 0)$

  Активными являются ограничения $g_1$ и $g_3$, их количество равно
  числу переменных $n=2$, поэтому согласно теореме
  \ref{th:then-extr-1} выполняется достаточное условие экстремума
  первого порядка, и $C$ есть точка \emph{локального максимума}.

  Действительно, из рисунка \ref{fig:cond-optim-zoom} видно, что любое
  допустимое перемещение из точки $C$ приводит к переходу с линии
  уровня $f(C)=41$ на линии меньших уровней. Вектор градиента $f'(C) =
  \left(\begin{smallmatrix}\phm 5\\-4\end{smallmatrix}\right)$ в
  данной точке образует с границами области тупые углы, так что
  возможные направления, вдоль которых функция растёт быстрее всего,
  будут направлениями убывания.
  
  \begin{figure}[!h]
    \centering
    \begin{tikzpicture}
      \begin{axis}[grid=both,x=15cm,y=15cm,
        xlabel=$x_1$, ylabel=$x_2$,
        enlargelimits=0.05]
        \input{condextr_zoom-contours.tkz.tex}
        
        \coordinate (C) at (axis cs:1,0);

        \draw[blue,thick,arrows=-triangle 45] (C) -- (axis cs:1.09375,-0.075);
        \addplot[very thick, red!50!black] coordinates{(0.85,0) (1,0) (1.1,0.2)};
        \node[circle,fill=black,scale=0.5,label=below:$C$] at (C) {};
      \end{axis}
    \end{tikzpicture}
    \caption[Задача \eqref{eq:cond-optim-problem} вблизи
    $C$]{Направление вектора градиента $f'(C)$, границы допустимого
      множества и линии уровня целевой функции $f(x)$ вблизи точки
      $C$}
    \label{fig:cond-optim-zoom}
  \end{figure}
\end{itemize}

\subsubsection{Использование метода}

Аналитический метод решения, построенный на использовании теоремы
Куна—Таккера \ref{th:kuhn-tucker}, позволяет точно определить условные
экстремумы.

В то же время, реализация данного метода на алгоритмическом языке
сопряжена с рядом проблем, основной из которых является концептуальная
сложность алгоритмов логического вывода, которые придётся использовать
при проверке условий теорем необходимости и достаточности. Кроме того,
в общем случае метод также требует решения систем нелинейных
уравнений.

Реализация аналитических машинных алгоритмов традиционно не является
простой задачей, и в большинстве случаев на практике оказывается более
выгодным (во всех отношениях) использование численных методов.
